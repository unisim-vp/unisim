\section{User guide}

\subsection{Simulator features}

The TMS320C3X is a 32-bit floating-point DSP from Texas Instrument.
The UNISIM TMS320C3X simulator features:
\begin{itemize}
\item Simulation of the TMS320C3X Instruction set
\item A simulation speed average around 10 MIPS and up to 28 MIPS on a 2.4 Ghz Core2 Duo machine under Linux
\item Support for instruction cache
\item Support for TI COFF v0, v1, and v2
\item Built-in debugger (Inline Debugger)
\item Support for GDB serial remote protocol (GDB server)
\item Support for TI C I/O
\end{itemize}

\subsection{Compiling the simulator}

Up-to-date instructions for compiling the simulator are available in file \texttt{INSTALL}.

\subsection{Invoking the simulator}

The general command line format for invoking the simulator is the following:

\begin{verbatim}
tms320c3x [<options>] <binary to simulate>
\end{verbatim}

\noindent The binary to simulate must be a TI's COFF v0, v1 or v2 file. See~\ref{tms320c3x_cross_compiler} to generate such files.
\newline\\
\noindent The command line options of the simulator are:

\begin{itemize}
\item \texttt{--help} or \texttt{-h}: print integrated help message on the screen.
\item \texttt{--get-variables <XML file>} or \texttt{-v <XML file>}: Get the simulation variables (parameters, statistics, \ldots) into a XML file
\item \texttt{--config <XML file>} or \texttt{-c <XML file>}: configures the simulator with the given xml configuration file.
\item \texttt{--get-config <XML file>} or \texttt{-g <XML file>}: Get the simulator default configuration XML file (you can use it to create your own configuration). This option can be combined with option \texttt{-c} to get a new configuration file with existing variables from another file.
\item \texttt{--logger} or \texttt{-l}: Enable the logger.
\item \texttt{--xml-gdb <file>} or \texttt{-x <file>}: processor XML description file for GDB server.
\item \texttt{--gdb-server <port\_number>} or \texttt{-d <port\_number>}: Enable the GDB server and use the given TCP/IP port for communicating with a GDB client.
\item \texttt{--inline-debugger} or \texttt{-i}: Enable the inline debugger (only active if logger option used).
\end{itemize}

\subsection{The Texas Instrument cross-compiler for TMS320C3X}
\label{tms320c3x_cross_compiler}

To compile programs for the TMS320C3X simulator, you can use the free evaluation cross-compiler for TMS320C3X running on a Windows host (SPRC147, TMS320C3x DSK Software) available at \url{http://focus.ti.com/docs/toolsw/folders/print/tmdsdsk33.html}.
This cross-compiler also runs under other x86 operating systems such as Linux or MacOSX using Wine, a Windows emulator (\url{http://www.winehq.org/}).

The cross-compiler tool chain (\texttt{CL30.EXE, LNK30.EXE, ASM30.EXE, MK30.EXE, AR30.EXE, \ldots}) should be in your \texttt{PATH}, for instance:
\begin{verbatim}
set PATH=C:\TI\TIC3X4X\BIN;%PATH%
\end{verbatim}

The shell variable \texttt{C\_DIR} points to the location where cross-compiler should search for the standard C headers and libraries, for instance:
\begin{verbatim}
set C_DIR=C:\TI\TIC3X4X\INCLUDE;C:\TI\TIC3X4X\LIB
\end{verbatim}

Be aware that any call to the C standard library requires linking the program with \texttt{RTS30.LIB}.
Moreover, any call to I/O functions (open, close, read, write, printf, \ldots) requires TI C I/O support enabled in the TMS320C3X simulator.

\subsection{Simulator configuration}

\noindent The simulator stores its configuration (a set of parameters) in a XML configuration file. 
\newline\\
\noindent The simulator can provide the user with a default XML configuration file with option \texttt{-g}:

\begin{verbatim}
$ tms320c3x -g default_sim_config.xml
\end{verbatim}

\noindent The simulator can loads a XML configuration file with option \texttt{-c}:

\begin{verbatim}
$ tms320c3x -c sim_config.xml
\end{verbatim}

\newpage
\noindent The available parameters are summarized in table below:

\begin{center}
	\begin{tabular}{|p{7.5cm}|p{7.5cm}|}
	\hline
	\multicolumn{2}{|c|}{\textbf{\large CPU run-time parameters}}\\
	\hline
	\multicolumn{1}{|p{7.5cm}}{\textbf{Name:} \texttt{cpu.max-inst}} & \multicolumn{1}{p{7.5cm}|}{\textbf{Type:} \texttt{64-bit unsigned integer}}\\
	\multicolumn{2}{|p{15cm}|}{\textbf{Default value:} \texttt{$2^{64} - 1$}}\\
	\multicolumn{2}{|l|}{}\\
	\multicolumn{2}{|p{15cm}|}{\textbf{Description:} \newline Maximum number of instructions to execute before stopping simulation.}\\
	\hline
	\multicolumn{1}{|p{7.5cm}}{\textbf{Name:} \texttt{cpu.trap-on-instruction-counter}} & \multicolumn{1}{p{7.5cm}|}{\textbf{Type:} \texttt{64-bit unsigned integer}}\\
	\multicolumn{2}{|p{15cm}|}{\textbf{Default value:} \texttt{$2^{64} - 1$}}\\
	\multicolumn{2}{|l|}{}\\
	\multicolumn{2}{|p{15cm}|}{\textbf{Description:} \newline Number of instructions to execute before leaving control to the debugger.}\\
	\hline
	\multicolumn{1}{|p{7.5cm}}{\textbf{Name:} \texttt{cpu.verbose-all}} & \multicolumn{1}{p{7.5cm}|}{\textbf{Type:} \texttt{bool}}\\
	\multicolumn{2}{|p{15cm}|}{\textbf{Default value:} \texttt{false}}\\
	\multicolumn{2}{|l|}{}\\
	\multicolumn{2}{|p{15cm}|}{\textbf{Description:} \newline Globally enable/disable verbosity of CPU.}\\
	\hline
	\multicolumn{1}{|p{7.5cm}}{\textbf{Name:} \texttt{cpu.verbose-setup}} & \multicolumn{1}{p{7.5cm}|}{\textbf{Type:} \texttt{bool}}\\
	\multicolumn{2}{|p{15cm}|}{\textbf{Default value:} \texttt{false}}\\
	\multicolumn{2}{|l|}{}\\
	\multicolumn{2}{|p{15cm}|}{\textbf{Description:} \newline Enable/disable verbosity of CPU while setup.}\\
	\hline
	\multicolumn{1}{|p{7.5cm}}{\textbf{Name:} \texttt{cpu.verbose-all}} & \multicolumn{1}{p{7.5cm}|}{\textbf{Type:} \texttt{bool}}\\
	\multicolumn{2}{|p{15cm}|}{\textbf{Default value:} \texttt{false}}\\
	\multicolumn{2}{|l|}{}\\
	\multicolumn{2}{|p{15cm}|}{\textbf{Description:} \newline Globally enable/disable verbosity of CPU.}\\
	\hline
	\hline
	\multicolumn{2}{|c|}{\textbf{\large Service GDB Server run-time parameters}}\\
	\hline
	\multicolumn{1}{|p{7.5cm}}{\textbf{Name:} \texttt{gdb-server.tcp-port}} & \multicolumn{1}{p{7.5cm}|}{\textbf{Type:} \texttt{integer}}\\
	\multicolumn{2}{|p{15cm}|}{\textbf{Default value:} \texttt{12345}}\\
	\multicolumn{2}{|l|}{}\\
	\multicolumn{2}{|p{15cm}|}{\textbf{Description:} \newline TCP/IP port to listen waiting for a GDB client connection.}\\
	\hline
	\multicolumn{1}{|p{7.5cm}}{\textbf{Name:} \texttt{gdb-server.architecture-}\newline$\hookrightarrow$\texttt{description-filename}} & \multicolumn{1}{p{7.5cm}|}{\textbf{Type:} \texttt{string}}\\
	\multicolumn{2}{|p{15cm}|}{\textbf{Default value:} \texttt{empty string}}\\
	\multicolumn{2}{|l|}{}\\
	\multicolumn{2}{|p{15cm}|}{\textbf{Description:} \newline Filename of an XML description of the connected processor.}\\
	\hline
	\end{tabular}
\end{center}

\newpage
\begin{center}
	\begin{tabular}{|p{7.5cm}|p{7.5cm}|}
	\hline
	\multicolumn{2}{|c|}{\textbf{\large Inline Debugger run-time parameters}}\\
	\hline
	\multicolumn{1}{|p{7.5cm}}{\textbf{Name:} \texttt{inline-debugger.memory-atom-size}} & \multicolumn{1}{p{7.5cm}|}{\textbf{Type:} \texttt{unsigned integer}}\\
	\multicolumn{2}{|p{15cm}|}{\textbf{Default value:} \texttt{1}}\\
	\multicolumn{2}{|l|}{}\\
	\multicolumn{2}{|p{15cm}|}{\textbf{Description:} \newline Size of the smallest addressable element in memory.}\\
	\hline
	\multicolumn{2}{|c|}{\textbf{\large Kernel Service Logger run-time parameters}}\\
	\hline
	\multicolumn{1}{|p{7.5cm}}{\textbf{Name:} \texttt{kernel\_logger.std\_err}} & \multicolumn{1}{p{7.5cm}|}{\textbf{Type:} \texttt{bool}}\\
	\multicolumn{2}{|p{15cm}|}{\textbf{Default value:} \texttt{false}}\\
	\multicolumn{2}{|l|}{}\\
	\multicolumn{2}{|p{15cm}|}{\textbf{Description:} \newline If true, show logger output through the standard error output.}\\
	\hline
	\multicolumn{1}{|p{7.5cm}}{\textbf{Name:} \texttt{kernel\_logger.std\_out}} & \multicolumn{1}{p{7.5cm}|}{\textbf{Type:} \texttt{bool}}\\
	\multicolumn{2}{|p{15cm}|}{\textbf{Default value:} \texttt{false}}\\
	\multicolumn{2}{|l|}{}\\
	\multicolumn{2}{|p{15cm}|}{\textbf{Description:} \newline If true, show logger output through the standard output.}\\
	\hline
	\multicolumn{1}{|p{7.5cm}}{\textbf{Name:} \texttt{kernel\_logger.std\_err\_color}} & \multicolumn{1}{p{7.5cm}|}{\textbf{Type:} \texttt{bool}}\\
	\multicolumn{2}{|p{15cm}|}{\textbf{Default value:} \texttt{false}}\\
	\multicolumn{2}{|l|}{}\\
	\multicolumn{2}{|p{15cm}|}{\textbf{Description:} \newline If true, colorize logger output through the standard error output (only works if std\_err is active).}\\
	\hline
	\multicolumn{1}{|p{7.5cm}}{\textbf{Name:} \texttt{kernel\_logger.std\_out\_color}} & \multicolumn{1}{p{7.5cm}|}{\textbf{Type:} \texttt{bool}}\\
	\multicolumn{2}{|p{15cm}|}{\textbf{Default value:} \texttt{false}}\\
	\multicolumn{2}{|l|}{}\\
	\multicolumn{2}{|p{15cm}|}{\textbf{Description:} \newline If true, colorize logger output through the standard output (only works if std\_out is active).}\\
	\hline
	\multicolumn{1}{|p{7.5cm}}{\textbf{Name:} \texttt{kernel\_logger.file}} & \multicolumn{1}{p{7.5cm}|}{\textbf{Type:} \texttt{bool}}\\
	\multicolumn{2}{|p{15cm}|}{\textbf{Default value:} \texttt{false}}\\
	\multicolumn{2}{|l|}{}\\
	\multicolumn{2}{|p{15cm}|}{\textbf{Description:} \newline If true, keep logger output in a file.}\\
	\hline
	\multicolumn{1}{|p{7.5cm}}{\textbf{Name:} \texttt{kernel\_logger.filename}} & \multicolumn{1}{p{7.5cm}|}{\textbf{Type:} \texttt{string}}\\
	\multicolumn{2}{|p{15cm}|}{\textbf{Default value:} \texttt{empty string}}\\
	\multicolumn{2}{|l|}{}\\
	\multicolumn{2}{|p{15cm}|}{\textbf{Description:} \newline Filename to keep logger output (the option \texttt{file} must be activated).}\\
	\hline
	\end{tabular}
\end{center}

\newpage
\begin{center}
	\begin{tabular}{|p{7.5cm}|p{7.5cm}|}
	\hline
	\multicolumn{1}{|p{7.5cm}}{\textbf{Name:} \texttt{kernel\_logger.xml\_file}} & \multicolumn{1}{p{7.5cm}|}{\textbf{Type:} \texttt{bool}}\\
	\multicolumn{2}{|p{15cm}|}{\textbf{Default value:} \texttt{false}}\\
	\multicolumn{2}{|l|}{}\\
	\multicolumn{2}{|p{15cm}|}{\textbf{Description:} \newline If true, keep logger output in a file XML formatted.}\\
	\hline
	\multicolumn{1}{|p{7.5cm}}{\textbf{Name:} \texttt{kernel\_logger.xml\_filename}} & \multicolumn{1}{p{7.5cm}|}{\textbf{Type:} \texttt{string}}\\
	\multicolumn{2}{|p{15cm}|}{\textbf{Default value:} \texttt{empty string}}\\
	\multicolumn{2}{|l|}{}\\
	\multicolumn{2}{|p{15cm}|}{\textbf{Description:} \newline Filename to keep logger XML output (the option \texttt{xml\_file} must be activated).}\\
	\hline
	\multicolumn{1}{|p{7.5cm}}{\textbf{Name:} \texttt{kernel\_logger.xml\_file\_gzipped}} & \multicolumn{1}{p{7.5cm}|}{\textbf{Type:} \texttt{bool}}\\
	\multicolumn{2}{|p{15cm}|}{\textbf{Default value:} \texttt{false}}\\
	\multicolumn{2}{|l|}{}\\
	\multicolumn{2}{|p{15cm}|}{\textbf{Description:} \newline If the \texttt{xml\_file} option is active, the output file will be compressed (a .gz extension will be automatically added to the \texttt{xml\_filename} option.}\\
	\hline
	\multicolumn{2}{|c|}{\textbf{\large Service COFF Loader run-time parameters}}\\
	\hline
	\multicolumn{1}{|p{7.5cm}}{\textbf{Name:} \texttt{loader.filename}} & \multicolumn{1}{p{7.5cm}|}{\textbf{Type:} \texttt{string}}\\
	\multicolumn{2}{|p{15cm}|}{\textbf{Default value:} \texttt{empty string}}\\
	\multicolumn{2}{|l|}{}\\
	\multicolumn{2}{|p{15cm}|}{\textbf{Description:} \newline The COFF file name to load into memory.}\\
	\hline
	\multicolumn{1}{|p{7.5cm}}{\textbf{Name:} \texttt{loader.dump-headers}} & \multicolumn{1}{p{7.5cm}|}{\textbf{Type:} \texttt{bool}}\\
	\multicolumn{2}{|p{15cm}|}{\textbf{Default value:} \texttt{empty string}}\\
	\multicolumn{2}{|l|}{}\\
	\multicolumn{2}{|p{15cm}|}{\textbf{Description:} \newline If true this parameter makes the COFF loader print the file headers on the screen (file header, section headers, symbol table \ldots) while loading the program.}\\
	\hline
	\multicolumn{2}{|c|}{\textbf{\large Memory run-time parameters}}\\
	\hline
	\multicolumn{1}{|p{7.5cm}}{\textbf{Name:} \texttt{memory.org}} & \multicolumn{1}{p{7.5cm}|}{\textbf{Type:} \texttt{64-bit unsigned integer}}\\
	\multicolumn{2}{|p{15cm}|}{\textbf{Default value:} \texttt{0}}\\
	\multicolumn{2}{|l|}{}\\
	\multicolumn{2}{|p{15cm}|}{\textbf{Description:} \newline Starting address of memory (typically 0).}\\
	\hline
	\multicolumn{1}{|p{7.5cm}}{\textbf{Name:} \texttt{memory.bytesize}} & \multicolumn{1}{p{7.5cm}|}{\textbf{Type:} \texttt{64-bit unsigned integer}}\\
	\multicolumn{2}{|p{15cm}|}{\textbf{Default value:} \texttt{0}}\\
	\multicolumn{2}{|l|}{}\\
	\multicolumn{2}{|p{15cm}|}{\textbf{Description:} \newline Size in bytes (8-bit) of memory.}\\
	\hline
	\end{tabular}
\end{center}

\newpage
\begin{center}
	\begin{tabular}{|p{7.5cm}|p{7.5cm}|}
	\hline
	\multicolumn{2}{|c|}{\textbf{\large Service TI C I/O run-time parameters}}\\
	\hline
	\multicolumn{1}{|p{7.5cm}}{\textbf{Name:} \texttt{ti\_c\_io.enable}} & \multicolumn{1}{p{7.5cm}|}{\textbf{Type:} \texttt{bool}}\\
	\multicolumn{2}{|p{15cm}|}{\textbf{Default value:} \texttt{false}}\\
	\multicolumn{2}{|l|}{}\\
	\multicolumn{2}{|p{15cm}|}{\textbf{Description:} \newline Enable/Disable TI C I/O support.}\\
	\hline
	\multicolumn{1}{|p{7.5cm}}{\textbf{Name:} \texttt{ti-c-io.warning-as-error}} & \multicolumn{1}{p{7.5cm}|}{\textbf{Type:} \texttt{bool}}\\
	\multicolumn{2}{|p{15cm}|}{\textbf{Default value:} \texttt{false}}\\
	\multicolumn{2}{|l|}{}\\
	\multicolumn{2}{|p{15cm}|}{\textbf{Description:} \newline Whether Warnings are considered as error or not.}\\
	\hline
	\multicolumn{1}{|p{7.5cm}}{\textbf{Name:} \texttt{ti-c-io.pc-register-name}} & \multicolumn{1}{p{7.5cm}|}{\textbf{Type:} \texttt{string}}\\
	\multicolumn{2}{|p{15cm}|}{\textbf{Default value:} \texttt{"PC"}}\\
	\multicolumn{2}{|l|}{}\\
	\multicolumn{2}{|p{15cm}|}{\textbf{Description:} \newline Name of the CPU program counter register.}\\
	\hline
	\multicolumn{1}{|p{7.5cm}}{\textbf{Name:} \texttt{ti-c-io.c-io-buffer-symbol-name}} & \multicolumn{1}{p{7.5cm}|}{\textbf{Type:} \texttt{string}}\\
	\multicolumn{2}{|p{15cm}|}{\textbf{Default value:} \texttt{"\_\_CIOBUF\_"}}\\
	\multicolumn{2}{|l|}{}\\
	\multicolumn{2}{|p{15cm}|}{\textbf{Description:} \newline C I/O buffer symbol name.}\\
	\hline
	\multicolumn{1}{|p{7.5cm}}{\textbf{Name:} \texttt{ti-c-io.c-io-breakpoint-}\newline$\hookrightarrow$\texttt{symbol-name}} & \multicolumn{1}{p{7.5cm}|}{\textbf{Type:} \texttt{string}}\\
	\multicolumn{2}{|p{15cm}|}{\textbf{Default value:} \texttt{"C\$\$IO\$\$"}}\\
	\multicolumn{2}{|l|}{}\\
	\multicolumn{2}{|p{15cm}|}{\textbf{Description:} \newline C I/O breakpoint symbol name.}\\
	\hline
	\multicolumn{1}{|p{7.5cm}}{\textbf{Name:} \texttt{ti-c-io.c-exit-breakpoint-}\newline$\hookrightarrow$\texttt{symbol-name}} & \multicolumn{1}{p{7.5cm}|}{\textbf{Type:} \texttt{string}}\\
	\multicolumn{2}{|p{15cm}|}{\textbf{Default value:} \texttt{"C\$\$EXIT"}}\\
	\multicolumn{2}{|l|}{}\\
	\multicolumn{2}{|p{15cm}|}{\textbf{Description:} \newline C EXIT breakpoint symbol name.}\\
	\hline
	\multicolumn{1}{|p{7.5cm}}{\textbf{Name:} \texttt{ti-c-io.verbose-all}} & \multicolumn{1}{p{7.5cm}|}{\textbf{Type:} \texttt{bool}}\\
	\multicolumn{2}{|p{15cm}|}{\textbf{Default value:} \texttt{false}}\\
	\multicolumn{2}{|l|}{}\\
	\multicolumn{2}{|p{15cm}|}{\textbf{Description:} \newline Globally enable/disable verbosity of TI C I/O service.}\\
	\hline
	\multicolumn{1}{|p{7.5cm}}{\textbf{Name:} \texttt{ti-c-io.verbose-io}} & \multicolumn{1}{p{7.5cm}|}{\textbf{Type:} \texttt{bool}}\\
	\multicolumn{2}{|p{15cm}|}{\textbf{Default value:} \texttt{false}}\\
	\multicolumn{2}{|l|}{}\\
	\multicolumn{2}{|p{15cm}|}{\textbf{Description:} \newline Enable/disable verbosity of TI C I/O service while I/Os.}\\
	\hline
	\multicolumn{1}{|p{7.5cm}}{\textbf{Name:} \texttt{ti-c-io.verbose-setup}} & \multicolumn{1}{p{7.5cm}|}{\textbf{Type:} \texttt{bool}}\\
	\multicolumn{2}{|p{15cm}|}{\textbf{Default value:} \texttt{false}}\\
	\multicolumn{2}{|l|}{}\\
	\multicolumn{2}{|p{15cm}|}{\textbf{Description:} \newline Enable/disable verbosity of TI C I/O service while setup.}\\
	\hline
	\end{tabular}
\end{center}

\subsection{Status of implementation}

Current implementation of the simulator completely decodes the TMS320C3X instruction set.
All registers are present but no on-chip devices are implemented.
The simulator is supposed to have a complete support for:
\begin{itemize}
\item integer instructions (2-ops, 3-ops, parallel ops, load/store)
\item control instructions (branches, delayed branches, RPTS, RPTB)
\item power instructions
\item instructions \texttt{FLOAT}, \texttt{PUSHF}, and \texttt{POPF}
\item interrupt handling
\end{itemize}

The simulator does not implement or partially implement:
\begin{itemize}
\item floating point instructions
\item interlocked instructions
\end{itemize}

The indirect addressing with circular modify and the indirect addressing with bit-reversed modify modes are fully implemented but untested.
The current status of the simulator should allow to run any integer benchmark.

\section{Developer guide}

This part is not yet available.

\subsection{Overview}

\subsection{TMS320C3X instruction set simulator}

\subsection{COFF loader service}

\subsection{TI C I/O service}

\subsection{Instruction level Debugging services}

\subsubsection{Inline debugger}

\subsubsection{GDB server}

\section{Validation guide}

\subsection{Integer benchmarks}

\subsubsection{Fibonacci}

This benchmarks recursively (and quite inefficiently) computes the Fibonacci numbers:

\indent $F_1=1$
\newline
\indent $F_2=1$
\newline
\indent $F_n = F_{n - 2} + F_{n - 1}$ where $n > 2$
\newline

This benchmark requires TI C I/O service enabled to run in the TMS320C3X simulator.
A precompiled binary (\texttt{fibo.bin}) is provided together with a GNU Make compatible \texttt{Makefile}.
A simulation configuration (\texttt{sim\_config.xml}) for this simulator is also provided, so that simulator can run the benchmark:

\begin{verbatim}
$ tms320c3x -c sim_config.xml
\end{verbatim}

\noindent The expected ouput on the screen of the benchmarks is:

\begin{verbatim}
Fibo(1)=1 (0x1)
Fibo(2)=1 (0x1)
Fibo(3)=2 (0x2)
Fibo(4)=3 (0x3)
Fibo(5)=5 (0x5)
Fibo(6)=8 (0x8)
Fibo(7)=13 (0xd)
Fibo(8)=21 (0x15)
Fibo(9)=34 (0x22)
Fibo(10)=55 (0x37)
Fibo(11)=89 (0x59)
Fibo(12)=144 (0x90)
Fibo(13)=233 (0xe9)
Fibo(14)=377 (0x179)
Fibo(15)=610 (0x262)
Fibo(16)=987 (0x3db)
Fibo(17)=1597 (0x63d)
Fibo(18)=2584 (0xa18)
Fibo(19)=4181 (0x1055)
Fibo(20)=6765 (0x1a6d)
Fibo(21)=10946 (0x2ac2)
Fibo(22)=17711 (0x452f)
Fibo(23)=28657 (0x6ff1)
Fibo(24)=46368 (0xb520)
Fibo(25)=75025 (0x12511)
Fibo(26)=121393 (0x1da31)
Fibo(27)=196418 (0x2ff42)
Fibo(28)=317811 (0x4d973)
Fibo(29)=514229 (0x7d8b5)
Fibo(30)=832040 (0xcb228)
Fibo(31)=1346269 (0x148add)
Fibo(32)=2178309 (0x213d05)
Fibo(33)=3524578 (0x35c7e2)
Fibo(34)=5702887 (0x5704e7)
\end{verbatim}

\subsubsection{Quick sort}

This benchmark sorts 65536 integer numbers using the quick sort recursive algorithm.
The input data set is in file \texttt{random.txt} that contains random generated integer numbers.
The output data set after the benchmark run is in file \texttt{sort.sim.txt}.

This benchmark requires TI C I/O service enabled to run in the TMS320C3X simulator.
A precompiled binary (\texttt{quicksort.bin}) is provided together with a GNU Make compatible \texttt{Makefile}.
A simulation configuration (\texttt{sim\_config.xml}) for this simulator is also provided, so that simulator can run the benchmark:

\begin{verbatim}
 $ tms320c3x -c sim_config.xml
\end{verbatim}

\noindent The expected ouput data set is in file \texttt{sort.linux.txt}.

\subsubsection{CRC32 (check sum)}

This benchmark is based on CRC32 benchmark from MiBench Version 1.0 (\url{http://www.eecs.umich.edu/mibench}).
It performs a 32-bit Cyclic Redundancy Check (CRC) on a file. CRC checks are often used to detect errors in data transmission.
The benchmark reads file \texttt{large.pcm} and print the check sum on the screen

This benchmark requires TI C I/O service enabled to run in the TMS320C3X simulator.
A precompiled binary is provided together with a GNU Make compatible Makefile.
A simulation configuration (\texttt{sim\_config.xml}) for this simulator is also provided, so that simulator can run the benchmark:

\begin{verbatim}
$ tms320c3x -c sim_config.xml
\end{verbatim}

\noindent The expected ouput on the screen of the benchmarks is in file ref.txt:

\begin{verbatim}
32 BIT ANSI X3.66 CRC checksum:
Opening input file "large.pcm"
Total number of bytes read: 26611200
CRC32: dc8b7c2a
\end{verbatim}

\subsubsection{Rijndael (encryption/decryption)}

This benchmark is based on Rijndael benchmark from MiBench Version 1.0 (\url{http://www.eecs.umich.edu/mibench}).
Rijndael was selected as the National Institute of Standards and Technologies Advanced Encryption Standard (AES).
It is a block cipher with the option of 128-, 192-, and 256-bit keys and blocks.

In this benchmark, encryption is followed by decryption so that input data set and output data set should be identical.
The benchmark uses this hexadecimal encryption key:
\begin{verbatim}
1234567890abcdeffedcba09876543211234567890abcdeffedcba0987654321
\end{verbatim}
The benchmark reads file \texttt{input\_small.asc}, and encrypt it into file \texttt{output\_small.enc}.
It decrypts \texttt{output\_small.enc} into file \texttt{dec\_output\_small.enc}.

This benchmark requires TI C I/O service enabled to run in the TMS320C3X simulator.
A precompiled binary (\texttt{rijndael.bin}) is provided together with a GNU Make compatible Makefile.
A simulation configuration (\texttt{sim\_config.xml}) for this simulator is also provided, so that simulator can run the benchmark:

\begin{verbatim}
$ tms320c3x -c sim_config.xml
\end{verbatim}

It is expected that files \texttt{input\_small.asc} and \texttt{dec\_output\_small.enc} be identical after the benchmark run.

\subsubsection{Sha (encryption/decryption)}

This benchmark is based on SHA benchmark from MiBench Version 1.0 (\url{http://www.eecs.umich.edu/mibench}).
SHA is the secure hash algorithm that produces a 160-bit message digest for a given input. It is often
used in the secure exchange of cryptographic keys and for generating digital signatures. It is also used in the
well-known MD4 and MD5 hashing functions.

The benchmark has its input data set embedded its C source code.

A precompiled binary (\texttt{sha.bin}) is provided together with a GNU Make compatible Makefile.
A simulation configuration (\texttt{sim\_config.xml}) for this simulator is also provided, so that simulator can run the benchmark:

\begin{verbatim}
$ tms320c3x -c sim_config.xml
\end{verbatim}

\subsubsection{ADPCM (sound encoding/decoding)}

This benchmark is based on ADPCM benchmark from MiBench Version 1.0 (\url{http://www.eecs.umich.edu/mibench}).
It performs ADPCM encoding/decoding. Adaptive Differential Pulse Code Modulation (ADPCM) is a variation
of the well-known standard Pulse Code Modulation (PCM). A common implementation takes 16-bit linear
PCM samples and converts them to 4-bit samples, yielding a compression rate of 4:1. The input data are
small and large speech samples.
The ADPCM coder benchmark reads file \texttt{large.pcm} and writes the compressed data in file \texttt{output\_large.sim.adpcm}.
The ADPCM decoder benchmark reads file \texttt{large.adpcm} and writes the uncompressed data in file \texttt{output\_large.sim.pcm}.

This benchmark requires TI C I/O service enabled to run in the TMS320C3X simulator.
Precompiled binaries (\texttt{coder.bin} and \texttt{decoder.bin}) are provided together with a GNU Make compatible Makefile.
Simulation configurations (\texttt{coder\_sim\_config.xml} and \texttt{decoder\_sim\_config.xml})) for this simulator are also provided, so that simulator can run the benchmarks:

\begin{verbatim}
$ tms320c3x -c coder_sim_config.xml
$ tms320c3x -c decoder_sim_config.xml
\end{verbatim}

The expected ouput data set of the ADPCM coder benchmark is in file \texttt{output\_large.linux.adpcm}.
The expected ouput data set of the ADPCM decoder benchmark is in file \texttt{output\_large.linux.pcm}.

\subsubsection{DCT/Quantization (image processing)}

This benchmark is based on XVID video codec (\url{http://www.xvid.org}).
The benchmarks has the following steps that are the base of the JPEG lossy image compression standard:
\begin{enumerate}
\item Load a Windows 24-bit RGB Bitmap from a .bmp file;
\item Convert from RGB to YUV 4:4:4 for each 8x8 pixel blocks;
\item Compute a DCT on each 8x8 pixel blocks;
\item Quantize each 8x8 pixels blocks;
\item Dequantize each 8x8 pixels blocks;
\item Compute an iDCT on each 8x8 pixel blocks;
\item Convert from YUV 4:4:4 to RGB each 8x8 pixel blocks
\item Save the resulting Windows 24-bit RGB bitmap into a .bmp file
\end{enumerate}

The benchmark reads the input image from file \texttt{image.bmp} and the quantization matrix from file \texttt{quant\_mat.txt}.
It save the resulting image in file \texttt{output\_image.sim.bmp}

This benchmark requires TI C I/O service enabled to run in the TMS320C3X simulator.
A precompiled binary (\texttt{dct\_quant.bin}) is provided together with a GNU Make compatible Makefile.
A simulation configuration (\texttt{sim\_config.xml}) for this simulator is also provided, so that simulator can run the benchmark:

\begin{verbatim}
$ tms320c3x -c sim_config.xml
\end{verbatim}

The expected ouput image is in file \texttt{output\_image.linux.bmp}.

\subsection{Floating point benchmarks}

This part is not yet available.

\subsection{Instruction level unit test}

This part is not yet available.
