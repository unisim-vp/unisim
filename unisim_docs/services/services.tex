Designing a new emulator, and particularly for a research purpose, means implementing an instruction set emulator but also involves several software components not directly related to pure instruction set execution.
The most obvious needed software components are memories, debuggers, loaders, but components such as chipsets and peripherals are still mandatory to enable running real unmodified applications.
Making all these components running together requires programming interfaces as much standard as possible.

Usually the programmer faces to the problems of sharing source codes among several emulators, reusing existing source codes, and building a fully functional emulator from all these heterogeneous pieces of source codes.
Most of the time, the software components are strongly dependent for each other: components are statically linked together through explicit function calls and adhoc interfaces.
Replacing these adhoc interfaces with C++ pure interfaces (C++ classes with only unimplemented virtual methods, see your C++ manual for more details) and linking the components through pointers is a step toward avoiding such strong dependencies between the components. But still finding a standard manner to initialize those pointers is necessary. This can be done either by directly writting in those pointers or calling special functions to do the job.

Another problem with heterogeneous software components is the manner to instantiate and parameterize those components in a standard way, so that it is easier for the component's user to use a new component.
Usually, parameterizing a component means passing arguments to an initialization function or a class constructor. It implies that the programmers agree on using only one of these two solutions or both.
Still the programmers must know the setup order of these components: it is an error prone process because determining a correct order from the components documentation will likely fail the first times.

% problems:\\
% - C++ class code sharing/reuse/composition\\
% - direct call problem\\
% - parameterization problem\\
% - setup problem: dependencies, which setup order ?\\
% \\

In this section, we propose a standard way to share, reuse, link, parameterize and setup these software components.
C++ object oriented programming and pure C++ interfaces enable sharing and reuse.
Some special pointers (classes \texttt{ServiceImport} and \texttt{ServiceExport}) linking the software components (classes \texttt{Service} and \texttt{Client}) together with some base software component classes have been introduced, thus enabling easier component composition and connection.
The parameterization have been standardized (class \texttt{Parameter}) and the framework (class \texttt{ServiceManager}) uses the call graph to provide the user with an automatic setup order.

Section~\ref{services_library} documents the available services in the library. Section~\ref{building_a_service_graph} presents how to use services, link clients and services, and setup each components. Section~\ref{designing_clients_and_services} not only presents how to design a service, either a totally new service, or an assembly of existing services, but also how to design a client invoking a service.

% solution:\\
% sharing/reuse -> classes, abstract interfaces\\
% composition -> service, client, import/export, call/use graph (connections)\\
% parameterization -> generic parameters\\
% setup problem -> call/use graph enables automatic setup order\\

\section{Service Interfaces}
\label{service_interfaces}

\section{Services library}
\label{services_library}

\subsection{Loaders}

\newpage

\begin{center}
	\begin{tabular}{|p{7.5cm}|p{7.5cm}|}
		\hline
		\multicolumn{2}{|l|}{\textbf{\Large Service ELF Loader}}\\
		\hline
		\multicolumn{1}{|p{7.5cm}}{\textbf{Class Name:} \newline \texttt{unisim::service::loader::elf\_loader}\newline$\hookrightarrow$\texttt{::ElfLoaderImpl}} & \multicolumn{1}{p{7.5cm}|}{\textbf{Header:} \newline \texttt{unisim/service/loader/elf\_loader}\newline$\hookrightarrow$\texttt{/elf\_loader.hh}}\\
		\multicolumn{2}{|l|}{}\\
		\multicolumn{2}{|p{15cm}|}{\textbf{Description:} \newline The ELF loader service allows to load a binary program into a memory and fill a symbol table.
		The loader also provides information about the loaded file such as the code and data locations (base address and size). The ELF loader loads the program during setup. \texttt{Elf32Loader} and \texttt{Elf64Loader} are two specialized versions of class \texttt{ElfLoaderImpl<>} which the user should use whenever possible.}\\
		\hline
		\hline
		\multicolumn{2}{|l|}{\textbf{Template Parameters}}\\
		\hline
		\multicolumn{1}{|p{7.5cm}}{\textbf{Name:} \texttt{MEMORY\_ADDR}} & \multicolumn{1}{p{7.5cm}|}{\textbf{Type:} \texttt{class}}\\
		\multicolumn{2}{|p{15cm}|}{\textbf{Default value:} none}\\
		\multicolumn{2}{|l|}{}\\
		\multicolumn{2}{|p{15cm}|}{\textbf{Description:} \newline This is the C++ type of a memory address (e.g. uint32\_t or uint64\_t).}\\
		\hline
		\multicolumn{1}{|p{7.5cm}}{\textbf{Name:} \texttt{T}} & \multicolumn{1}{p{7.5cm}|}{\textbf{Type:} \texttt{class}}\\
		\multicolumn{2}{|p{15cm}|}{\textbf{Default value:} none}\\
		\multicolumn{2}{|l|}{}\\
		\multicolumn{2}{|p{15cm}|}{\textbf{Description:} \newline We should consider removing this parameter or have a reason to have it.}\\
		\hline
		\multicolumn{1}{|p{7.5cm}}{\textbf{Name:} \texttt{ElfClass}} & \multicolumn{1}{p{7.5cm}|}{\textbf{Type:} \texttt{unsigned int}}\\
		\multicolumn{2}{|p{15cm}|}{\textbf{Default value:} none}\\
		\multicolumn{2}{|l|}{}\\
		\multicolumn{2}{|p{15cm}|}{\textbf{Description:} \newline class of machine (either \texttt{ELFCLASS32} or \texttt{ELFCLASS64}).}\\
		\hline
		\multicolumn{1}{|p{7.5cm}}{\textbf{Name:} \texttt{Elf\_Ehdr}} & \multicolumn{1}{p{7.5cm}|}{\textbf{Type:} \texttt{class}}\\
		\multicolumn{2}{|p{15cm}|}{\textbf{Default value:} none}\\
		\multicolumn{2}{|l|}{}\\
		\multicolumn{2}{|p{15cm}|}{\textbf{Description:} \newline ELF Header structure (either Elf32\_Ehdr or Elf64\_Ehdr).}\\
		\hline
		\multicolumn{1}{|p{7.5cm}}{\textbf{Name:} \texttt{Elf\_Phdr}} & \multicolumn{1}{p{7.5cm}|}{\textbf{Type:} \texttt{class}}\\
		\multicolumn{2}{|p{15cm}|}{\textbf{Default value:} none}\\
		\multicolumn{2}{|l|}{}\\
		\multicolumn{2}{|p{15cm}|}{\textbf{Description:} \newline ELF Program Header structure (either Elf32\_Phdr or Elf64\_Phdr).}\\
		\hline
		\multicolumn{1}{|p{7.5cm}}{\textbf{Name:} \texttt{Elf\_Shdr}} & \multicolumn{1}{p{7.5cm}|}{\textbf{Type:} \texttt{class}}\\
		\multicolumn{2}{|p{15cm}|}{\textbf{Default value:} none}\\
		\multicolumn{2}{|l|}{}\\
		\multicolumn{2}{|p{15cm}|}{\textbf{Description:} \newline ELF Section Header structure (either \texttt{Elf32\_Shdr} or \texttt{Elf64\_Shdr}).}\\
		\hline
		\multicolumn{1}{|p{7.5cm}}{\textbf{Name:} \texttt{Elf\_Sym}} & \multicolumn{1}{p{7.5cm}|}{\textbf{Type:} \texttt{class}}\\
		\multicolumn{2}{|p{15cm}|}{\textbf{Default value:} none}\\
		\multicolumn{2}{|l|}{}\\
		\multicolumn{2}{|p{15cm}|}{\textbf{Description:} \newline ELF Symbol structure (either \texttt{Elf32\_Sym} or \texttt{Elf64\_Sym}).}\\
		\hline
	\end{tabular}
\end{center}

\begin{center}
	\begin{tabular}{|p{7.5cm}|p{7.5cm}|}
		\hline
		\multicolumn{2}{|c|}{\textbf{\large Run-Time Parameters}}\\
		\hline
		\multicolumn{1}{|p{7.5cm}}{\textbf{Name:} \texttt{filename}} & \multicolumn{1}{p{7.5cm}|}{\textbf{Type:} \texttt{string}}\\
		\multicolumn{2}{|p{15cm}|}{\textbf{Default value:} empty string}\\
		\multicolumn{2}{|l|}{}\\
		\multicolumn{2}{|p{15cm}|}{\textbf{Description:} \newline The ELF file name to load into the connected memory.}\\
		\hline
		\multicolumn{1}{|p{7.5cm}}{\textbf{Name:} \texttt{base-addr}} & \multicolumn{1}{p{7.5cm}|}{\textbf{Type:} \texttt{MEMORY\_ADDR}}\\
		\multicolumn{2}{|p{15cm}|}{\textbf{Default value:} \texttt{0}}\\
		\multicolumn{2}{|l|}{}\\
		\multicolumn{2}{|p{15cm}|}{\textbf{Description:} \newline If this parameter is non-zero, force the ELF Loader to load the unique program segment at this base address.}\\
		\hline
		\multicolumn{1}{|p{7.5cm}}{\textbf{Name:} \texttt{force-use-virtual-address}} & \multicolumn{1}{p{7.5cm}|}{\textbf{Type:} \texttt{bool}}\\
		\multicolumn{2}{|p{15cm}|}{\textbf{Default value:} \texttt{false}}\\
		\multicolumn{2}{|l|}{}\\
		\multicolumn{2}{|p{15cm}|}{\textbf{Description:} \newline If true, this parameter forces the ELF loader to use the segment virtual address instead of the segment physical address when loading the segments into memory.}\\
		\hline
		\hline
		\multicolumn{2}{|c|}{\textbf{\large Service Exports}}\\
		\hline
		\multicolumn{1}{|p{7.5cm}}{\textbf{Name:} \texttt{logger\_export}} & \multicolumn{1}{p{7.5cm}|}{\textbf{Interface:} \newline \texttt{unisim::service::interfaces::} \newline$\hookrightarrow$\texttt{Loader<MEMORY\_ADDR>}}\\
		\multicolumn{2}{|l|}{}\\
		\multicolumn{2}{|p{15cm}|}{\textbf{Description:} \newline The ELF loader provides information about the code and data location through this export.}\\
		\hline
		\hline
		\multicolumn{2}{|c|}{\textbf{\large Service Imports}}\\
		\hline
		\multicolumn{1}{|p{7.5cm}}{\textbf{Name:} \texttt{memory\_import}} & \multicolumn{1}{p{7.5cm}|}{\textbf{Interface:} \newline \texttt{unisim::service::interfaces::} \newline$\hookrightarrow$\texttt{Memory<MEMORY\_ADDR>}}\\
		\multicolumn{2}{|p{15cm}|}{\textbf{Mandatory connected:} no}\\
		\multicolumn{2}{|l|}{}\\
		\multicolumn{2}{|p{15cm}|}{\textbf{Description:} \newline The ELF loader accesses to the memory through this import.}\\
		\hline
		\multicolumn{1}{|p{7.5cm}}{\textbf{Name:} \texttt{symbol\_table\_build\_import}} & \multicolumn{1}{p{7.5cm}|}{\textbf{Interface:} \newline \texttt{unisim::service::interfaces::} \newline$\hookrightarrow$\texttt{SymbolTableBuild<MEMORY\_ADDR>}}\\
		\multicolumn{2}{|p{15cm}|}{\textbf{Mandatory connected:} no}\\
		\multicolumn{2}{|l|}{}\\
		\multicolumn{2}{|p{15cm}|}{\textbf{Description:} \newline The ELF loader fills the symbol table using this import.}\\
		\hline
	\end{tabular}
\end{center}

% 		\multicolumn{16}{c}{}\\
% 		15  &  14  &  13  &  12  &  11  &  10  &  ~9  & ~8  &  ~7  &  ~6  &  ~5  &  ~4  &  ~3  &  ~2  &  ~1  &  ~0 \\
% 		\hline
% 		\multicolumn{1}{|c|}{1}  & \multicolumn{1}{|c|}{0} & \multicolumn{1}{|c|}{1} & \multicolumn{1}{|c|}{0} & \multicolumn{4}{|c|}{rs} &\multicolumn{8}{|c|}{offset}\\
% 		\hline
% 		\multicolumn{16}{c}{}\\
% 		\multicolumn{16}{l}{if (rs) \textless \textgreater 0 then}\\
% 		\multicolumn{1}{l}{} &\multicolumn{15}{l}{NIA $\leftarrow$ CIA + offset}\\
% 		\multicolumn{16}{l}{else}\\
% 		\multicolumn{1}{l}{} & \multicolumn{15}{l}{NIA $\leftarrow$ CIA + 1}\\

% \subsubsection{PowerMac BootX Loader}
% 
% \subsubsection{PowerMAC Linux Kernel Loader}
% 
% \subsubsection{S19 Loader}

\subsection{Operating System Application Binary Interface (OS ABI)}

\subsubsection{Linux OS system call translator}

\subsection{Debug}

\subsubsection{Inline Debugger \& GDB Server}

The library provides two debuggers using the same interfaces, and naming convention for the imports and exports: an inline debugger (text base) and a GDB server communicating through TCP/IP with the standard GDB client supporting the target machine.

%	ServiceImport<SymbolTableLookupInterface<ADDRESS> > symbol_table;

\noindent Exports:
\begin{itemize}
\item member:~\texttt{exp}\\
service interface:~{\scriptsize ~\texttt{full\_system::plugins::debug::InstructionLevelDebuggerInterface<ADDRESS>}}\\
client interface:~{\scriptsize \texttt{full\_system::plugins::debug::CPUInstructionLevelDebugInterface<ADDRESS>}}\\
header:~{\scriptsize \texttt{plugins/debug/instruction\_level\_debug\_interface.hh}}\\
description: the debugger provides debug facilities to the CPU client through this export.
\end{itemize}


\subsubsection{Symbol Table}

\subsection{Power Estimators}

\subsubsection{Cache/SRAM Power Estimator}

\subsection{Simple DirectMedia Layer (SDL)}

\subsection{Tees}

\subsection{Time}

\section{Building a service graph}
\label{building_a_service_graph}

Using services implies building a service graph.
For instance, consider that the client is a loader, and the service is a memory.
The programmer creates objects \texttt{loader} and \texttt{memory}, see Figure~\ref{fig:instanciation}.

\begin{figure}[h]
  \begin{center}
    \input{services/instanciation}
    \caption{\label{fig:instanciation} Client/Service instanciation.}
  \end{center}
\end{figure}

Object \texttt{loader} is a client because it needs a service (reading/writing in memory) from object \texttt{memory} to load the program.
\texttt{loader} has a member \texttt{import} named \texttt{memory\_import} whereas \texttt{memory} object has a member \texttt{export} named \texttt{memory\_export}.
The programmer connects the loader to the memory using \texttt{loader.memory\_import} and \texttt{memory.memory\_export}, see Figure~\ref{fig:connection}.

\begin{figure}[h]
  \begin{center}
    \input{services/connection}
    \caption{\label{fig:connection} Import/Export connection.}
  \end{center}
\end{figure}

Once \hfill the \hfill programmer \hfill has \hfill created \hfill a \hfill service \hfill graph, \hfill he \hfill must \hfill perform \hfill a \hfill call \hfill to \hfill \texttt{ServiceManager::Setup()}.
\texttt{ServiceManager::Setup()} returns \texttt{true} if setup of each service and client in the graph has been successful, other it returns \texttt{false}.

\section{Designing clients and services}
\label{designing_clients_and_services}

\subsection{Design a service by composing existing services}

\subsection{Designing a service}

A service is a C++ object inheriting from template class \texttt{Service<SERVICE\_INTERFACE>} \ding{202}, see Figure~\ref{fig:simple_service}.
SERVICE\_INTERFACE is a C++ abstract class defining the virtual methods implemented by the service.
To export its interface, a service must have a member of type \texttt{ServiceExport<SERVICE\_INTERFACE>} \ding{203}.
For normalization purposes, the service constructor should take only two parameters \ding{204}: the service name and the pointer to the parent (a container service).
The pointer to the parent is \texttt{null} if the service is a top level service (no parent).
The base \texttt{Object} constructor \ding{205} and the base \texttt{Service} constructor \ding{206} must be called with the name and the pointer to the parent.
\texttt{ServiceExport} member constructor must be called with the export name and a pointer to the owner, i.e. the service itself \ding{207}.

\begin{figure}[h]
  \begin{center}
    \input{services/simple_service}
    \caption{\label{fig:simple_service} Simple service.}
  \end{center}
\end{figure}


\subsection{Designing a client}

\subsection{Adding parameterization to a service}

\subsection{Handling service setup}


% \begin{figure}[h]
%   \begin{center}
%     \input{tlm/interface1}
%     \caption{\label{fig:interface1} TLM interface (simplified).}
%   \end{center}
% \end{figure}

% \subsection{Drawbacks \& Programming facilities: Pointer and event\_list}
% \label{drawbacks_and_programming_facilities}

% \begin{figure}[p]
%   \begin{center}
%     \input{tlm/ttlm_router_adv2}
%     \caption{\label{fig:ttlm_router_adv2} Example of TTLM router
%       module with requests and responses contention.}
%   \end{center}
% \end{figure}
