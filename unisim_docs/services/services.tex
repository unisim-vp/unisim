Designing a new emulator, and particularly for a research purpose, means implementing an instruction set emulator but also involves several software components not directly related to pure instruction set execution.
The most obvious needed software components are memories, debuggers, loaders, but components such as chipsets and peripherals are still mandatory to enable running real unmodified applications.
Abstracting the underlying host hardware is also something useful to emulators.
Making all these components running together requires programming interfaces as much standard as possible.

Usually the programmer faces to the problems of sharing source codes among several emulators, reusing existing source codes, and building a fully functional emulator from all these heterogeneous pieces of source codes.
Most of the time, the software components are strongly dependent for each other: components are statically linked together through explicit function calls and adhoc interfaces.
Replacing these adhoc interfaces with C++ pure interfaces (C++ classes with only unimplemented virtual methods, see your C++ manual for more details) and linking the components through pointers is a step toward avoiding such strong dependencies between the components. But still finding a standard manner to initialize those pointers is necessary. This can be done either by directly writting in those pointers or calling special functions to do the job.

Another problem with heterogeneous software components is the manner to instantiate and parameterize those components in a standard way, so that it is easier for the component's user to use a new component.
Usually, parameterizing a component means passing arguments to an initialization function or a class constructor. It implies that the programmers agree on using only one of these two solutions or both.
Still the programmers must know the setup order of these components: it is an error prone process because determining a correct order from the components documentation will likely fail the first times.

% problems:\\
% - C++ class code sharing/reuse/composition\\
% - direct call problem\\
% - parameterization problem\\
% - setup problem: dependencies, which setup order ?\\
% \\

In this section, we propose a standard way to share, reuse, link, parameterize and setup these software components.
C++ object oriented programming and pure C++ interfaces enable sharing and reuse.
Some special pointers (classes \texttt{ServiceImport} and \texttt{ServiceExport}) linking the software components (classes \texttt{Service} and \texttt{Client}) together with some base software component classes have been introduced, thus enabling easier component composition and connection.
The parameterization have been standardized (class \texttt{Parameter}) and the framework (class \texttt{ServiceManager}) uses the call graph to provide the user with an automatic setup order.

Section~\ref{services_library} documents the available services in the library. Section~\ref{building_a_service_graph} presents how to use services, link clients and services, and setup each components. Section~\ref{designing_clients_and_services} not only presents how to design a service, either a totally new service, or an assembly of existing services, but also how to design a client invoking a service.

% solution:\\
% sharing/reuse -> classes, abstract interfaces\\
% composition -> service, client, import/export, call/use graph (connections)\\
% parameterization -> generic parameters\\
% setup problem -> call/use graph enables automatic setup order\\

\section{Service Interfaces}
\label{service_interfaces}

This section will be completed with a comprehensive documentation of the service interfaces.

\section{Services library}
\label{services_library}

The service library currently contains the following services:
\begin{itemize}
	\item Loaders: ELF loaders, PowerMac Linux Kernel Loader, PowerMac BootX emulator, S19 Loader
	\item Debug Stuff: Inline Debugger, GDB Server, Symbol Table
	\item Host hardware abstraction layer: "Simple DirectMedia Layer" wrapper (SDL)
	\item Operating System Application Binary Interface Translation (OS ABI): LinuxOS (ARM/PowerPC Linux system call translator)
	\item Host Time abstraction layer: host\_time
	\item Simulated Time abstraction layer: sc\_time
	\item Power Estimation: Cache/SRAM Power estimator based on CACTI analytical power/layout estimation tool
\end{itemize}

% \newpage
% \begin{center}
% 	\begin{tabular}{|p{7.5cm}|p{7.5cm}|}
% 		\hline
% 		\multicolumn{2}{|l|}{\textbf{\Large Service Sample}}\\
% 		\hline
% 		\multicolumn{1}{|p{7.5cm}}{\textbf{Class Name:} \newline \texttt{sample::sample::sample}\newline$\hookrightarrow$\texttt{::sample}} & \multicolumn{1}{p{7.5cm}|}{\textbf{Header:} \newline \texttt{sample/sample/sample}\newline$\hookrightarrow$\texttt{/sample.hh}}\\
% 		\multicolumn{2}{|l|}{}\\
% 		\multicolumn{2}{|p{15cm}|}{\textbf{Description:} \newline Sample.}\\
% 		\hline
% 		\hline
% 		\multicolumn{2}{|c|}{\textbf{\large Template Parameters}}\\
% 		\hline
% 		\multicolumn{1}{|p{7.5cm}}{\textbf{Name:} \texttt{sample}} & \multicolumn{1}{p{7.5cm}|}{\textbf{Type:} \texttt{sample}}\\
% 		\multicolumn{2}{|p{15cm}|}{\textbf{Default value:} \texttt{sample}}\\
% 		\multicolumn{2}{|l|}{}\\
% 		\multicolumn{2}{|p{15cm}|}{\textbf{Description:} \newline Sample.}\\
% 		\hline
% 		\hline
% 		\multicolumn{2}{|c|}{\textbf{\large Run-Time Parameters}}\\
% 		\hline
% 		\multicolumn{1}{|p{7.5cm}}{\textbf{Name:} \texttt{sample}} & \multicolumn{1}{p{7.5cm}|}{\textbf{Type:} \texttt{sample}}\\
% 		\multicolumn{2}{|p{15cm}|}{\textbf{Default value:} \texttt{sample}}\\
% 		\multicolumn{2}{|l|}{}\\
% 		\multicolumn{2}{|p{15cm}|}{\textbf{Description:} \newline Sample.}\\
% 		\hline
% 		\hline
% 		\multicolumn{2}{|c|}{\textbf{\large Service Exports}}\\
% 		\hline
% 		\multicolumn{1}{|p{7.5cm}}{\textbf{Name:} \texttt{sample}} & \multicolumn{1}{p{7.5cm}|}{\textbf{Interface:} \newline \texttt{sample} \newline$\hookrightarrow$\texttt{sample}}\\
% 		\multicolumn{2}{|l|}{}\\
% 		\multicolumn{2}{|p{15cm}|}{\textbf{Description:} \newline Sample.}\\
% 		\hline
% 		\hline
% 		\multicolumn{2}{|c|}{\textbf{\large Service Imports}}\\
% 		\hline
% 		\multicolumn{1}{|p{7.5cm}}{\textbf{Name:} \texttt{sample} \newline \textbf{Mandatory connected:} yes/no} & \multicolumn{1}{p{7.5cm}|}{\textbf{Interface:} \newline \texttt{sample::sample::sample} \newline$\hookrightarrow$\texttt{::sample}}\\
% 		\multicolumn{2}{|l|}{}\\
% 		\multicolumn{2}{|p{15cm}|}{\textbf{Description:} \newline Sample.}\\
% 		\hline
% 	\end{tabular}
% \end{center}

\subsection{Loaders}

\newpage
\begin{center}
	\begin{tabular}{|p{7.5cm}|p{7.5cm}|}
		\hline
		\multicolumn{2}{|l|}{\textbf{\Large Service ELF Loader}}\\
		\hline
		\multicolumn{1}{|p{7.5cm}}{\textbf{Class Name:} \newline \texttt{unisim::service::loader::elf\_loader}\newline$\hookrightarrow$\texttt{::ElfLoaderImpl}} & \multicolumn{1}{p{7.5cm}|}{\textbf{Header:} \newline \texttt{unisim/service/loader/elf\_loader}\newline$\hookrightarrow$\texttt{/elf\_loader.hh}}\\
		\multicolumn{2}{|l|}{}\\
		\multicolumn{2}{|p{15cm}|}{\textbf{Description:} \newline The ELF loader service allows to load a binary program into a memory and fill a symbol table.
		The loader also provides information about the loaded file such as the code and data locations (base address and size). The ELF loader loads the program during setup. \texttt{Elf32Loader} and \texttt{Elf64Loader} are two specialized versions of class \texttt{ElfLoaderImpl<>} which the user should use whenever possible.}\\
		\hline
		\hline
		\multicolumn{2}{|l|}{\textbf{Template Parameters}}\\
		\hline
		\multicolumn{1}{|p{7.5cm}}{\textbf{Name:} \texttt{MEMORY\_ADDR}} & \multicolumn{1}{p{7.5cm}|}{\textbf{Type:} \texttt{class}}\\
		\multicolumn{2}{|p{15cm}|}{\textbf{Default value:} none}\\
		\multicolumn{2}{|l|}{}\\
		\multicolumn{2}{|p{15cm}|}{\textbf{Description:} \newline This is the C++ type of a memory address (e.g. uint32\_t or uint64\_t).}\\
		\hline
		\multicolumn{1}{|p{7.5cm}}{\textbf{Name:} \texttt{T}} & \multicolumn{1}{p{7.5cm}|}{\textbf{Type:} \texttt{class}}\\
		\multicolumn{2}{|p{15cm}|}{\textbf{Default value:} none}\\
		\multicolumn{2}{|l|}{}\\
		\multicolumn{2}{|p{15cm}|}{\textbf{Description:} \newline We should consider removing this parameter or have a reason to have it.}\\
		\hline
		\multicolumn{1}{|p{7.5cm}}{\textbf{Name:} \texttt{ElfClass}} & \multicolumn{1}{p{7.5cm}|}{\textbf{Type:} \texttt{unsigned int}}\\
		\multicolumn{2}{|p{15cm}|}{\textbf{Default value:} none}\\
		\multicolumn{2}{|l|}{}\\
		\multicolumn{2}{|p{15cm}|}{\textbf{Description:} \newline class of machine (either \texttt{ELFCLASS32} or \texttt{ELFCLASS64}).}\\
		\hline
		\multicolumn{1}{|p{7.5cm}}{\textbf{Name:} \texttt{Elf\_Ehdr}} & \multicolumn{1}{p{7.5cm}|}{\textbf{Type:} \texttt{class}}\\
		\multicolumn{2}{|p{15cm}|}{\textbf{Default value:} none}\\
		\multicolumn{2}{|l|}{}\\
		\multicolumn{2}{|p{15cm}|}{\textbf{Description:} \newline ELF Header structure (either Elf32\_Ehdr or Elf64\_Ehdr).}\\
		\hline
		\multicolumn{1}{|p{7.5cm}}{\textbf{Name:} \texttt{Elf\_Phdr}} & \multicolumn{1}{p{7.5cm}|}{\textbf{Type:} \texttt{class}}\\
		\multicolumn{2}{|p{15cm}|}{\textbf{Default value:} none}\\
		\multicolumn{2}{|l|}{}\\
		\multicolumn{2}{|p{15cm}|}{\textbf{Description:} \newline ELF Program Header structure (either Elf32\_Phdr or Elf64\_Phdr).}\\
		\hline
		\multicolumn{1}{|p{7.5cm}}{\textbf{Name:} \texttt{Elf\_Shdr}} & \multicolumn{1}{p{7.5cm}|}{\textbf{Type:} \texttt{class}}\\
		\multicolumn{2}{|p{15cm}|}{\textbf{Default value:} none}\\
		\multicolumn{2}{|l|}{}\\
		\multicolumn{2}{|p{15cm}|}{\textbf{Description:} \newline ELF Section Header structure (either \texttt{Elf32\_Shdr} or \texttt{Elf64\_Shdr}).}\\
		\hline
		\multicolumn{1}{|p{7.5cm}}{\textbf{Name:} \texttt{Elf\_Sym}} & \multicolumn{1}{p{7.5cm}|}{\textbf{Type:} \texttt{class}}\\
		\multicolumn{2}{|p{15cm}|}{\textbf{Default value:} none}\\
		\multicolumn{2}{|l|}{}\\
		\multicolumn{2}{|p{15cm}|}{\textbf{Description:} \newline ELF Symbol structure (either \texttt{Elf32\_Sym} or \texttt{Elf64\_Sym}).}\\
		\hline
	\end{tabular}
\end{center}

\begin{center}
	\begin{tabular}{|p{7.5cm}|p{7.5cm}|}
		\hline
		\multicolumn{2}{|c|}{\textbf{\large Run-Time Parameters}}\\
		\hline
		\multicolumn{1}{|p{7.5cm}}{\textbf{Name:} \texttt{filename}} & \multicolumn{1}{p{7.5cm}|}{\textbf{Type:} \texttt{string}}\\
		\multicolumn{2}{|p{15cm}|}{\textbf{Default value:} empty string}\\
		\multicolumn{2}{|l|}{}\\
		\multicolumn{2}{|p{15cm}|}{\textbf{Description:} \newline The ELF file name to load into the connected memory.}\\
		\hline
		\multicolumn{1}{|p{7.5cm}}{\textbf{Name:} \texttt{base-addr}} & \multicolumn{1}{p{7.5cm}|}{\textbf{Type:} \texttt{MEMORY\_ADDR}}\\
		\multicolumn{2}{|p{15cm}|}{\textbf{Default value:} \texttt{0}}\\
		\multicolumn{2}{|l|}{}\\
		\multicolumn{2}{|p{15cm}|}{\textbf{Description:} \newline If this parameter is non-zero, force the ELF Loader to load the unique program segment at this base address.}\\
		\hline
		\multicolumn{1}{|p{7.5cm}}{\textbf{Name:} \texttt{force-use-virtual-address}} & \multicolumn{1}{p{7.5cm}|}{\textbf{Type:} \texttt{bool}}\\
		\multicolumn{2}{|p{15cm}|}{\textbf{Default value:} \texttt{false}}\\
		\multicolumn{2}{|l|}{}\\
		\multicolumn{2}{|p{15cm}|}{\textbf{Description:} \newline If true, this parameter forces the ELF loader to use the segment virtual address instead of the segment physical address when loading the segments into memory.}\\
		\hline
		\hline
		\multicolumn{2}{|c|}{\textbf{\large Service Exports}}\\
		\hline
		\multicolumn{1}{|p{7.5cm}}{\textbf{Name:} \texttt{logger\_export}} & \multicolumn{1}{p{7.5cm}|}{\textbf{Interface:} \newline \texttt{unisim::service::interfaces::} \newline$\hookrightarrow$\texttt{Loader<MEMORY\_ADDR>}}\\
		\multicolumn{2}{|l|}{}\\
		\multicolumn{2}{|p{15cm}|}{\textbf{Description:} \newline The ELF loader provides information about the code and data location through this export.}\\
		\hline
		\hline
		\multicolumn{2}{|c|}{\textbf{\large Service Imports}}\\
		\hline
		\multicolumn{1}{|p{7.5cm}}{\textbf{Name:} \texttt{memory\_import}} & \multicolumn{1}{p{7.5cm}|}{\textbf{Interface:} \newline \texttt{unisim::service::interfaces::} \newline$\hookrightarrow$\texttt{Memory<MEMORY\_ADDR>}}\\
		\multicolumn{2}{|p{15cm}|}{\textbf{Mandatory connected:} no}\\
		\multicolumn{2}{|l|}{}\\
		\multicolumn{2}{|p{15cm}|}{\textbf{Description:} \newline The ELF loader accesses to the memory through this import.}\\
		\hline
		\multicolumn{1}{|p{7.5cm}}{\textbf{Name:} \texttt{symbol\_table\_build\_import}} & \multicolumn{1}{p{7.5cm}|}{\textbf{Interface:} \newline \texttt{unisim::service::interfaces::} \newline$\hookrightarrow$\texttt{SymbolTableBuild<MEMORY\_ADDR>}}\\
		\multicolumn{2}{|p{15cm}|}{\textbf{Mandatory connected:} no}\\
		\multicolumn{2}{|l|}{}\\
		\multicolumn{2}{|p{15cm}|}{\textbf{Description:} \newline The ELF loader fills the symbol table using this import.}\\
		\hline
	\end{tabular}
\end{center}

\newpage
\begin{center}
	\begin{tabular}{|p{7.5cm}|p{7.5cm}|}
		\hline
		\multicolumn{2}{|l|}{\textbf{\Large Service ELF32 Loader}}\\
		\hline
		\multicolumn{1}{|p{7.5cm}}{\textbf{Class Name:} \newline \texttt{unisim::service::loader::elf\_loader}\newline$\hookrightarrow$\texttt{::Elf32Loader}} & \multicolumn{1}{p{7.5cm}|}{\textbf{Header:} \newline \texttt{unisim/service/loader/elf\_loader}\newline$\hookrightarrow$\texttt{/elf\_loader.hh}}\\
		\multicolumn{2}{|l|}{}\\
		\multicolumn{2}{|p{15cm}|}{\textbf{Description:} \newline The ELF32 loader service allows to load an ELF32 binary program into a memory and fill a symbol table. The loader also provides information about the loaded file such as the code and data locations (base address and size). The ELF loader loads the program during setup.}\\
		\hline
		\hline
		\multicolumn{2}{|c|}{\textbf{\large Run-Time Parameters}}\\
		\hline
		\multicolumn{1}{|p{7.5cm}}{\textbf{Name:} \texttt{filename}} & \multicolumn{1}{p{7.5cm}|}{\textbf{Type:} \texttt{string}}\\
		\multicolumn{2}{|p{15cm}|}{\textbf{Default value:} empty string}\\
		\multicolumn{2}{|l|}{}\\
		\multicolumn{2}{|p{15cm}|}{\textbf{Description:} \newline The ELF file name to load into the connected memory.}\\
		\hline
		\multicolumn{1}{|p{7.5cm}}{\textbf{Name:} \texttt{base-addr}} & \multicolumn{1}{p{7.5cm}|}{\textbf{Type:} \texttt{uint32\_t}}\\
		\multicolumn{2}{|p{15cm}|}{\textbf{Default value:} \texttt{0}}\\
		\multicolumn{2}{|l|}{}\\
		\multicolumn{2}{|p{15cm}|}{\textbf{Description:} \newline If this parameter is non-zero, force the ELF Loader to load the unique program segment at this base address.}\\
		\hline
		\multicolumn{1}{|p{7.5cm}}{\textbf{Name:} \texttt{force-use-virtual-address}} & \multicolumn{1}{p{7.5cm}|}{\textbf{Type:} \texttt{bool}}\\
		\multicolumn{2}{|p{15cm}|}{\textbf{Default value:} \texttt{false}}\\
		\multicolumn{2}{|l|}{}\\
		\multicolumn{2}{|p{15cm}|}{\textbf{Description:} \newline If true, this parameter forces the ELF loader to use the segment virtual address instead of the segment physical address when loading the segments into memory.}\\
		\hline
		\hline
		\multicolumn{2}{|c|}{\textbf{\large Service Exports}}\\
		\hline
		\multicolumn{1}{|p{7.5cm}}{\textbf{Name:} \texttt{logger\_export}} & \multicolumn{1}{p{7.5cm}|}{\textbf{Interface:} \newline \texttt{unisim::service::interfaces::} \newline$\hookrightarrow$\texttt{Loader<uint32\_t>}}\\
		\multicolumn{2}{|l|}{}\\
		\multicolumn{2}{|p{15cm}|}{\textbf{Description:} \newline The ELF loader provides information about the code and data location through this export.}\\
		\hline
		\hline
		\multicolumn{2}{|c|}{\textbf{\large Service Imports}}\\
		\hline
		\multicolumn{1}{|p{7.5cm}}{\textbf{Name:} \texttt{memory\_import} \newline \textbf{Mandatory connected:} no} & \multicolumn{1}{p{7.5cm}|}{\textbf{Interface:} \newline \texttt{unisim::service::interfaces::} \newline$\hookrightarrow$\texttt{Memory<uint32\_t>}}\\
		\multicolumn{2}{|l|}{}\\
		\multicolumn{2}{|p{15cm}|}{\textbf{Description:} \newline The ELF loader accesses to the memory through this import.}\\
		\hline
		\multicolumn{1}{|p{7.5cm}}{\textbf{Name:} \texttt{symbol\_table\_build\_import} \newline \textbf{Mandatory connected:} no} & \multicolumn{1}{p{7.5cm}|}{\textbf{Interface:} \newline \texttt{unisim::service::interfaces::} \newline$\hookrightarrow$\texttt{SymbolTableBuild<uint32\_t>}}\\
		\multicolumn{2}{|l|}{}\\
		\multicolumn{2}{|p{15cm}|}{\textbf{Description:} \newline The ELF loader fills the symbol table using this import.}\\
		\hline
	\end{tabular}
\end{center}

\newpage
\begin{center}
	\begin{tabular}{|p{7.5cm}|p{7.5cm}|}
		\hline
		\multicolumn{2}{|l|}{\textbf{\Large Service ELF64 Loader}}\\
		\hline
		\multicolumn{1}{|p{7.5cm}}{\textbf{Class Name:} \newline \texttt{unisim::service::loader::elf\_loader}\newline$\hookrightarrow$\texttt{::Elf32Loader}} & \multicolumn{1}{p{7.5cm}|}{\textbf{Header:} \newline \texttt{unisim/service/loader/elf\_loader}\newline$\hookrightarrow$\texttt{/elf\_loader.hh}}\\
		\multicolumn{2}{|l|}{}\\
		\multicolumn{2}{|p{15cm}|}{\textbf{Description:} \newline The ELF32 loader service allows to load an ELF64 binary program into a memory and fill a symbol table. The loader also provides information about the loaded file such as the code and data locations (base address and size). The ELF loader loads the program during setup.}\\
		\hline
		\hline
		\multicolumn{2}{|c|}{\textbf{\large Run-Time Parameters}}\\
		\hline
		\multicolumn{1}{|p{7.5cm}}{\textbf{Name:} \texttt{filename}} & \multicolumn{1}{p{7.5cm}|}{\textbf{Type:} \texttt{string}}\\
		\multicolumn{2}{|p{15cm}|}{\textbf{Default value:} empty string}\\
		\multicolumn{2}{|l|}{}\\
		\multicolumn{2}{|p{15cm}|}{\textbf{Description:} \newline The ELF file name to load into the connected memory.}\\
		\hline
		\multicolumn{1}{|p{7.5cm}}{\textbf{Name:} \texttt{base-addr}} & \multicolumn{1}{p{7.5cm}|}{\textbf{Type:} \texttt{uint64\_t}}\\
		\multicolumn{2}{|p{15cm}|}{\textbf{Default value:} \texttt{0}}\\
		\multicolumn{2}{|l|}{}\\
		\multicolumn{2}{|p{15cm}|}{\textbf{Description:} \newline If this parameter is non-zero, force the ELF Loader to load the unique program segment at this base address.}\\
		\hline
		\multicolumn{1}{|p{7.5cm}}{\textbf{Name:} \texttt{force-use-virtual-address}} & \multicolumn{1}{p{7.5cm}|}{\textbf{Type:} \texttt{bool}}\\
		\multicolumn{2}{|p{15cm}|}{\textbf{Default value:} \texttt{false}}\\
		\multicolumn{2}{|l|}{}\\
		\multicolumn{2}{|p{15cm}|}{\textbf{Description:} \newline If true, this parameter forces the ELF loader to use the segment virtual address instead of the segment physical address when loading the segments into memory.}\\
		\hline
		\hline
		\multicolumn{2}{|c|}{\textbf{\large Service Exports}}\\
		\hline
		\multicolumn{1}{|p{7.5cm}}{\textbf{Name:} \texttt{logger\_export}} & \multicolumn{1}{p{7.5cm}|}{\textbf{Interface:} \newline \texttt{unisim::service::interfaces::} \newline$\hookrightarrow$\texttt{Loader<uint64\_t>}}\\
		\multicolumn{2}{|l|}{}\\
		\multicolumn{2}{|p{15cm}|}{\textbf{Description:} \newline The ELF loader provides information about the code and data location through this export.}\\
		\hline
		\hline
		\multicolumn{2}{|c|}{\textbf{\large Service Imports}}\\
		\hline
		\multicolumn{1}{|p{7.5cm}}{\textbf{Name:} \texttt{memory\_import} \newline \textbf{Mandatory connected:} no} & \multicolumn{1}{p{7.5cm}|}{\textbf{Interface:} \newline \texttt{unisim::service::interfaces::} \newline$\hookrightarrow$\texttt{Memory<uint64\_t>}}\\
		\multicolumn{2}{|l|}{}\\
		\multicolumn{2}{|p{15cm}|}{\textbf{Description:} \newline The ELF loader accesses to the memory through this import.}\\
		\hline
		\multicolumn{1}{|p{7.5cm}}{\textbf{Name:} \texttt{symbol\_table\_build\_import} \newline \textbf{Mandatory connected:} no} & \multicolumn{1}{p{7.5cm}|}{\textbf{Interface:} \newline \texttt{unisim::service::interfaces::} \newline$\hookrightarrow$\texttt{SymbolTableBuild<uint64\_t>}}\\
		\multicolumn{2}{|l|}{}\\
		\multicolumn{2}{|p{15cm}|}{\textbf{Description:} \newline The ELF loader fills the symbol table using this import.}\\
		\hline
	\end{tabular}
\end{center}

\newpage
\begin{center}
	\begin{tabular}{|p{7.5cm}|p{7.5cm}|}
		\hline
		\multicolumn{2}{|l|}{\textbf{\Large Service PowerMac Linux Kernel Loader}}\\
		\hline
		\multicolumn{1}{|p{7.5cm}}{\textbf{Class Name:} \newline \texttt{sample::sample::sample}\newline$\hookrightarrow$\texttt{::sample}} & \multicolumn{1}{p{7.5cm}|}{\textbf{Header:} \newline \texttt{sample/sample/sample}\newline$\hookrightarrow$\texttt{/sample.hh}}\\
		\multicolumn{2}{|l|}{}\\
		\multicolumn{2}{|p{15cm}|}{\textbf{Description:} \newline Sample.}\\
		\hline
		\hline
		\multicolumn{2}{|c|}{\textbf{\large Template Parameters}}\\
		\hline
		\multicolumn{1}{|p{7.5cm}}{\textbf{Name:} \texttt{sample}} & \multicolumn{1}{p{7.5cm}|}{\textbf{Type:} \texttt{sample}}\\
		\multicolumn{2}{|p{15cm}|}{\textbf{Default value:} \texttt{sample}}\\
		\multicolumn{2}{|l|}{}\\
		\multicolumn{2}{|p{15cm}|}{\textbf{Description:} \newline Sample.}\\
		\hline
		\hline
		\multicolumn{2}{|c|}{\textbf{\large Run-Time Parameters}}\\
		\hline
		\multicolumn{1}{|p{7.5cm}}{\textbf{Name:} \texttt{sample}} & \multicolumn{1}{p{7.5cm}|}{\textbf{Type:} \texttt{sample}}\\
		\multicolumn{2}{|p{15cm}|}{\textbf{Default value:} \texttt{sample}}\\
		\multicolumn{2}{|l|}{}\\
		\multicolumn{2}{|p{15cm}|}{\textbf{Description:} \newline Sample.}\\
		\hline
		\hline
		\multicolumn{2}{|c|}{\textbf{\large Service Exports}}\\
		\hline
		\multicolumn{1}{|p{7.5cm}}{\textbf{Name:} \texttt{sample}} & \multicolumn{1}{p{7.5cm}|}{\textbf{Interface:} \newline \texttt{sample} \newline$\hookrightarrow$\texttt{sample}}\\
		\multicolumn{2}{|l|}{}\\
		\multicolumn{2}{|p{15cm}|}{\textbf{Description:} \newline Sample.}\\
		\hline
		\hline
		\multicolumn{2}{|c|}{\textbf{\large Service Imports}}\\
		\hline
		\multicolumn{1}{|p{7.5cm}}{\textbf{Name:} \texttt{sample} \newline \textbf{Mandatory connected:} yes/no} & \multicolumn{1}{p{7.5cm}|}{\textbf{Interface:} \newline \texttt{sample::sample::sample} \newline$\hookrightarrow$\texttt{::sample}}\\
		\multicolumn{2}{|l|}{}\\
		\multicolumn{2}{|p{15cm}|}{\textbf{Description:} \newline Sample.}\\
		\hline
	\end{tabular}
\end{center}

\newpage
\begin{center}
	\begin{tabular}{|p{7.5cm}|p{7.5cm}|}
		\hline
		\multicolumn{2}{|l|}{\textbf{\Large Service PowerMac BootX Loader}}\\
		\hline
		\multicolumn{1}{|p{7.5cm}}{\textbf{Class Name:} \newline \texttt{sample::sample::sample}\newline$\hookrightarrow$\texttt{::sample}} & \multicolumn{1}{p{7.5cm}|}{\textbf{Header:} \newline \texttt{sample/sample/sample}\newline$\hookrightarrow$\texttt{/sample.hh}}\\
		\multicolumn{2}{|l|}{}\\
		\multicolumn{2}{|p{15cm}|}{\textbf{Description:} \newline Sample.}\\
		\hline
		\hline
		\multicolumn{2}{|c|}{\textbf{\large Template Parameters}}\\
		\hline
		\multicolumn{1}{|p{7.5cm}}{\textbf{Name:} \texttt{sample}} & \multicolumn{1}{p{7.5cm}|}{\textbf{Type:} \texttt{sample}}\\
		\multicolumn{2}{|p{15cm}|}{\textbf{Default value:} \texttt{sample}}\\
		\multicolumn{2}{|l|}{}\\
		\multicolumn{2}{|p{15cm}|}{\textbf{Description:} \newline Sample.}\\
		\hline
		\hline
		\multicolumn{2}{|c|}{\textbf{\large Run-Time Parameters}}\\
		\hline
		\multicolumn{1}{|p{7.5cm}}{\textbf{Name:} \texttt{sample}} & \multicolumn{1}{p{7.5cm}|}{\textbf{Type:} \texttt{sample}}\\
		\multicolumn{2}{|p{15cm}|}{\textbf{Default value:} \texttt{sample}}\\
		\multicolumn{2}{|l|}{}\\
		\multicolumn{2}{|p{15cm}|}{\textbf{Description:} \newline Sample.}\\
		\hline
		\hline
		\multicolumn{2}{|c|}{\textbf{\large Service Exports}}\\
		\hline
		\multicolumn{1}{|p{7.5cm}}{\textbf{Name:} \texttt{sample}} & \multicolumn{1}{p{7.5cm}|}{\textbf{Interface:} \newline \texttt{sample} \newline$\hookrightarrow$\texttt{sample}}\\
		\multicolumn{2}{|l|}{}\\
		\multicolumn{2}{|p{15cm}|}{\textbf{Description:} \newline Sample.}\\
		\hline
		\hline
		\multicolumn{2}{|c|}{\textbf{\large Service Imports}}\\
		\hline
		\multicolumn{1}{|p{7.5cm}}{\textbf{Name:} \texttt{sample} \newline \textbf{Mandatory connected:} yes/no} & \multicolumn{1}{p{7.5cm}|}{\textbf{Interface:} \newline \texttt{sample::sample::sample} \newline$\hookrightarrow$\texttt{::sample}}\\
		\multicolumn{2}{|l|}{}\\
		\multicolumn{2}{|p{15cm}|}{\textbf{Description:} \newline Sample.}\\
		\hline
	\end{tabular}
\end{center}

\newpage
\begin{center}
	\begin{tabular}{|p{7.5cm}|p{7.5cm}|}
		\hline
		\multicolumn{2}{|l|}{\textbf{\Large Service S19 Loader}}\\
		\hline
		\multicolumn{1}{|p{7.5cm}}{\textbf{Class Name:} \newline \texttt{sample::sample::sample}\newline$\hookrightarrow$\texttt{::sample}} & \multicolumn{1}{p{7.5cm}|}{\textbf{Header:} \newline \texttt{sample/sample/sample}\newline$\hookrightarrow$\texttt{/sample.hh}}\\
		\multicolumn{2}{|l|}{}\\
		\multicolumn{2}{|p{15cm}|}{\textbf{Description:} \newline Sample.}\\
		\hline
		\hline
		\multicolumn{2}{|c|}{\textbf{\large Template Parameters}}\\
		\hline
		\multicolumn{1}{|p{7.5cm}}{\textbf{Name:} \texttt{sample}} & \multicolumn{1}{p{7.5cm}|}{\textbf{Type:} \texttt{sample}}\\
		\multicolumn{2}{|p{15cm}|}{\textbf{Default value:} \texttt{sample}}\\
		\multicolumn{2}{|l|}{}\\
		\multicolumn{2}{|p{15cm}|}{\textbf{Description:} \newline Sample.}\\
		\hline
		\hline
		\multicolumn{2}{|c|}{\textbf{\large Run-Time Parameters}}\\
		\hline
		\multicolumn{1}{|p{7.5cm}}{\textbf{Name:} \texttt{sample}} & \multicolumn{1}{p{7.5cm}|}{\textbf{Type:} \texttt{sample}}\\
		\multicolumn{2}{|p{15cm}|}{\textbf{Default value:} \texttt{sample}}\\
		\multicolumn{2}{|l|}{}\\
		\multicolumn{2}{|p{15cm}|}{\textbf{Description:} \newline Sample.}\\
		\hline
		\hline
		\multicolumn{2}{|c|}{\textbf{\large Service Exports}}\\
		\hline
		\multicolumn{1}{|p{7.5cm}}{\textbf{Name:} \texttt{sample}} & \multicolumn{1}{p{7.5cm}|}{\textbf{Interface:} \newline \texttt{sample} \newline$\hookrightarrow$\texttt{sample}}\\
		\multicolumn{2}{|l|}{}\\
		\multicolumn{2}{|p{15cm}|}{\textbf{Description:} \newline Sample.}\\
		\hline
		\hline
		\multicolumn{2}{|c|}{\textbf{\large Service Imports}}\\
		\hline
		\multicolumn{1}{|p{7.5cm}}{\textbf{Name:} \texttt{sample} \newline \textbf{Mandatory connected:} yes/no} & \multicolumn{1}{p{7.5cm}|}{\textbf{Interface:} \newline \texttt{sample::sample::sample} \newline$\hookrightarrow$\texttt{::sample}}\\
		\multicolumn{2}{|l|}{}\\
		\multicolumn{2}{|p{15cm}|}{\textbf{Description:} \newline Sample.}\\
		\hline
	\end{tabular}
\end{center}

\subsection{Operating System Application Binary Interface (OS ABI)}

\newpage
\begin{center}
	\begin{tabular}{|p{7.5cm}|p{7.5cm}|}
		\hline
		\multicolumn{2}{|l|}{\textbf{\Large Service Linux System Call Translator: LinuxOS}}\\
		\hline
		\multicolumn{1}{|p{7.5cm}}{\textbf{Class Name:} \newline \texttt{sample::sample::sample}\newline$\hookrightarrow$\texttt{::sample}} & \multicolumn{1}{p{7.5cm}|}{\textbf{Header:} \newline \texttt{sample/sample/sample}\newline$\hookrightarrow$\texttt{/sample.hh}}\\
		\multicolumn{2}{|l|}{}\\
		\multicolumn{2}{|p{15cm}|}{\textbf{Description:} \newline Sample.}\\
		\hline
		\hline
		\multicolumn{2}{|c|}{\textbf{\large Template Parameters}}\\
		\hline
		\multicolumn{1}{|p{7.5cm}}{\textbf{Name:} \texttt{sample}} & \multicolumn{1}{p{7.5cm}|}{\textbf{Type:} \texttt{sample}}\\
		\multicolumn{2}{|p{15cm}|}{\textbf{Default value:} \texttt{sample}}\\
		\multicolumn{2}{|l|}{}\\
		\multicolumn{2}{|p{15cm}|}{\textbf{Description:} \newline Sample.}\\
		\hline
		\hline
		\multicolumn{2}{|c|}{\textbf{\large Run-Time Parameters}}\\
		\hline
		\multicolumn{1}{|p{7.5cm}}{\textbf{Name:} \texttt{sample}} & \multicolumn{1}{p{7.5cm}|}{\textbf{Type:} \texttt{sample}}\\
		\multicolumn{2}{|p{15cm}|}{\textbf{Default value:} \texttt{sample}}\\
		\multicolumn{2}{|l|}{}\\
		\multicolumn{2}{|p{15cm}|}{\textbf{Description:} \newline Sample.}\\
		\hline
		\hline
		\multicolumn{2}{|c|}{\textbf{\large Service Exports}}\\
		\hline
		\multicolumn{1}{|p{7.5cm}}{\textbf{Name:} \texttt{sample}} & \multicolumn{1}{p{7.5cm}|}{\textbf{Interface:} \newline \texttt{sample} \newline$\hookrightarrow$\texttt{sample}}\\
		\multicolumn{2}{|l|}{}\\
		\multicolumn{2}{|p{15cm}|}{\textbf{Description:} \newline Sample.}\\
		\hline
		\hline
		\multicolumn{2}{|c|}{\textbf{\large Service Imports}}\\
		\hline
		\multicolumn{1}{|p{7.5cm}}{\textbf{Name:} \texttt{sample} \newline \textbf{Mandatory connected:} yes/no} & \multicolumn{1}{p{7.5cm}|}{\textbf{Interface:} \newline \texttt{sample::sample::sample} \newline$\hookrightarrow$\texttt{::sample}}\\
		\multicolumn{2}{|l|}{}\\
		\multicolumn{2}{|p{15cm}|}{\textbf{Description:} \newline Sample.}\\
		\hline
	\end{tabular}
\end{center}

\subsection{Debug}

\newpage
\begin{center}
	\begin{tabular}{|p{7.5cm}|p{7.5cm}|}
		\hline
		\multicolumn{2}{|l|}{\textbf{\Large Service Inline Debugger}}\\
		\hline
		\multicolumn{1}{|p{7.5cm}}{\textbf{Class Name:} \newline \texttt{unisim::service::debug}\newline$\hookrightarrow$\texttt{::inline\_debugger::InlineDebugger}} & \multicolumn{1}{p{7.5cm}|}{\textbf{Header:} \newline \texttt{unisim/service/debug}\newline$\hookrightarrow$\texttt{/inline\_debugger/inline\_debugger.hh}}\\
		\multicolumn{2}{|l|}{}\\
		\multicolumn{2}{|p{15cm}|}{\textbf{Description:} \newline The inline debugger service provides the user with a simple text-based interface to interactively debug a target application running on a CPU module. The debug is at the instruction level. The inline debugger may be connected to a CPU module and (optionally) to a symbol table.}\\
		\hline
		\hline
		\multicolumn{2}{|c|}{\textbf{\large Template Parameters}}\\
		\hline
		\multicolumn{1}{|p{7.5cm}}{\textbf{Name:} \texttt{ADDRESS}} & \multicolumn{1}{p{7.5cm}|}{\textbf{Type:} \texttt{class}}\\
		\multicolumn{2}{|p{15cm}|}{\textbf{Default value:} none}\\
		\multicolumn{2}{|l|}{}\\
		\multicolumn{2}{|p{15cm}|}{\textbf{Description:} \newline This is the C++ type of a memory address (e.g. uint32\_t or uint64\_t).}\\
		\hline
		\hline
		\multicolumn{2}{|c|}{\textbf{\large Service Exports}}\\
		\hline
		\multicolumn{1}{|p{7.5cm}}{\textbf{Name:} \texttt{debug\_control\_export}} & \multicolumn{1}{p{7.5cm}|}{\textbf{Interface:} \newline \texttt{unisim::service::interfaces} \newline$\hookrightarrow$\texttt{::DebugControl<ADDRESS>}}\\
		\multicolumn{2}{|l|}{}\\
		\multicolumn{2}{|p{15cm}|}{\textbf{Description:} \newline This service export should be connected to a CPU module. The CPU module calls method \texttt{FetchDebugCommand} through its service import to leave control to the debugger and fetch a new debug command.}\\
		\hline
		\multicolumn{1}{|p{7.5cm}}{\textbf{Name:} \texttt{memory\_access\_reporting\_export}} & \multicolumn{1}{p{7.5cm}|}{\textbf{Interface:} \newline \texttt{unisim::service::interfaces} \newline$\hookrightarrow$\texttt{MemoryAccessReporting<ADDRESS>}}\\
		\multicolumn{2}{|l|}{}\\
		\multicolumn{2}{|p{15cm}|}{\textbf{Description:} \newline This service export should be connected to a CPU module. The CPU module calls methods \texttt{ReportMemoryAccess} and \texttt{ReportFinishedInstruction} through its service import. This allows the debugger to spy memory accesses and thus handle breakpoints and watchpoints.}\\
		\hline
		\multicolumn{1}{|p{7.5cm}}{\textbf{Name:} \texttt{trap\_reporting\_export}} & \multicolumn{1}{p{7.5cm}|}{\textbf{Interface:} \newline \texttt{unisim::service::interfaces} \newline$\hookrightarrow$\texttt{::TrapReporting}}\\
		\multicolumn{2}{|l|}{}\\
		\multicolumn{2}{|p{15cm}|}{\textbf{Description:} \newline This service export should be connected to a CPU module. A CPU module calls method \texttt{ReportTrap} through its service import. This allows the debugger to break execution on the simulated CPU once a trap condition is detected by the CPU module.}\\
		\hline
	\end{tabular}
\end{center}

\newpage
\begin{center}
	\begin{tabular}{|p{7.5cm}|p{7.5cm}|}
		\hline
		\multicolumn{2}{|c|}{\textbf{\large Service Imports}}\\
		\hline
		\multicolumn{1}{|p{7.5cm}}{\textbf{Name:} \texttt{disasm\_import} \newline \textbf{Mandatory connected:} yes} & \multicolumn{1}{p{7.5cm}|}{\textbf{Interface:} \newline \texttt{unisim::service::interfaces} \newline$\hookrightarrow$\texttt{::Disassembly<ADDRESS>}}\\
		\multicolumn{2}{|l|}{}\\
		\multicolumn{2}{|p{15cm}|}{\textbf{Description:} \newline This service import should be connected to a CPU module. The CPU module should implements method \texttt{Disassemble} which provides the debugger with disassembling of the instructions.}\\
		\hline
		\multicolumn{1}{|p{7.5cm}}{\textbf{Name:} \texttt{memory\_import} \newline \textbf{Mandatory connected:} yes} & \multicolumn{1}{p{7.5cm}|}{\textbf{Interface:} \newline \texttt{unisim::service::interfaces} \newline$\hookrightarrow$\texttt{::Memory<ADDRESS>}}\\
		\multicolumn{2}{|l|}{}\\
		\multicolumn{2}{|p{15cm}|}{\textbf{Description:} \newline This service import should be connected to a CPU or a memory module. The debugger uses this service import to access to memory using methods \texttt{ReadMemory} and \texttt{WriteMemory}.}\\
		\hline
		\multicolumn{1}{|p{7.5cm}}{\textbf{Name:} \texttt{memory\_access\_reporting\newline$\hookrightarrow$\_control\_import} \newline \textbf{Mandatory connected:} no} & \multicolumn{1}{p{7.5cm}|}{\textbf{Interface:} \newline \texttt{unisim::service::interfaces} \newline$\hookrightarrow$\texttt{::MemoryAccessReportingControl}}\\
		\multicolumn{2}{|l|}{}\\
		\multicolumn{2}{|p{15cm}|}{\textbf{Description:} \newline This service import should be connected to a CPU module. The debugger calls methods \texttt{RequiresMemoryAccessReporting} and \texttt{RequiresFinishedInstructionReporting} through this service import to enable/disable memory access reporting from the CPU module.}\\
		\hline
		\multicolumn{1}{|p{7.5cm}}{\textbf{Name:} \texttt{registers\_import} \newline \textbf{Mandatory connected:} yes} & \multicolumn{1}{p{7.5cm}|}{\textbf{Interface:} \newline \texttt{unisim::service::interfaces} \newline$\hookrightarrow$\texttt{::Registers}}\\
		\multicolumn{2}{|l|}{}\\
		\multicolumn{2}{|p{15cm}|}{\textbf{Description:} \newline This service import should be connected to a CPU module. The debugger calls method \texttt{GetRegister} through this service import to get an interface to the CPU registers. The debugger uses methods \texttt{GetName}, \texttt{GetValue}, \texttt{GetSize} and \texttt{SetValue} of that interface to access to CPU registers.}\\
		\hline
		\multicolumn{1}{|p{7.5cm}}{\textbf{Name:} \texttt{symbol\_table\_lookup\_import} \newline \textbf{Mandatory connected:} no} & \multicolumn{1}{p{7.5cm}|}{\textbf{Interface:} \newline \texttt{unisim::service::interfaces} \newline$\hookrightarrow$\texttt{::SymbolTableLookup}}\\
		\multicolumn{2}{|l|}{}\\
		\multicolumn{2}{|p{15cm}|}{\textbf{Description:} \newline This service import should be connected to a symbol table. The debugger calls method \texttt{FindSymbol}, \texttt{FindSymbolByAddr}, \texttt{FindSymbolByName} through this service import to translate address to symbol and vice-versa.}\\
		\hline
	\end{tabular}
\end{center}

\newpage
\begin{center}
	\begin{tabular}{|p{7.5cm}|p{7.5cm}|}
		\hline
		\multicolumn{2}{|l|}{\textbf{\Large Service GDB Server}}\\
		\hline
		\multicolumn{1}{|p{7.5cm}}{\textbf{Class Name:} \newline \texttt{unisim::service::debug}\newline$\hookrightarrow$\texttt{::gdb\_server::GDBServer}} & \multicolumn{1}{p{7.5cm}|}{\textbf{Header:} \newline \texttt{unisim/service/debug}\newline$\hookrightarrow$\texttt{/gdb\_server/gdb\_server.hh}}\\
		\multicolumn{2}{|l|}{}\\
		\multicolumn{2}{|p{15cm}|}{\textbf{Description:} \newline The GDB server service allows debugging a software running on a simulated hardware by connecting (over TCP/IP) a GDB client to it (and thus to the simulator). The GDB client can be either the standard text based client (i.e. command \texttt{gdb}), a graphical front-end to GDB (e.g. \texttt{ddd}), or even Eclipse CDT. The GDB server service directly speaks the GDB serial remote protocol (over TCP/IP), so that a GDB client can connect (over TCP/IP) to the simulator using GDB command \texttt{target remote}. The GDB server service may be connected to a CPU module.}\\
		\hline
		\hline
		\multicolumn{2}{|c|}{\textbf{\large Template Parameters}}\\
		\hline
		\multicolumn{1}{|p{7.5cm}}{\textbf{Name:} \texttt{ADDRESS}} & \multicolumn{1}{p{7.5cm}|}{\textbf{Type:} \texttt{class}}\\
		\multicolumn{2}{|p{15cm}|}{\textbf{Default value:} none}\\
		\multicolumn{2}{|l|}{}\\
		\multicolumn{2}{|p{15cm}|}{\textbf{Description:} \newline This is the C++ type of a memory address (e.g. uint32\_t or uint64\_t).}\\
		\hline
		\hline
		\multicolumn{2}{|c|}{\textbf{\large Run-Time Parameters}}\\
		\hline
		\multicolumn{1}{|p{7.5cm}}{\textbf{Name:} \texttt{tcp-port}} & \multicolumn{1}{p{7.5cm}|}{\textbf{Type:} \texttt{int}}\\
		\multicolumn{2}{|p{15cm}|}{\textbf{Default value:} \texttt{12345}}\\
		\multicolumn{2}{|l|}{}\\
		\multicolumn{2}{|p{15cm}|}{\textbf{Description:} \newline The TCP port used by GDB server service to communicate with the GDB client.}\\
		\hline
		\multicolumn{1}{|p{7.5cm}}{\textbf{Name:} \texttt{architecture-description\newline$\hookrightarrow$-filename}} & \multicolumn{1}{p{7.5cm}|}{\textbf{Type:} \texttt{string}}\\
		\multicolumn{2}{|p{15cm}|}{\textbf{Default value:} empty string}\\
		\multicolumn{2}{|l|}{}\\
		\multicolumn{2}{|p{15cm}|}{\textbf{Description:} \newline The path to the architecture description file that the GDB server service must use. The description file provides retargetability to the GDB server service. The following files brings support of the ARM, PowerPC and HCS12X processors to the GDB server service: 
		\begin{itemize}
			\item \texttt{unisim/service/debug/gdb\_server/gdb\_armv4l.xml}
			\item \texttt{unisim/service/debug/gdb\_server/gdb\_armv5b.xml}
			\item \texttt{unisim/service/debug/gdb\_server/gdb\_powerpc.xml}
			\item \texttt{unisim/service/debug/gdb\_server/gdb\_hcs12x.xml}
		\end{itemize}
		}\\
		\hline
	\end{tabular}
\end{center}

\newpage
\begin{center}
	\begin{tabular}{|p{7.5cm}|p{7.5cm}|}
		\hline
		\multicolumn{2}{|c|}{\textbf{\large Service Exports}}\\
		\hline
		\multicolumn{1}{|p{7.5cm}}{\textbf{Name:} \texttt{debug\_control\_export}} & \multicolumn{1}{p{7.5cm}|}{\textbf{Interface:} \newline \texttt{unisim::service::interfaces} \newline$\hookrightarrow$\texttt{::DebugControl<ADDRESS>}}\\
		\multicolumn{2}{|l|}{}\\
		\multicolumn{2}{|p{15cm}|}{\textbf{Description:} \newline This service export should be connected to a CPU module. The CPU module calls method \texttt{FetchDebugCommand} through its service import to leave control to the debugger and fetch a new debug command.}\\
		\hline
		\multicolumn{1}{|p{7.5cm}}{\textbf{Name:} \texttt{memory\_access\_reporting\_export}} & \multicolumn{1}{p{7.5cm}|}{\textbf{Interface:} \newline \texttt{unisim::service::interfaces} \newline$\hookrightarrow$\texttt{MemoryAccessReporting<ADDRESS>}}\\
		\multicolumn{2}{|l|}{}\\
		\multicolumn{2}{|p{15cm}|}{\textbf{Description:} \newline This service export should be connected to a CPU module. The CPU module calls methods \texttt{ReportMemoryAccess} and \texttt{ReportFinishedInstruction} through its service import. This allows the debugger to spy memory accesses and thus handle breakpoints and watchpoints.}\\
		\hline
		\multicolumn{1}{|p{7.5cm}}{\textbf{Name:} \texttt{trap\_reporting\_export}} & \multicolumn{1}{p{7.5cm}|}{\textbf{Interface:} \newline \texttt{unisim::service::interfaces} \newline$\hookrightarrow$\texttt{::TrapReporting}}\\
		\multicolumn{2}{|l|}{}\\
		\multicolumn{2}{|p{15cm}|}{\textbf{Description:} \newline This service export should be connected to a CPU module. A CPU module calls method \texttt{ReportTrap} through its service import. This allows the debugger to break execution on the simulated CPU once a trap condition is detected by the CPU module.}\\
		\hline
		\hline
		\multicolumn{2}{|c|}{\textbf{\large Service Imports}}\\
		\hline
		\multicolumn{1}{|p{7.5cm}}{\textbf{Name:} \texttt{memory\_import} \newline \textbf{Mandatory connected:} yes} & \multicolumn{1}{p{7.5cm}|}{\textbf{Interface:} \newline \texttt{unisim::service::interfaces} \newline$\hookrightarrow$\texttt{::Memory<ADDRESS>}}\\
		\multicolumn{2}{|l|}{}\\
		\multicolumn{2}{|p{15cm}|}{\textbf{Description:} \newline This service import should be connected to a CPU or a memory module. The debugger uses this service import to access to memory using methods \texttt{ReadMemory} and \texttt{WriteMemory}.}\\
		\hline
		\multicolumn{1}{|p{7.5cm}}{\textbf{Name:} \texttt{memory\_access\_reporting\newline$\hookrightarrow$\_control\_import} \newline \textbf{Mandatory connected:} no} & \multicolumn{1}{p{7.5cm}|}{\textbf{Interface:} \newline \texttt{unisim::service::interfaces} \newline$\hookrightarrow$\texttt{::MemoryAccessReportingControl}}\\
		\multicolumn{2}{|l|}{}\\
		\multicolumn{2}{|p{15cm}|}{\textbf{Description:} \newline This service import should be connected to a CPU module. The debugger calls methods \texttt{RequiresMemoryAccessReporting} and \texttt{RequiresFinishedInstructionReporting} through this service import to enable/disable memory access reporting from the CPU module.}\\
		\hline
		\multicolumn{1}{|p{7.5cm}}{\textbf{Name:} \texttt{registers\_import} \newline \textbf{Mandatory connected:} yes} & \multicolumn{1}{p{7.5cm}|}{\textbf{Interface:} \newline \texttt{unisim::service::interfaces} \newline$\hookrightarrow$\texttt{::Registers}}\\
		\multicolumn{2}{|l|}{}\\
		\multicolumn{2}{|p{15cm}|}{\textbf{Description:} \newline This service import should be connected to a CPU module. The debugger calls method \texttt{GetRegister} through this service import to get an interface to the CPU registers. The debugger uses methods \texttt{GetName}, \texttt{GetValue}, \texttt{GetSize} and \texttt{SetValue} of that interface to access to CPU registers.}\\
		\hline
	\end{tabular}
\end{center}

\newpage
\begin{center}
	\begin{tabular}{|p{7.5cm}|p{7.5cm}|}
		\hline
		\multicolumn{2}{|l|}{\textbf{\Large Service Symbol Table}}\\
		\hline
		\multicolumn{1}{|p{7.5cm}}{\textbf{Class Name:} \newline \texttt{sample::sample::sample}\newline$\hookrightarrow$\texttt{::sample}} & \multicolumn{1}{p{7.5cm}|}{\textbf{Header:} \newline \texttt{sample/sample/sample}\newline$\hookrightarrow$\texttt{/sample.hh}}\\
		\multicolumn{2}{|l|}{}\\
		\multicolumn{2}{|p{15cm}|}{\textbf{Description:} \newline Sample.}\\
		\hline
		\hline
		\multicolumn{2}{|c|}{\textbf{\large Template Parameters}}\\
		\hline
		\multicolumn{1}{|p{7.5cm}}{\textbf{Name:} \texttt{sample}} & \multicolumn{1}{p{7.5cm}|}{\textbf{Type:} \texttt{sample}}\\
		\multicolumn{2}{|p{15cm}|}{\textbf{Default value:} \texttt{sample}}\\
		\multicolumn{2}{|l|}{}\\
		\multicolumn{2}{|p{15cm}|}{\textbf{Description:} \newline Sample.}\\
		\hline
		\hline
		\multicolumn{2}{|c|}{\textbf{\large Run-Time Parameters}}\\
		\hline
		\multicolumn{1}{|p{7.5cm}}{\textbf{Name:} \texttt{sample}} & \multicolumn{1}{p{7.5cm}|}{\textbf{Type:} \texttt{sample}}\\
		\multicolumn{2}{|p{15cm}|}{\textbf{Default value:} \texttt{sample}}\\
		\multicolumn{2}{|l|}{}\\
		\multicolumn{2}{|p{15cm}|}{\textbf{Description:} \newline Sample.}\\
		\hline
		\hline
		\multicolumn{2}{|c|}{\textbf{\large Service Exports}}\\
		\hline
		\multicolumn{1}{|p{7.5cm}}{\textbf{Name:} \texttt{sample}} & \multicolumn{1}{p{7.5cm}|}{\textbf{Interface:} \newline \texttt{sample} \newline$\hookrightarrow$\texttt{sample}}\\
		\multicolumn{2}{|l|}{}\\
		\multicolumn{2}{|p{15cm}|}{\textbf{Description:} \newline Sample.}\\
		\hline
		\hline
		\multicolumn{2}{|c|}{\textbf{\large Service Imports}}\\
		\hline
		\multicolumn{1}{|p{7.5cm}}{\textbf{Name:} \texttt{sample} \newline \textbf{Mandatory connected:} yes/no} & \multicolumn{1}{p{7.5cm}|}{\textbf{Interface:} \newline \texttt{sample::sample::sample} \newline$\hookrightarrow$\texttt{::sample}}\\
		\multicolumn{2}{|l|}{}\\
		\multicolumn{2}{|p{15cm}|}{\textbf{Description:} \newline Sample.}\\
		\hline
	\end{tabular}
\end{center}

\subsection{Power Estimators}

\newpage
\begin{center}
	\begin{tabular}{|p{7.5cm}|p{7.5cm}|}
		\hline
		\multicolumn{2}{|l|}{\textbf{\Large Service Cache/SRAM Power Estimator}}\\
		\hline
		\multicolumn{1}{|p{7.5cm}}{\textbf{Class Name:} \newline \texttt{sample::sample::sample}\newline$\hookrightarrow$\texttt{::sample}} & \multicolumn{1}{p{7.5cm}|}{\textbf{Header:} \newline \texttt{sample/sample/sample}\newline$\hookrightarrow$\texttt{/sample.hh}}\\
		\multicolumn{2}{|l|}{}\\
		\multicolumn{2}{|p{15cm}|}{\textbf{Description:} \newline Sample.}\\
		\hline
		\hline
		\multicolumn{2}{|c|}{\textbf{\large Template Parameters}}\\
		\hline
		\multicolumn{1}{|p{7.5cm}}{\textbf{Name:} \texttt{sample}} & \multicolumn{1}{p{7.5cm}|}{\textbf{Type:} \texttt{sample}}\\
		\multicolumn{2}{|p{15cm}|}{\textbf{Default value:} \texttt{sample}}\\
		\multicolumn{2}{|l|}{}\\
		\multicolumn{2}{|p{15cm}|}{\textbf{Description:} \newline Sample.}\\
		\hline
		\hline
		\multicolumn{2}{|c|}{\textbf{\large Run-Time Parameters}}\\
		\hline
		\multicolumn{1}{|p{7.5cm}}{\textbf{Name:} \texttt{sample}} & \multicolumn{1}{p{7.5cm}|}{\textbf{Type:} \texttt{sample}}\\
		\multicolumn{2}{|p{15cm}|}{\textbf{Default value:} \texttt{sample}}\\
		\multicolumn{2}{|l|}{}\\
		\multicolumn{2}{|p{15cm}|}{\textbf{Description:} \newline Sample.}\\
		\hline
		\hline
		\multicolumn{2}{|c|}{\textbf{\large Service Exports}}\\
		\hline
		\multicolumn{1}{|p{7.5cm}}{\textbf{Name:} \texttt{sample}} & \multicolumn{1}{p{7.5cm}|}{\textbf{Interface:} \newline \texttt{sample} \newline$\hookrightarrow$\texttt{sample}}\\
		\multicolumn{2}{|l|}{}\\
		\multicolumn{2}{|p{15cm}|}{\textbf{Description:} \newline Sample.}\\
		\hline
		\hline
		\multicolumn{2}{|c|}{\textbf{\large Service Imports}}\\
		\hline
		\multicolumn{1}{|p{7.5cm}}{\textbf{Name:} \texttt{sample} \newline \textbf{Mandatory connected:} yes/no} & \multicolumn{1}{p{7.5cm}|}{\textbf{Interface:} \newline \texttt{sample::sample::sample} \newline$\hookrightarrow$\texttt{::sample}}\\
		\multicolumn{2}{|l|}{}\\
		\multicolumn{2}{|p{15cm}|}{\textbf{Description:} \newline Sample.}\\
		\hline
	\end{tabular}
\end{center}

\subsection{Host hardware abstraction}

\newpage
\begin{center}
	\begin{tabular}{|p{7.5cm}|p{7.5cm}|}
		\hline
		\multicolumn{2}{|l|}{\textbf{\Large Service SDL Wrapper}}\\
		\hline
		\multicolumn{1}{|p{7.5cm}}{\textbf{Class Name:} \newline \texttt{sample::sample::sample}\newline$\hookrightarrow$\texttt{::sample}} & \multicolumn{1}{p{7.5cm}|}{\textbf{Header:} \newline \texttt{sample/sample/sample}\newline$\hookrightarrow$\texttt{/sample.hh}}\\
		\multicolumn{2}{|l|}{}\\
		\multicolumn{2}{|p{15cm}|}{\textbf{Description:} \newline Sample.}\\
		\hline
		\hline
		\multicolumn{2}{|c|}{\textbf{\large Template Parameters}}\\
		\hline
		\multicolumn{1}{|p{7.5cm}}{\textbf{Name:} \texttt{sample}} & \multicolumn{1}{p{7.5cm}|}{\textbf{Type:} \texttt{sample}}\\
		\multicolumn{2}{|p{15cm}|}{\textbf{Default value:} \texttt{sample}}\\
		\multicolumn{2}{|l|}{}\\
		\multicolumn{2}{|p{15cm}|}{\textbf{Description:} \newline Sample.}\\
		\hline
		\hline
		\multicolumn{2}{|c|}{\textbf{\large Run-Time Parameters}}\\
		\hline
		\multicolumn{1}{|p{7.5cm}}{\textbf{Name:} \texttt{sample}} & \multicolumn{1}{p{7.5cm}|}{\textbf{Type:} \texttt{sample}}\\
		\multicolumn{2}{|p{15cm}|}{\textbf{Default value:} \texttt{sample}}\\
		\multicolumn{2}{|l|}{}\\
		\multicolumn{2}{|p{15cm}|}{\textbf{Description:} \newline Sample.}\\
		\hline
		\hline
		\multicolumn{2}{|c|}{\textbf{\large Service Exports}}\\
		\hline
		\multicolumn{1}{|p{7.5cm}}{\textbf{Name:} \texttt{sample}} & \multicolumn{1}{p{7.5cm}|}{\textbf{Interface:} \newline \texttt{sample} \newline$\hookrightarrow$\texttt{sample}}\\
		\multicolumn{2}{|l|}{}\\
		\multicolumn{2}{|p{15cm}|}{\textbf{Description:} \newline Sample.}\\
		\hline
		\hline
		\multicolumn{2}{|c|}{\textbf{\large Service Imports}}\\
		\hline
		\multicolumn{1}{|p{7.5cm}}{\textbf{Name:} \texttt{sample} \newline \textbf{Mandatory connected:} yes/no} & \multicolumn{1}{p{7.5cm}|}{\textbf{Interface:} \newline \texttt{sample::sample::sample} \newline$\hookrightarrow$\texttt{::sample}}\\
		\multicolumn{2}{|l|}{}\\
		\multicolumn{2}{|p{15cm}|}{\textbf{Description:} \newline Sample.}\\
		\hline
	\end{tabular}
\end{center}

\subsection{Time}

\newpage
\begin{center}
	\begin{tabular}{|p{7.5cm}|p{7.5cm}|}
		\hline
		\multicolumn{2}{|l|}{\textbf{\Large Service Host Time}}\\
		\hline
		\multicolumn{1}{|p{7.5cm}}{\textbf{Class Name:} \newline \texttt{sample::sample::sample}\newline$\hookrightarrow$\texttt{::sample}} & \multicolumn{1}{p{7.5cm}|}{\textbf{Header:} \newline \texttt{sample/sample/sample}\newline$\hookrightarrow$\texttt{/sample.hh}}\\
		\multicolumn{2}{|l|}{}\\
		\multicolumn{2}{|p{15cm}|}{\textbf{Description:} \newline Sample.}\\
		\hline
		\hline
		\multicolumn{2}{|c|}{\textbf{\large Template Parameters}}\\
		\hline
		\multicolumn{1}{|p{7.5cm}}{\textbf{Name:} \texttt{sample}} & \multicolumn{1}{p{7.5cm}|}{\textbf{Type:} \texttt{sample}}\\
		\multicolumn{2}{|p{15cm}|}{\textbf{Default value:} \texttt{sample}}\\
		\multicolumn{2}{|l|}{}\\
		\multicolumn{2}{|p{15cm}|}{\textbf{Description:} \newline Sample.}\\
		\hline
		\hline
		\multicolumn{2}{|c|}{\textbf{\large Run-Time Parameters}}\\
		\hline
		\multicolumn{1}{|p{7.5cm}}{\textbf{Name:} \texttt{sample}} & \multicolumn{1}{p{7.5cm}|}{\textbf{Type:} \texttt{sample}}\\
		\multicolumn{2}{|p{15cm}|}{\textbf{Default value:} \texttt{sample}}\\
		\multicolumn{2}{|l|}{}\\
		\multicolumn{2}{|p{15cm}|}{\textbf{Description:} \newline Sample.}\\
		\hline
		\hline
		\multicolumn{2}{|c|}{\textbf{\large Service Exports}}\\
		\hline
		\multicolumn{1}{|p{7.5cm}}{\textbf{Name:} \texttt{sample}} & \multicolumn{1}{p{7.5cm}|}{\textbf{Interface:} \newline \texttt{sample} \newline$\hookrightarrow$\texttt{sample}}\\
		\multicolumn{2}{|l|}{}\\
		\multicolumn{2}{|p{15cm}|}{\textbf{Description:} \newline Sample.}\\
		\hline
		\hline
		\multicolumn{2}{|c|}{\textbf{\large Service Imports}}\\
		\hline
		\multicolumn{1}{|p{7.5cm}}{\textbf{Name:} \texttt{sample} \newline \textbf{Mandatory connected:} yes/no} & \multicolumn{1}{p{7.5cm}|}{\textbf{Interface:} \newline \texttt{sample::sample::sample} \newline$\hookrightarrow$\texttt{::sample}}\\
		\multicolumn{2}{|l|}{}\\
		\multicolumn{2}{|p{15cm}|}{\textbf{Description:} \newline Sample.}\\
		\hline
	\end{tabular}
\end{center}

\newpage
\begin{center}
	\begin{tabular}{|p{7.5cm}|p{7.5cm}|}
		\hline
		\multicolumn{2}{|l|}{\textbf{\Large Service SystemC Time}}\\
		\hline
		\multicolumn{1}{|p{7.5cm}}{\textbf{Class Name:} \newline \texttt{sample::sample::sample}\newline$\hookrightarrow$\texttt{::sample}} & \multicolumn{1}{p{7.5cm}|}{\textbf{Header:} \newline \texttt{sample/sample/sample}\newline$\hookrightarrow$\texttt{/sample.hh}}\\
		\multicolumn{2}{|l|}{}\\
		\multicolumn{2}{|p{15cm}|}{\textbf{Description:} \newline Sample.}\\
		\hline
		\hline
		\multicolumn{2}{|c|}{\textbf{\large Template Parameters}}\\
		\hline
		\multicolumn{1}{|p{7.5cm}}{\textbf{Name:} \texttt{sample}} & \multicolumn{1}{p{7.5cm}|}{\textbf{Type:} \texttt{sample}}\\
		\multicolumn{2}{|p{15cm}|}{\textbf{Default value:} \texttt{sample}}\\
		\multicolumn{2}{|l|}{}\\
		\multicolumn{2}{|p{15cm}|}{\textbf{Description:} \newline Sample.}\\
		\hline
		\hline
		\multicolumn{2}{|c|}{\textbf{\large Run-Time Parameters}}\\
		\hline
		\multicolumn{1}{|p{7.5cm}}{\textbf{Name:} \texttt{sample}} & \multicolumn{1}{p{7.5cm}|}{\textbf{Type:} \texttt{sample}}\\
		\multicolumn{2}{|p{15cm}|}{\textbf{Default value:} \texttt{sample}}\\
		\multicolumn{2}{|l|}{}\\
		\multicolumn{2}{|p{15cm}|}{\textbf{Description:} \newline Sample.}\\
		\hline
		\hline
		\multicolumn{2}{|c|}{\textbf{\large Service Exports}}\\
		\hline
		\multicolumn{1}{|p{7.5cm}}{\textbf{Name:} \texttt{sample}} & \multicolumn{1}{p{7.5cm}|}{\textbf{Interface:} \newline \texttt{sample} \newline$\hookrightarrow$\texttt{sample}}\\
		\multicolumn{2}{|l|}{}\\
		\multicolumn{2}{|p{15cm}|}{\textbf{Description:} \newline Sample.}\\
		\hline
		\hline
		\multicolumn{2}{|c|}{\textbf{\large Service Imports}}\\
		\hline
		\multicolumn{1}{|p{7.5cm}}{\textbf{Name:} \texttt{sample} \newline \textbf{Mandatory connected:} yes/no} & \multicolumn{1}{p{7.5cm}|}{\textbf{Interface:} \newline \texttt{sample::sample::sample} \newline$\hookrightarrow$\texttt{::sample}}\\
		\multicolumn{2}{|l|}{}\\
		\multicolumn{2}{|p{15cm}|}{\textbf{Description:} \newline Sample.}\\
		\hline
	\end{tabular}
\end{center}

\section{Building a service graph}
\label{building_a_service_graph}

Using services implies building a service graph.
For instance, consider that the client is a loader, and the service is a memory.
The programmer creates objects \texttt{loader} and \texttt{memory}, see Figure~\ref{fig:instanciation}.

\begin{figure}[h]
  \begin{center}
    \input{services/instanciation}
    \caption{\label{fig:instanciation} Client/Service instanciation.}
  \end{center}
\end{figure}

Object \texttt{loader} is a client because it needs a service (reading/writing in memory) from object \texttt{memory} to load the program.
\texttt{loader} has a member \texttt{import} named \texttt{memory\_import} whereas \texttt{memory} object has a member \texttt{export} named \texttt{memory\_export}.
The programmer connects the loader to the memory using \texttt{loader.memory\_import} and \texttt{memory.memory\_export}, see Figure~\ref{fig:connection}.

\begin{figure}[h]
  \begin{center}
    \input{services/connection}
    \caption{\label{fig:connection} Import/Export connection.}
  \end{center}
\end{figure}

Once \hfill the \hfill programmer \hfill has \hfill created \hfill a \hfill service \hfill graph, \hfill he \hfill must \hfill perform \hfill a \hfill call \hfill to \hfill \texttt{ServiceManager::Setup()}.
\texttt{ServiceManager::Setup()} returns \texttt{true} if setup of each service and client in the graph has been successful, other it returns \texttt{false}.

\section{Designing clients and services}
\label{designing_clients_and_services}

\subsection{Design a service by composing existing services}

\subsection{Designing a service}

A service is a C++ object inheriting from template class \texttt{Service<SERVICE\_INTERFACE>} \ding{202}, see Figure~\ref{fig:simple_service}.
SERVICE\_INTERFACE is a C++ abstract class defining the virtual methods implemented by the service.
To export its interface, a service must have a member of type \texttt{ServiceExport<SERVICE\_INTERFACE>} \ding{203}.
For normalization purposes, the service constructor should take only two parameters \ding{204}: the service name and the pointer to the parent (a container service).
The pointer to the parent is \texttt{null} if the service is a top level service (no parent).
The base \texttt{Object} constructor \ding{205} and the base \texttt{Service} constructor \ding{206} must be called with the name and the pointer to the parent.
\texttt{ServiceExport} member constructor must be called with the export name and a pointer to the owner, i.e. the service itself \ding{207}.

\begin{figure}[h]
  \begin{center}
    \input{services/simple_service}
    \caption{\label{fig:simple_service} Simple service.}
  \end{center}
\end{figure}


\subsection{Designing a client}

A client is a C++ object inheriting from template class \texttt{Client<SERVICE\_INTERFACE>} \ding{202}, see Figure~\ref{fig:simple_client}.
SERVICE\_INTERFACE is a C++ abstract class defining the virtual methods implemented by the service the client can call.
To import an interface, a client must have a member of type \texttt{ServiceImport<SERVICE\_INTERFACE>} \ding{203}.
For normalization purposes, the client constructor should take only two parameters \ding{204}: the client name and the pointer to the parent (a container client).
The pointer to the parent is \texttt{null} if the client is a top level client (no parent).
The base \texttt{Object} constructor \ding{205} and the base \texttt{Client} constructor \ding{206} must be called with the name and the pointer to the parent.
\texttt{ServiceImport} member constructor must be called with the import name and a pointer to the owner, i.e. the client itself \ding{207}.

\begin{figure}[h]
  \begin{center}
    \input{services/simple_client}
    \caption{\label{fig:simple_client} Simple client.}
  \end{center}
\end{figure}

\subsection{Adding run-time parameterization to a service/client}

Run-time parameterization can be added to a service or a client. "Run-time parametrization" means that the service and/or client can be reconfigured at run-time. It is opposed to "Static parametrization or template parametrization" which allows configuring a service and/or client at compilation-time.
To expose a member variable as a run-time parameter, a client/service must have member variable of type \texttt{Parameter<TYPE>}, where \texttt{TYPE} is the C++ type of the exposed member variable, see Figure~\ref{fig:run_time_parameter}. Consider that a service would expose a member variable \texttt{x} \ding{202}. An instance of class \texttt{Parameter} is defined as a member of the service \ding{203}. 
The parameter is bound to the exposed variable \ding{204} in the service/client constructor.

\begin{figure}[h]
  \begin{center}
    \input{services/run_time_parameter}
    \caption{\label{fig:run_time_parameter} Exposing a service member variable as a run-time parameter.}
  \end{center}
\end{figure}

\subsection{Handling service setup}

To force service/client setup order, method \texttt{SetupDependsOn} has been introduced. A call to this method within the client/service constructor defines a setup dependency between clients and services. Method \texttt{ServiceManager::Setup()} uses these metadata to correctly handle setup order of clients and services. \texttt{ServiceManager::Setup()} determines a topological order of client/service to find a legal setup order, thus cyclic dependencies are not supported and reported to the user when they are detected.

% \begin{figure}[h]
%   \begin{center}
%     \input{tlm/interface1}
%     \caption{\label{fig:interface1} TLM interface (simplified).}
%   \end{center}
% \end{figure}

% \subsection{Drawbacks \& Programming facilities: Pointer and event\_list}
% \label{drawbacks_and_programming_facilities}

% \begin{figure}[p]
%   \begin{center}
%     \input{tlm/ttlm_router_adv2}
%     \caption{\label{fig:ttlm_router_adv2} Example of TTLM router
%       module with requests and responses contention.}
%   \end{center}
% \end{figure}
