\mode<article>{\usepackage{fullpage}}
\mode<presentation>{\usetheme{CEA}}
% everyone:
\usepackage[english]{babel}
\usepackage{pgf}
\usepackage{graphicx}
\usepackage{pstricks}
\usepackage{pdfpages}
\usepackage{listings}
\usepackage{verbatim}
\usepackage{tikz}
\usetikzlibrary{arrows}

\mode<article> {
	\usepackage[center]{caption}
}
%%\usepackage{xmpmulti}
%%\pgfdeclareimage[height=1cm]{myimage}{filename}

\mode<article> {
	\setjobnamebeamerversion{main.beamer}
}


\title{System-C \& System-C TLM Tutorial}
\author{Gilles Mouchard \\ Reda Nouacer \\ Daniel Gracia P\'erez}
\date{2 June, 2008}

\usetheme{CEA}

\begin{document}
\mode<presentation>{
\lstset{% general command to set parameter(s)
language=c++,
basicstyle=\tiny, % print whole listing tiny
keywordstyle=\color{black}\bfseries, % bold black keywords
identifierstyle=, % nothing happens
commentstyle=\color{blue}, % white comments
stringstyle=\ttfamily, % typewriter type for strings
showstringspaces=false,
tabsize=4,
breaklines=true,
numbers=left,
framexleftmargin=7.5mm,
xleftmargin=0.13\textwidth,
xrightmargin=0.1\textwidth,
frame=tlbr}
}
\mode<article>{
\lstset{% general command to set parameter(s)
language=c++,
basicstyle=\footnotesize, % print whole listing small
keywordstyle=\color{black}\bfseries, % bold black keywords
identifierstyle=, % nothing happens
commentstyle=\color{blue}, % white comments
stringstyle=\ttfamily, % typewriter type for strings
showstringspaces=false,
tabsize=4,
breaklines=true,
numbers=left,
framexleftmargin=7.5mm,
xleftmargin=0.13\textwidth,
xrightmargin=0.1\textwidth,
frame=tlbr}
}
% \lstset{language=c++,basicstyle={\ttfamily\footnotesize},tabsize=4,breaklines=true,numbers=left,framexleftmargin=7.5mm,xleftmargin=0.13\textwidth,xrightmargin=0.1\textwidth,frame=tlbr}
\maketitle

\mode<article>{
\chapter*{Outline}
}

\mode<presentation>{
\part{Outline}
}

\mode<presentation>{
\begin{frame}
	\frametitle{Outline}
	\begin{itemize}
		\item Introduction
		\item System-C basics/CLM
		\item System-C TLM\newline
		\item {\color{blue}\ldots and your questions}
	\end{itemize}
\end{frame}
}

\mode<article>{
Welcome to the SystemC tutorial. 
We will introduce you to the exciting world of System-C and the possibilities opened to you with System-C Transaction Level Modelling (aka TLM).
We start with a presentation on what is System-C and what you can use it for.
Then the basics of System-C programming will be introduced, so you can start coding System-C cycle level models.
The third part introduces System-C TLM programming model, and more concretely TLM 2.0, which is deemed to become an IEEE standard like its big brother SystemC.
%Once the basics have been introduced, we will give an overview of the different existent frameworks that use System-C.
%A special section will be reserved for the UNISIM simulation framework.
%More on them at the end of the tutorial.

% \begin{figure}[!h]
% 	\begin{center}
% 		\includegraphics[page=1,height=6cm]{main_beamer.pdf}
% 	\end{center}
% 	\caption{Outline slide.}
% 	\label{slide:outline}
% \end{figure}
}

\newpage

\mode<article>{
\tableofcontents
\listoffigures
\listoftables
}

\input{introduction/intro_main}

\newpage

\mode<article>{\usepackage{fullpage}}
\mode<presentation>{\usetheme{CEA}}
% everyone:
\usepackage[english]{babel}
\usepackage{pgf}
\usepackage{graphicx}
\usepackage{pstricks}
\usepackage{pdfpages}
\usepackage{listings}
\usepackage{verbatim}
\usepackage{tikz}
\usetikzlibrary{arrows}

\mode<article> {
	\usepackage[center]{caption}
}
%%\usepackage{xmpmulti}
%%\pgfdeclareimage[height=1cm]{myimage}{filename}

\mode<article> {
	\setjobnamebeamerversion{main.beamer}
}


\title{System-C \& System-C TLM Tutorial}
\author{Gilles Mouchard \\ Reda Nouacer \\ Daniel Gracia P\'erez}
\date{2 June, 2008}

\usetheme{CEA}

\begin{document}
\mode<presentation>{
\lstset{% general command to set parameter(s)
language=c++,
basicstyle=\tiny, % print whole listing tiny
keywordstyle=\color{black}\bfseries, % bold black keywords
identifierstyle=, % nothing happens
commentstyle=\color{blue}, % white comments
stringstyle=\ttfamily, % typewriter type for strings
showstringspaces=false,
tabsize=4,
breaklines=true,
numbers=left,
framexleftmargin=7.5mm,
xleftmargin=0.13\textwidth,
xrightmargin=0.1\textwidth,
frame=tlbr}
}
\mode<article>{
\lstset{% general command to set parameter(s)
language=c++,
basicstyle=\footnotesize, % print whole listing small
keywordstyle=\color{black}\bfseries, % bold black keywords
identifierstyle=, % nothing happens
commentstyle=\color{blue}, % white comments
stringstyle=\ttfamily, % typewriter type for strings
showstringspaces=false,
tabsize=4,
breaklines=true,
numbers=left,
framexleftmargin=7.5mm,
xleftmargin=0.13\textwidth,
xrightmargin=0.1\textwidth,
frame=tlbr}
}
% \lstset{language=c++,basicstyle={\ttfamily\footnotesize},tabsize=4,breaklines=true,numbers=left,framexleftmargin=7.5mm,xleftmargin=0.13\textwidth,xrightmargin=0.1\textwidth,frame=tlbr}
\maketitle

\mode<article>{
\chapter*{Outline}
}

\mode<presentation>{
\part{Outline}
}

\mode<presentation>{
\begin{frame}
	\frametitle{Outline}
	\begin{itemize}
		\item Introduction
		\item System-C basics/CLM
		\item System-C TLM\newline
		\item {\color{blue}\ldots and your questions}
	\end{itemize}
\end{frame}
}

\mode<article>{
Welcome to the SystemC tutorial. 
We will introduce you to the exciting world of System-C and the possibilities opened to you with System-C Transaction Level Modelling (aka TLM).
We start with a presentation on what is System-C and what you can use it for.
Then the basics of System-C programming will be introduced, so you can start coding System-C cycle level models.
The third part introduces System-C TLM programming model, and more concretely TLM 2.0, which is deemed to become an IEEE standard like its big brother SystemC.
%Once the basics have been introduced, we will give an overview of the different existent frameworks that use System-C.
%A special section will be reserved for the UNISIM simulation framework.
%More on them at the end of the tutorial.

% \begin{figure}[!h]
% 	\begin{center}
% 		\includegraphics[page=1,height=6cm]{main_beamer.pdf}
% 	\end{center}
% 	\caption{Outline slide.}
% 	\label{slide:outline}
% \end{figure}
}

\newpage

\mode<article>{
\tableofcontents
\listoffigures
\listoftables
}

\input{introduction/intro_main}

\newpage

\mode<article>{\usepackage{fullpage}}
\mode<presentation>{\usetheme{CEA}}
% everyone:
\usepackage[english]{babel}
\usepackage{pgf}
\usepackage{graphicx}
\usepackage{pstricks}
\usepackage{pdfpages}
\usepackage{listings}
\usepackage{verbatim}
\usepackage{tikz}
\usetikzlibrary{arrows}

\mode<article> {
	\usepackage[center]{caption}
}
%%\usepackage{xmpmulti}
%%\pgfdeclareimage[height=1cm]{myimage}{filename}

\mode<article> {
	\setjobnamebeamerversion{main.beamer}
}


\title{System-C \& System-C TLM Tutorial}
\author{Gilles Mouchard \\ Reda Nouacer \\ Daniel Gracia P\'erez}
\date{2 June, 2008}

\usetheme{CEA}

\begin{document}
\mode<presentation>{
\lstset{% general command to set parameter(s)
language=c++,
basicstyle=\tiny, % print whole listing tiny
keywordstyle=\color{black}\bfseries, % bold black keywords
identifierstyle=, % nothing happens
commentstyle=\color{blue}, % white comments
stringstyle=\ttfamily, % typewriter type for strings
showstringspaces=false,
tabsize=4,
breaklines=true,
numbers=left,
framexleftmargin=7.5mm,
xleftmargin=0.13\textwidth,
xrightmargin=0.1\textwidth,
frame=tlbr}
}
\mode<article>{
\lstset{% general command to set parameter(s)
language=c++,
basicstyle=\footnotesize, % print whole listing small
keywordstyle=\color{black}\bfseries, % bold black keywords
identifierstyle=, % nothing happens
commentstyle=\color{blue}, % white comments
stringstyle=\ttfamily, % typewriter type for strings
showstringspaces=false,
tabsize=4,
breaklines=true,
numbers=left,
framexleftmargin=7.5mm,
xleftmargin=0.13\textwidth,
xrightmargin=0.1\textwidth,
frame=tlbr}
}
% \lstset{language=c++,basicstyle={\ttfamily\footnotesize},tabsize=4,breaklines=true,numbers=left,framexleftmargin=7.5mm,xleftmargin=0.13\textwidth,xrightmargin=0.1\textwidth,frame=tlbr}
\maketitle

\mode<article>{
\chapter*{Outline}
}

\mode<presentation>{
\part{Outline}
}

\mode<presentation>{
\begin{frame}
	\frametitle{Outline}
	\begin{itemize}
		\item Introduction
		\item System-C basics/CLM
		\item System-C TLM\newline
		\item {\color{blue}\ldots and your questions}
	\end{itemize}
\end{frame}
}

\mode<article>{
Welcome to the SystemC tutorial. 
We will introduce you to the exciting world of System-C and the possibilities opened to you with System-C Transaction Level Modelling (aka TLM).
We start with a presentation on what is System-C and what you can use it for.
Then the basics of System-C programming will be introduced, so you can start coding System-C cycle level models.
The third part introduces System-C TLM programming model, and more concretely TLM 2.0, which is deemed to become an IEEE standard like its big brother SystemC.
%Once the basics have been introduced, we will give an overview of the different existent frameworks that use System-C.
%A special section will be reserved for the UNISIM simulation framework.
%More on them at the end of the tutorial.

% \begin{figure}[!h]
% 	\begin{center}
% 		\includegraphics[page=1,height=6cm]{main_beamer.pdf}
% 	\end{center}
% 	\caption{Outline slide.}
% 	\label{slide:outline}
% \end{figure}
}

\newpage

\mode<article>{
\tableofcontents
\listoffigures
\listoftables
}

\input{introduction/intro_main}

\newpage

\mode<article>{\usepackage{fullpage}}
\mode<presentation>{\usetheme{CEA}}
% everyone:
\usepackage[english]{babel}
\usepackage{pgf}
\usepackage{graphicx}
\usepackage{pstricks}
\usepackage{pdfpages}
\usepackage{listings}
\usepackage{verbatim}
\usepackage{tikz}
\usetikzlibrary{arrows}

\mode<article> {
	\usepackage[center]{caption}
}
%%\usepackage{xmpmulti}
%%\pgfdeclareimage[height=1cm]{myimage}{filename}

\mode<article> {
	\setjobnamebeamerversion{main.beamer}
}


\title{System-C \& System-C TLM Tutorial}
\author{Gilles Mouchard \\ Reda Nouacer \\ Daniel Gracia P\'erez}
\date{2 June, 2008}

\usetheme{CEA}

\begin{document}
\mode<presentation>{
\lstset{% general command to set parameter(s)
language=c++,
basicstyle=\tiny, % print whole listing tiny
keywordstyle=\color{black}\bfseries, % bold black keywords
identifierstyle=, % nothing happens
commentstyle=\color{blue}, % white comments
stringstyle=\ttfamily, % typewriter type for strings
showstringspaces=false,
tabsize=4,
breaklines=true,
numbers=left,
framexleftmargin=7.5mm,
xleftmargin=0.13\textwidth,
xrightmargin=0.1\textwidth,
frame=tlbr}
}
\mode<article>{
\lstset{% general command to set parameter(s)
language=c++,
basicstyle=\footnotesize, % print whole listing small
keywordstyle=\color{black}\bfseries, % bold black keywords
identifierstyle=, % nothing happens
commentstyle=\color{blue}, % white comments
stringstyle=\ttfamily, % typewriter type for strings
showstringspaces=false,
tabsize=4,
breaklines=true,
numbers=left,
framexleftmargin=7.5mm,
xleftmargin=0.13\textwidth,
xrightmargin=0.1\textwidth,
frame=tlbr}
}
% \lstset{language=c++,basicstyle={\ttfamily\footnotesize},tabsize=4,breaklines=true,numbers=left,framexleftmargin=7.5mm,xleftmargin=0.13\textwidth,xrightmargin=0.1\textwidth,frame=tlbr}
\maketitle

\mode<article>{
\chapter*{Outline}
}

\mode<presentation>{
\part{Outline}
}

\mode<presentation>{
\begin{frame}
	\frametitle{Outline}
	\begin{itemize}
		\item Introduction
		\item System-C basics/CLM
		\item System-C TLM\newline
		\item {\color{blue}\ldots and your questions}
	\end{itemize}
\end{frame}
}

\mode<article>{
Welcome to the SystemC tutorial. 
We will introduce you to the exciting world of System-C and the possibilities opened to you with System-C Transaction Level Modelling (aka TLM).
We start with a presentation on what is System-C and what you can use it for.
Then the basics of System-C programming will be introduced, so you can start coding System-C cycle level models.
The third part introduces System-C TLM programming model, and more concretely TLM 2.0, which is deemed to become an IEEE standard like its big brother SystemC.
%Once the basics have been introduced, we will give an overview of the different existent frameworks that use System-C.
%A special section will be reserved for the UNISIM simulation framework.
%More on them at the end of the tutorial.

% \begin{figure}[!h]
% 	\begin{center}
% 		\includegraphics[page=1,height=6cm]{main_beamer.pdf}
% 	\end{center}
% 	\caption{Outline slide.}
% 	\label{slide:outline}
% \end{figure}
}

\newpage

\mode<article>{
\tableofcontents
\listoffigures
\listoftables
}

\input{introduction/intro_main}

\newpage

\input{systemc_cycle/main}

\newpage

\input{tlm/main}

% \newpage

% \input{systemc_basics/basics_main}

% \newpage

% \input{tlm/tlm_main}

% \newpage

% \input{services/services_main}

\end{document}


\newpage

\mode<article>{\usepackage{fullpage}}
\mode<presentation>{\usetheme{CEA}}
% everyone:
\usepackage[english]{babel}
\usepackage{pgf}
\usepackage{graphicx}
\usepackage{pstricks}
\usepackage{pdfpages}
\usepackage{listings}
\usepackage{verbatim}
\usepackage{tikz}
\usetikzlibrary{arrows}

\mode<article> {
	\usepackage[center]{caption}
}
%%\usepackage{xmpmulti}
%%\pgfdeclareimage[height=1cm]{myimage}{filename}

\mode<article> {
	\setjobnamebeamerversion{main.beamer}
}


\title{System-C \& System-C TLM Tutorial}
\author{Gilles Mouchard \\ Reda Nouacer \\ Daniel Gracia P\'erez}
\date{2 June, 2008}

\usetheme{CEA}

\begin{document}
\mode<presentation>{
\lstset{% general command to set parameter(s)
language=c++,
basicstyle=\tiny, % print whole listing tiny
keywordstyle=\color{black}\bfseries, % bold black keywords
identifierstyle=, % nothing happens
commentstyle=\color{blue}, % white comments
stringstyle=\ttfamily, % typewriter type for strings
showstringspaces=false,
tabsize=4,
breaklines=true,
numbers=left,
framexleftmargin=7.5mm,
xleftmargin=0.13\textwidth,
xrightmargin=0.1\textwidth,
frame=tlbr}
}
\mode<article>{
\lstset{% general command to set parameter(s)
language=c++,
basicstyle=\footnotesize, % print whole listing small
keywordstyle=\color{black}\bfseries, % bold black keywords
identifierstyle=, % nothing happens
commentstyle=\color{blue}, % white comments
stringstyle=\ttfamily, % typewriter type for strings
showstringspaces=false,
tabsize=4,
breaklines=true,
numbers=left,
framexleftmargin=7.5mm,
xleftmargin=0.13\textwidth,
xrightmargin=0.1\textwidth,
frame=tlbr}
}
% \lstset{language=c++,basicstyle={\ttfamily\footnotesize},tabsize=4,breaklines=true,numbers=left,framexleftmargin=7.5mm,xleftmargin=0.13\textwidth,xrightmargin=0.1\textwidth,frame=tlbr}
\maketitle

\mode<article>{
\chapter*{Outline}
}

\mode<presentation>{
\part{Outline}
}

\mode<presentation>{
\begin{frame}
	\frametitle{Outline}
	\begin{itemize}
		\item Introduction
		\item System-C basics/CLM
		\item System-C TLM\newline
		\item {\color{blue}\ldots and your questions}
	\end{itemize}
\end{frame}
}

\mode<article>{
Welcome to the SystemC tutorial. 
We will introduce you to the exciting world of System-C and the possibilities opened to you with System-C Transaction Level Modelling (aka TLM).
We start with a presentation on what is System-C and what you can use it for.
Then the basics of System-C programming will be introduced, so you can start coding System-C cycle level models.
The third part introduces System-C TLM programming model, and more concretely TLM 2.0, which is deemed to become an IEEE standard like its big brother SystemC.
%Once the basics have been introduced, we will give an overview of the different existent frameworks that use System-C.
%A special section will be reserved for the UNISIM simulation framework.
%More on them at the end of the tutorial.

% \begin{figure}[!h]
% 	\begin{center}
% 		\includegraphics[page=1,height=6cm]{main_beamer.pdf}
% 	\end{center}
% 	\caption{Outline slide.}
% 	\label{slide:outline}
% \end{figure}
}

\newpage

\mode<article>{
\tableofcontents
\listoffigures
\listoftables
}

\input{introduction/intro_main}

\newpage

\input{systemc_cycle/main}

\newpage

\input{tlm/main}

% \newpage

% \input{systemc_basics/basics_main}

% \newpage

% \input{tlm/tlm_main}

% \newpage

% \input{services/services_main}

\end{document}


% \newpage

% \input{systemc_basics/basics_main}

% \newpage

% \input{tlm/tlm_main}

% \newpage

% \input{services/services_main}

\end{document}


\newpage

\mode<article>{\usepackage{fullpage}}
\mode<presentation>{\usetheme{CEA}}
% everyone:
\usepackage[english]{babel}
\usepackage{pgf}
\usepackage{graphicx}
\usepackage{pstricks}
\usepackage{pdfpages}
\usepackage{listings}
\usepackage{verbatim}
\usepackage{tikz}
\usetikzlibrary{arrows}

\mode<article> {
	\usepackage[center]{caption}
}
%%\usepackage{xmpmulti}
%%\pgfdeclareimage[height=1cm]{myimage}{filename}

\mode<article> {
	\setjobnamebeamerversion{main.beamer}
}


\title{System-C \& System-C TLM Tutorial}
\author{Gilles Mouchard \\ Reda Nouacer \\ Daniel Gracia P\'erez}
\date{2 June, 2008}

\usetheme{CEA}

\begin{document}
\mode<presentation>{
\lstset{% general command to set parameter(s)
language=c++,
basicstyle=\tiny, % print whole listing tiny
keywordstyle=\color{black}\bfseries, % bold black keywords
identifierstyle=, % nothing happens
commentstyle=\color{blue}, % white comments
stringstyle=\ttfamily, % typewriter type for strings
showstringspaces=false,
tabsize=4,
breaklines=true,
numbers=left,
framexleftmargin=7.5mm,
xleftmargin=0.13\textwidth,
xrightmargin=0.1\textwidth,
frame=tlbr}
}
\mode<article>{
\lstset{% general command to set parameter(s)
language=c++,
basicstyle=\footnotesize, % print whole listing small
keywordstyle=\color{black}\bfseries, % bold black keywords
identifierstyle=, % nothing happens
commentstyle=\color{blue}, % white comments
stringstyle=\ttfamily, % typewriter type for strings
showstringspaces=false,
tabsize=4,
breaklines=true,
numbers=left,
framexleftmargin=7.5mm,
xleftmargin=0.13\textwidth,
xrightmargin=0.1\textwidth,
frame=tlbr}
}
% \lstset{language=c++,basicstyle={\ttfamily\footnotesize},tabsize=4,breaklines=true,numbers=left,framexleftmargin=7.5mm,xleftmargin=0.13\textwidth,xrightmargin=0.1\textwidth,frame=tlbr}
\maketitle

\mode<article>{
\chapter*{Outline}
}

\mode<presentation>{
\part{Outline}
}

\mode<presentation>{
\begin{frame}
	\frametitle{Outline}
	\begin{itemize}
		\item Introduction
		\item System-C basics/CLM
		\item System-C TLM\newline
		\item {\color{blue}\ldots and your questions}
	\end{itemize}
\end{frame}
}

\mode<article>{
Welcome to the SystemC tutorial. 
We will introduce you to the exciting world of System-C and the possibilities opened to you with System-C Transaction Level Modelling (aka TLM).
We start with a presentation on what is System-C and what you can use it for.
Then the basics of System-C programming will be introduced, so you can start coding System-C cycle level models.
The third part introduces System-C TLM programming model, and more concretely TLM 2.0, which is deemed to become an IEEE standard like its big brother SystemC.
%Once the basics have been introduced, we will give an overview of the different existent frameworks that use System-C.
%A special section will be reserved for the UNISIM simulation framework.
%More on them at the end of the tutorial.

% \begin{figure}[!h]
% 	\begin{center}
% 		\includegraphics[page=1,height=6cm]{main_beamer.pdf}
% 	\end{center}
% 	\caption{Outline slide.}
% 	\label{slide:outline}
% \end{figure}
}

\newpage

\mode<article>{
\tableofcontents
\listoffigures
\listoftables
}

\input{introduction/intro_main}

\newpage

\mode<article>{\usepackage{fullpage}}
\mode<presentation>{\usetheme{CEA}}
% everyone:
\usepackage[english]{babel}
\usepackage{pgf}
\usepackage{graphicx}
\usepackage{pstricks}
\usepackage{pdfpages}
\usepackage{listings}
\usepackage{verbatim}
\usepackage{tikz}
\usetikzlibrary{arrows}

\mode<article> {
	\usepackage[center]{caption}
}
%%\usepackage{xmpmulti}
%%\pgfdeclareimage[height=1cm]{myimage}{filename}

\mode<article> {
	\setjobnamebeamerversion{main.beamer}
}


\title{System-C \& System-C TLM Tutorial}
\author{Gilles Mouchard \\ Reda Nouacer \\ Daniel Gracia P\'erez}
\date{2 June, 2008}

\usetheme{CEA}

\begin{document}
\mode<presentation>{
\lstset{% general command to set parameter(s)
language=c++,
basicstyle=\tiny, % print whole listing tiny
keywordstyle=\color{black}\bfseries, % bold black keywords
identifierstyle=, % nothing happens
commentstyle=\color{blue}, % white comments
stringstyle=\ttfamily, % typewriter type for strings
showstringspaces=false,
tabsize=4,
breaklines=true,
numbers=left,
framexleftmargin=7.5mm,
xleftmargin=0.13\textwidth,
xrightmargin=0.1\textwidth,
frame=tlbr}
}
\mode<article>{
\lstset{% general command to set parameter(s)
language=c++,
basicstyle=\footnotesize, % print whole listing small
keywordstyle=\color{black}\bfseries, % bold black keywords
identifierstyle=, % nothing happens
commentstyle=\color{blue}, % white comments
stringstyle=\ttfamily, % typewriter type for strings
showstringspaces=false,
tabsize=4,
breaklines=true,
numbers=left,
framexleftmargin=7.5mm,
xleftmargin=0.13\textwidth,
xrightmargin=0.1\textwidth,
frame=tlbr}
}
% \lstset{language=c++,basicstyle={\ttfamily\footnotesize},tabsize=4,breaklines=true,numbers=left,framexleftmargin=7.5mm,xleftmargin=0.13\textwidth,xrightmargin=0.1\textwidth,frame=tlbr}
\maketitle

\mode<article>{
\chapter*{Outline}
}

\mode<presentation>{
\part{Outline}
}

\mode<presentation>{
\begin{frame}
	\frametitle{Outline}
	\begin{itemize}
		\item Introduction
		\item System-C basics/CLM
		\item System-C TLM\newline
		\item {\color{blue}\ldots and your questions}
	\end{itemize}
\end{frame}
}

\mode<article>{
Welcome to the SystemC tutorial. 
We will introduce you to the exciting world of System-C and the possibilities opened to you with System-C Transaction Level Modelling (aka TLM).
We start with a presentation on what is System-C and what you can use it for.
Then the basics of System-C programming will be introduced, so you can start coding System-C cycle level models.
The third part introduces System-C TLM programming model, and more concretely TLM 2.0, which is deemed to become an IEEE standard like its big brother SystemC.
%Once the basics have been introduced, we will give an overview of the different existent frameworks that use System-C.
%A special section will be reserved for the UNISIM simulation framework.
%More on them at the end of the tutorial.

% \begin{figure}[!h]
% 	\begin{center}
% 		\includegraphics[page=1,height=6cm]{main_beamer.pdf}
% 	\end{center}
% 	\caption{Outline slide.}
% 	\label{slide:outline}
% \end{figure}
}

\newpage

\mode<article>{
\tableofcontents
\listoffigures
\listoftables
}

\input{introduction/intro_main}

\newpage

\input{systemc_cycle/main}

\newpage

\input{tlm/main}

% \newpage

% \input{systemc_basics/basics_main}

% \newpage

% \input{tlm/tlm_main}

% \newpage

% \input{services/services_main}

\end{document}


\newpage

\mode<article>{\usepackage{fullpage}}
\mode<presentation>{\usetheme{CEA}}
% everyone:
\usepackage[english]{babel}
\usepackage{pgf}
\usepackage{graphicx}
\usepackage{pstricks}
\usepackage{pdfpages}
\usepackage{listings}
\usepackage{verbatim}
\usepackage{tikz}
\usetikzlibrary{arrows}

\mode<article> {
	\usepackage[center]{caption}
}
%%\usepackage{xmpmulti}
%%\pgfdeclareimage[height=1cm]{myimage}{filename}

\mode<article> {
	\setjobnamebeamerversion{main.beamer}
}


\title{System-C \& System-C TLM Tutorial}
\author{Gilles Mouchard \\ Reda Nouacer \\ Daniel Gracia P\'erez}
\date{2 June, 2008}

\usetheme{CEA}

\begin{document}
\mode<presentation>{
\lstset{% general command to set parameter(s)
language=c++,
basicstyle=\tiny, % print whole listing tiny
keywordstyle=\color{black}\bfseries, % bold black keywords
identifierstyle=, % nothing happens
commentstyle=\color{blue}, % white comments
stringstyle=\ttfamily, % typewriter type for strings
showstringspaces=false,
tabsize=4,
breaklines=true,
numbers=left,
framexleftmargin=7.5mm,
xleftmargin=0.13\textwidth,
xrightmargin=0.1\textwidth,
frame=tlbr}
}
\mode<article>{
\lstset{% general command to set parameter(s)
language=c++,
basicstyle=\footnotesize, % print whole listing small
keywordstyle=\color{black}\bfseries, % bold black keywords
identifierstyle=, % nothing happens
commentstyle=\color{blue}, % white comments
stringstyle=\ttfamily, % typewriter type for strings
showstringspaces=false,
tabsize=4,
breaklines=true,
numbers=left,
framexleftmargin=7.5mm,
xleftmargin=0.13\textwidth,
xrightmargin=0.1\textwidth,
frame=tlbr}
}
% \lstset{language=c++,basicstyle={\ttfamily\footnotesize},tabsize=4,breaklines=true,numbers=left,framexleftmargin=7.5mm,xleftmargin=0.13\textwidth,xrightmargin=0.1\textwidth,frame=tlbr}
\maketitle

\mode<article>{
\chapter*{Outline}
}

\mode<presentation>{
\part{Outline}
}

\mode<presentation>{
\begin{frame}
	\frametitle{Outline}
	\begin{itemize}
		\item Introduction
		\item System-C basics/CLM
		\item System-C TLM\newline
		\item {\color{blue}\ldots and your questions}
	\end{itemize}
\end{frame}
}

\mode<article>{
Welcome to the SystemC tutorial. 
We will introduce you to the exciting world of System-C and the possibilities opened to you with System-C Transaction Level Modelling (aka TLM).
We start with a presentation on what is System-C and what you can use it for.
Then the basics of System-C programming will be introduced, so you can start coding System-C cycle level models.
The third part introduces System-C TLM programming model, and more concretely TLM 2.0, which is deemed to become an IEEE standard like its big brother SystemC.
%Once the basics have been introduced, we will give an overview of the different existent frameworks that use System-C.
%A special section will be reserved for the UNISIM simulation framework.
%More on them at the end of the tutorial.

% \begin{figure}[!h]
% 	\begin{center}
% 		\includegraphics[page=1,height=6cm]{main_beamer.pdf}
% 	\end{center}
% 	\caption{Outline slide.}
% 	\label{slide:outline}
% \end{figure}
}

\newpage

\mode<article>{
\tableofcontents
\listoffigures
\listoftables
}

\input{introduction/intro_main}

\newpage

\input{systemc_cycle/main}

\newpage

\input{tlm/main}

% \newpage

% \input{systemc_basics/basics_main}

% \newpage

% \input{tlm/tlm_main}

% \newpage

% \input{services/services_main}

\end{document}


% \newpage

% \input{systemc_basics/basics_main}

% \newpage

% \input{tlm/tlm_main}

% \newpage

% \input{services/services_main}

\end{document}


% \newpage

% \input{systemc_basics/basics_main}

% \newpage

% \input{tlm/tlm_main}

% \newpage

% \input{services/services_main}

\end{document}


\newpage

\mode<article>{\usepackage{fullpage}}
\mode<presentation>{\usetheme{CEA}}
% everyone:
\usepackage[english]{babel}
\usepackage{pgf}
\usepackage{graphicx}
\usepackage{pstricks}
\usepackage{pdfpages}
\usepackage{listings}
\usepackage{verbatim}
\usepackage{tikz}
\usetikzlibrary{arrows}

\mode<article> {
	\usepackage[center]{caption}
}
%%\usepackage{xmpmulti}
%%\pgfdeclareimage[height=1cm]{myimage}{filename}

\mode<article> {
	\setjobnamebeamerversion{main.beamer}
}


\title{System-C \& System-C TLM Tutorial}
\author{Gilles Mouchard \\ Reda Nouacer \\ Daniel Gracia P\'erez}
\date{2 June, 2008}

\usetheme{CEA}

\begin{document}
\mode<presentation>{
\lstset{% general command to set parameter(s)
language=c++,
basicstyle=\tiny, % print whole listing tiny
keywordstyle=\color{black}\bfseries, % bold black keywords
identifierstyle=, % nothing happens
commentstyle=\color{blue}, % white comments
stringstyle=\ttfamily, % typewriter type for strings
showstringspaces=false,
tabsize=4,
breaklines=true,
numbers=left,
framexleftmargin=7.5mm,
xleftmargin=0.13\textwidth,
xrightmargin=0.1\textwidth,
frame=tlbr}
}
\mode<article>{
\lstset{% general command to set parameter(s)
language=c++,
basicstyle=\footnotesize, % print whole listing small
keywordstyle=\color{black}\bfseries, % bold black keywords
identifierstyle=, % nothing happens
commentstyle=\color{blue}, % white comments
stringstyle=\ttfamily, % typewriter type for strings
showstringspaces=false,
tabsize=4,
breaklines=true,
numbers=left,
framexleftmargin=7.5mm,
xleftmargin=0.13\textwidth,
xrightmargin=0.1\textwidth,
frame=tlbr}
}
% \lstset{language=c++,basicstyle={\ttfamily\footnotesize},tabsize=4,breaklines=true,numbers=left,framexleftmargin=7.5mm,xleftmargin=0.13\textwidth,xrightmargin=0.1\textwidth,frame=tlbr}
\maketitle

\mode<article>{
\chapter*{Outline}
}

\mode<presentation>{
\part{Outline}
}

\mode<presentation>{
\begin{frame}
	\frametitle{Outline}
	\begin{itemize}
		\item Introduction
		\item System-C basics/CLM
		\item System-C TLM\newline
		\item {\color{blue}\ldots and your questions}
	\end{itemize}
\end{frame}
}

\mode<article>{
Welcome to the SystemC tutorial. 
We will introduce you to the exciting world of System-C and the possibilities opened to you with System-C Transaction Level Modelling (aka TLM).
We start with a presentation on what is System-C and what you can use it for.
Then the basics of System-C programming will be introduced, so you can start coding System-C cycle level models.
The third part introduces System-C TLM programming model, and more concretely TLM 2.0, which is deemed to become an IEEE standard like its big brother SystemC.
%Once the basics have been introduced, we will give an overview of the different existent frameworks that use System-C.
%A special section will be reserved for the UNISIM simulation framework.
%More on them at the end of the tutorial.

% \begin{figure}[!h]
% 	\begin{center}
% 		\includegraphics[page=1,height=6cm]{main_beamer.pdf}
% 	\end{center}
% 	\caption{Outline slide.}
% 	\label{slide:outline}
% \end{figure}
}

\newpage

\mode<article>{
\tableofcontents
\listoffigures
\listoftables
}

\input{introduction/intro_main}

\newpage

\mode<article>{\usepackage{fullpage}}
\mode<presentation>{\usetheme{CEA}}
% everyone:
\usepackage[english]{babel}
\usepackage{pgf}
\usepackage{graphicx}
\usepackage{pstricks}
\usepackage{pdfpages}
\usepackage{listings}
\usepackage{verbatim}
\usepackage{tikz}
\usetikzlibrary{arrows}

\mode<article> {
	\usepackage[center]{caption}
}
%%\usepackage{xmpmulti}
%%\pgfdeclareimage[height=1cm]{myimage}{filename}

\mode<article> {
	\setjobnamebeamerversion{main.beamer}
}


\title{System-C \& System-C TLM Tutorial}
\author{Gilles Mouchard \\ Reda Nouacer \\ Daniel Gracia P\'erez}
\date{2 June, 2008}

\usetheme{CEA}

\begin{document}
\mode<presentation>{
\lstset{% general command to set parameter(s)
language=c++,
basicstyle=\tiny, % print whole listing tiny
keywordstyle=\color{black}\bfseries, % bold black keywords
identifierstyle=, % nothing happens
commentstyle=\color{blue}, % white comments
stringstyle=\ttfamily, % typewriter type for strings
showstringspaces=false,
tabsize=4,
breaklines=true,
numbers=left,
framexleftmargin=7.5mm,
xleftmargin=0.13\textwidth,
xrightmargin=0.1\textwidth,
frame=tlbr}
}
\mode<article>{
\lstset{% general command to set parameter(s)
language=c++,
basicstyle=\footnotesize, % print whole listing small
keywordstyle=\color{black}\bfseries, % bold black keywords
identifierstyle=, % nothing happens
commentstyle=\color{blue}, % white comments
stringstyle=\ttfamily, % typewriter type for strings
showstringspaces=false,
tabsize=4,
breaklines=true,
numbers=left,
framexleftmargin=7.5mm,
xleftmargin=0.13\textwidth,
xrightmargin=0.1\textwidth,
frame=tlbr}
}
% \lstset{language=c++,basicstyle={\ttfamily\footnotesize},tabsize=4,breaklines=true,numbers=left,framexleftmargin=7.5mm,xleftmargin=0.13\textwidth,xrightmargin=0.1\textwidth,frame=tlbr}
\maketitle

\mode<article>{
\chapter*{Outline}
}

\mode<presentation>{
\part{Outline}
}

\mode<presentation>{
\begin{frame}
	\frametitle{Outline}
	\begin{itemize}
		\item Introduction
		\item System-C basics/CLM
		\item System-C TLM\newline
		\item {\color{blue}\ldots and your questions}
	\end{itemize}
\end{frame}
}

\mode<article>{
Welcome to the SystemC tutorial. 
We will introduce you to the exciting world of System-C and the possibilities opened to you with System-C Transaction Level Modelling (aka TLM).
We start with a presentation on what is System-C and what you can use it for.
Then the basics of System-C programming will be introduced, so you can start coding System-C cycle level models.
The third part introduces System-C TLM programming model, and more concretely TLM 2.0, which is deemed to become an IEEE standard like its big brother SystemC.
%Once the basics have been introduced, we will give an overview of the different existent frameworks that use System-C.
%A special section will be reserved for the UNISIM simulation framework.
%More on them at the end of the tutorial.

% \begin{figure}[!h]
% 	\begin{center}
% 		\includegraphics[page=1,height=6cm]{main_beamer.pdf}
% 	\end{center}
% 	\caption{Outline slide.}
% 	\label{slide:outline}
% \end{figure}
}

\newpage

\mode<article>{
\tableofcontents
\listoffigures
\listoftables
}

\input{introduction/intro_main}

\newpage

\mode<article>{\usepackage{fullpage}}
\mode<presentation>{\usetheme{CEA}}
% everyone:
\usepackage[english]{babel}
\usepackage{pgf}
\usepackage{graphicx}
\usepackage{pstricks}
\usepackage{pdfpages}
\usepackage{listings}
\usepackage{verbatim}
\usepackage{tikz}
\usetikzlibrary{arrows}

\mode<article> {
	\usepackage[center]{caption}
}
%%\usepackage{xmpmulti}
%%\pgfdeclareimage[height=1cm]{myimage}{filename}

\mode<article> {
	\setjobnamebeamerversion{main.beamer}
}


\title{System-C \& System-C TLM Tutorial}
\author{Gilles Mouchard \\ Reda Nouacer \\ Daniel Gracia P\'erez}
\date{2 June, 2008}

\usetheme{CEA}

\begin{document}
\mode<presentation>{
\lstset{% general command to set parameter(s)
language=c++,
basicstyle=\tiny, % print whole listing tiny
keywordstyle=\color{black}\bfseries, % bold black keywords
identifierstyle=, % nothing happens
commentstyle=\color{blue}, % white comments
stringstyle=\ttfamily, % typewriter type for strings
showstringspaces=false,
tabsize=4,
breaklines=true,
numbers=left,
framexleftmargin=7.5mm,
xleftmargin=0.13\textwidth,
xrightmargin=0.1\textwidth,
frame=tlbr}
}
\mode<article>{
\lstset{% general command to set parameter(s)
language=c++,
basicstyle=\footnotesize, % print whole listing small
keywordstyle=\color{black}\bfseries, % bold black keywords
identifierstyle=, % nothing happens
commentstyle=\color{blue}, % white comments
stringstyle=\ttfamily, % typewriter type for strings
showstringspaces=false,
tabsize=4,
breaklines=true,
numbers=left,
framexleftmargin=7.5mm,
xleftmargin=0.13\textwidth,
xrightmargin=0.1\textwidth,
frame=tlbr}
}
% \lstset{language=c++,basicstyle={\ttfamily\footnotesize},tabsize=4,breaklines=true,numbers=left,framexleftmargin=7.5mm,xleftmargin=0.13\textwidth,xrightmargin=0.1\textwidth,frame=tlbr}
\maketitle

\mode<article>{
\chapter*{Outline}
}

\mode<presentation>{
\part{Outline}
}

\mode<presentation>{
\begin{frame}
	\frametitle{Outline}
	\begin{itemize}
		\item Introduction
		\item System-C basics/CLM
		\item System-C TLM\newline
		\item {\color{blue}\ldots and your questions}
	\end{itemize}
\end{frame}
}

\mode<article>{
Welcome to the SystemC tutorial. 
We will introduce you to the exciting world of System-C and the possibilities opened to you with System-C Transaction Level Modelling (aka TLM).
We start with a presentation on what is System-C and what you can use it for.
Then the basics of System-C programming will be introduced, so you can start coding System-C cycle level models.
The third part introduces System-C TLM programming model, and more concretely TLM 2.0, which is deemed to become an IEEE standard like its big brother SystemC.
%Once the basics have been introduced, we will give an overview of the different existent frameworks that use System-C.
%A special section will be reserved for the UNISIM simulation framework.
%More on them at the end of the tutorial.

% \begin{figure}[!h]
% 	\begin{center}
% 		\includegraphics[page=1,height=6cm]{main_beamer.pdf}
% 	\end{center}
% 	\caption{Outline slide.}
% 	\label{slide:outline}
% \end{figure}
}

\newpage

\mode<article>{
\tableofcontents
\listoffigures
\listoftables
}

\input{introduction/intro_main}

\newpage

\input{systemc_cycle/main}

\newpage

\input{tlm/main}

% \newpage

% \input{systemc_basics/basics_main}

% \newpage

% \input{tlm/tlm_main}

% \newpage

% \input{services/services_main}

\end{document}


\newpage

\mode<article>{\usepackage{fullpage}}
\mode<presentation>{\usetheme{CEA}}
% everyone:
\usepackage[english]{babel}
\usepackage{pgf}
\usepackage{graphicx}
\usepackage{pstricks}
\usepackage{pdfpages}
\usepackage{listings}
\usepackage{verbatim}
\usepackage{tikz}
\usetikzlibrary{arrows}

\mode<article> {
	\usepackage[center]{caption}
}
%%\usepackage{xmpmulti}
%%\pgfdeclareimage[height=1cm]{myimage}{filename}

\mode<article> {
	\setjobnamebeamerversion{main.beamer}
}


\title{System-C \& System-C TLM Tutorial}
\author{Gilles Mouchard \\ Reda Nouacer \\ Daniel Gracia P\'erez}
\date{2 June, 2008}

\usetheme{CEA}

\begin{document}
\mode<presentation>{
\lstset{% general command to set parameter(s)
language=c++,
basicstyle=\tiny, % print whole listing tiny
keywordstyle=\color{black}\bfseries, % bold black keywords
identifierstyle=, % nothing happens
commentstyle=\color{blue}, % white comments
stringstyle=\ttfamily, % typewriter type for strings
showstringspaces=false,
tabsize=4,
breaklines=true,
numbers=left,
framexleftmargin=7.5mm,
xleftmargin=0.13\textwidth,
xrightmargin=0.1\textwidth,
frame=tlbr}
}
\mode<article>{
\lstset{% general command to set parameter(s)
language=c++,
basicstyle=\footnotesize, % print whole listing small
keywordstyle=\color{black}\bfseries, % bold black keywords
identifierstyle=, % nothing happens
commentstyle=\color{blue}, % white comments
stringstyle=\ttfamily, % typewriter type for strings
showstringspaces=false,
tabsize=4,
breaklines=true,
numbers=left,
framexleftmargin=7.5mm,
xleftmargin=0.13\textwidth,
xrightmargin=0.1\textwidth,
frame=tlbr}
}
% \lstset{language=c++,basicstyle={\ttfamily\footnotesize},tabsize=4,breaklines=true,numbers=left,framexleftmargin=7.5mm,xleftmargin=0.13\textwidth,xrightmargin=0.1\textwidth,frame=tlbr}
\maketitle

\mode<article>{
\chapter*{Outline}
}

\mode<presentation>{
\part{Outline}
}

\mode<presentation>{
\begin{frame}
	\frametitle{Outline}
	\begin{itemize}
		\item Introduction
		\item System-C basics/CLM
		\item System-C TLM\newline
		\item {\color{blue}\ldots and your questions}
	\end{itemize}
\end{frame}
}

\mode<article>{
Welcome to the SystemC tutorial. 
We will introduce you to the exciting world of System-C and the possibilities opened to you with System-C Transaction Level Modelling (aka TLM).
We start with a presentation on what is System-C and what you can use it for.
Then the basics of System-C programming will be introduced, so you can start coding System-C cycle level models.
The third part introduces System-C TLM programming model, and more concretely TLM 2.0, which is deemed to become an IEEE standard like its big brother SystemC.
%Once the basics have been introduced, we will give an overview of the different existent frameworks that use System-C.
%A special section will be reserved for the UNISIM simulation framework.
%More on them at the end of the tutorial.

% \begin{figure}[!h]
% 	\begin{center}
% 		\includegraphics[page=1,height=6cm]{main_beamer.pdf}
% 	\end{center}
% 	\caption{Outline slide.}
% 	\label{slide:outline}
% \end{figure}
}

\newpage

\mode<article>{
\tableofcontents
\listoffigures
\listoftables
}

\input{introduction/intro_main}

\newpage

\input{systemc_cycle/main}

\newpage

\input{tlm/main}

% \newpage

% \input{systemc_basics/basics_main}

% \newpage

% \input{tlm/tlm_main}

% \newpage

% \input{services/services_main}

\end{document}


% \newpage

% \input{systemc_basics/basics_main}

% \newpage

% \input{tlm/tlm_main}

% \newpage

% \input{services/services_main}

\end{document}


\newpage

\mode<article>{\usepackage{fullpage}}
\mode<presentation>{\usetheme{CEA}}
% everyone:
\usepackage[english]{babel}
\usepackage{pgf}
\usepackage{graphicx}
\usepackage{pstricks}
\usepackage{pdfpages}
\usepackage{listings}
\usepackage{verbatim}
\usepackage{tikz}
\usetikzlibrary{arrows}

\mode<article> {
	\usepackage[center]{caption}
}
%%\usepackage{xmpmulti}
%%\pgfdeclareimage[height=1cm]{myimage}{filename}

\mode<article> {
	\setjobnamebeamerversion{main.beamer}
}


\title{System-C \& System-C TLM Tutorial}
\author{Gilles Mouchard \\ Reda Nouacer \\ Daniel Gracia P\'erez}
\date{2 June, 2008}

\usetheme{CEA}

\begin{document}
\mode<presentation>{
\lstset{% general command to set parameter(s)
language=c++,
basicstyle=\tiny, % print whole listing tiny
keywordstyle=\color{black}\bfseries, % bold black keywords
identifierstyle=, % nothing happens
commentstyle=\color{blue}, % white comments
stringstyle=\ttfamily, % typewriter type for strings
showstringspaces=false,
tabsize=4,
breaklines=true,
numbers=left,
framexleftmargin=7.5mm,
xleftmargin=0.13\textwidth,
xrightmargin=0.1\textwidth,
frame=tlbr}
}
\mode<article>{
\lstset{% general command to set parameter(s)
language=c++,
basicstyle=\footnotesize, % print whole listing small
keywordstyle=\color{black}\bfseries, % bold black keywords
identifierstyle=, % nothing happens
commentstyle=\color{blue}, % white comments
stringstyle=\ttfamily, % typewriter type for strings
showstringspaces=false,
tabsize=4,
breaklines=true,
numbers=left,
framexleftmargin=7.5mm,
xleftmargin=0.13\textwidth,
xrightmargin=0.1\textwidth,
frame=tlbr}
}
% \lstset{language=c++,basicstyle={\ttfamily\footnotesize},tabsize=4,breaklines=true,numbers=left,framexleftmargin=7.5mm,xleftmargin=0.13\textwidth,xrightmargin=0.1\textwidth,frame=tlbr}
\maketitle

\mode<article>{
\chapter*{Outline}
}

\mode<presentation>{
\part{Outline}
}

\mode<presentation>{
\begin{frame}
	\frametitle{Outline}
	\begin{itemize}
		\item Introduction
		\item System-C basics/CLM
		\item System-C TLM\newline
		\item {\color{blue}\ldots and your questions}
	\end{itemize}
\end{frame}
}

\mode<article>{
Welcome to the SystemC tutorial. 
We will introduce you to the exciting world of System-C and the possibilities opened to you with System-C Transaction Level Modelling (aka TLM).
We start with a presentation on what is System-C and what you can use it for.
Then the basics of System-C programming will be introduced, so you can start coding System-C cycle level models.
The third part introduces System-C TLM programming model, and more concretely TLM 2.0, which is deemed to become an IEEE standard like its big brother SystemC.
%Once the basics have been introduced, we will give an overview of the different existent frameworks that use System-C.
%A special section will be reserved for the UNISIM simulation framework.
%More on them at the end of the tutorial.

% \begin{figure}[!h]
% 	\begin{center}
% 		\includegraphics[page=1,height=6cm]{main_beamer.pdf}
% 	\end{center}
% 	\caption{Outline slide.}
% 	\label{slide:outline}
% \end{figure}
}

\newpage

\mode<article>{
\tableofcontents
\listoffigures
\listoftables
}

\input{introduction/intro_main}

\newpage

\mode<article>{\usepackage{fullpage}}
\mode<presentation>{\usetheme{CEA}}
% everyone:
\usepackage[english]{babel}
\usepackage{pgf}
\usepackage{graphicx}
\usepackage{pstricks}
\usepackage{pdfpages}
\usepackage{listings}
\usepackage{verbatim}
\usepackage{tikz}
\usetikzlibrary{arrows}

\mode<article> {
	\usepackage[center]{caption}
}
%%\usepackage{xmpmulti}
%%\pgfdeclareimage[height=1cm]{myimage}{filename}

\mode<article> {
	\setjobnamebeamerversion{main.beamer}
}


\title{System-C \& System-C TLM Tutorial}
\author{Gilles Mouchard \\ Reda Nouacer \\ Daniel Gracia P\'erez}
\date{2 June, 2008}

\usetheme{CEA}

\begin{document}
\mode<presentation>{
\lstset{% general command to set parameter(s)
language=c++,
basicstyle=\tiny, % print whole listing tiny
keywordstyle=\color{black}\bfseries, % bold black keywords
identifierstyle=, % nothing happens
commentstyle=\color{blue}, % white comments
stringstyle=\ttfamily, % typewriter type for strings
showstringspaces=false,
tabsize=4,
breaklines=true,
numbers=left,
framexleftmargin=7.5mm,
xleftmargin=0.13\textwidth,
xrightmargin=0.1\textwidth,
frame=tlbr}
}
\mode<article>{
\lstset{% general command to set parameter(s)
language=c++,
basicstyle=\footnotesize, % print whole listing small
keywordstyle=\color{black}\bfseries, % bold black keywords
identifierstyle=, % nothing happens
commentstyle=\color{blue}, % white comments
stringstyle=\ttfamily, % typewriter type for strings
showstringspaces=false,
tabsize=4,
breaklines=true,
numbers=left,
framexleftmargin=7.5mm,
xleftmargin=0.13\textwidth,
xrightmargin=0.1\textwidth,
frame=tlbr}
}
% \lstset{language=c++,basicstyle={\ttfamily\footnotesize},tabsize=4,breaklines=true,numbers=left,framexleftmargin=7.5mm,xleftmargin=0.13\textwidth,xrightmargin=0.1\textwidth,frame=tlbr}
\maketitle

\mode<article>{
\chapter*{Outline}
}

\mode<presentation>{
\part{Outline}
}

\mode<presentation>{
\begin{frame}
	\frametitle{Outline}
	\begin{itemize}
		\item Introduction
		\item System-C basics/CLM
		\item System-C TLM\newline
		\item {\color{blue}\ldots and your questions}
	\end{itemize}
\end{frame}
}

\mode<article>{
Welcome to the SystemC tutorial. 
We will introduce you to the exciting world of System-C and the possibilities opened to you with System-C Transaction Level Modelling (aka TLM).
We start with a presentation on what is System-C and what you can use it for.
Then the basics of System-C programming will be introduced, so you can start coding System-C cycle level models.
The third part introduces System-C TLM programming model, and more concretely TLM 2.0, which is deemed to become an IEEE standard like its big brother SystemC.
%Once the basics have been introduced, we will give an overview of the different existent frameworks that use System-C.
%A special section will be reserved for the UNISIM simulation framework.
%More on them at the end of the tutorial.

% \begin{figure}[!h]
% 	\begin{center}
% 		\includegraphics[page=1,height=6cm]{main_beamer.pdf}
% 	\end{center}
% 	\caption{Outline slide.}
% 	\label{slide:outline}
% \end{figure}
}

\newpage

\mode<article>{
\tableofcontents
\listoffigures
\listoftables
}

\input{introduction/intro_main}

\newpage

\input{systemc_cycle/main}

\newpage

\input{tlm/main}

% \newpage

% \input{systemc_basics/basics_main}

% \newpage

% \input{tlm/tlm_main}

% \newpage

% \input{services/services_main}

\end{document}


\newpage

\mode<article>{\usepackage{fullpage}}
\mode<presentation>{\usetheme{CEA}}
% everyone:
\usepackage[english]{babel}
\usepackage{pgf}
\usepackage{graphicx}
\usepackage{pstricks}
\usepackage{pdfpages}
\usepackage{listings}
\usepackage{verbatim}
\usepackage{tikz}
\usetikzlibrary{arrows}

\mode<article> {
	\usepackage[center]{caption}
}
%%\usepackage{xmpmulti}
%%\pgfdeclareimage[height=1cm]{myimage}{filename}

\mode<article> {
	\setjobnamebeamerversion{main.beamer}
}


\title{System-C \& System-C TLM Tutorial}
\author{Gilles Mouchard \\ Reda Nouacer \\ Daniel Gracia P\'erez}
\date{2 June, 2008}

\usetheme{CEA}

\begin{document}
\mode<presentation>{
\lstset{% general command to set parameter(s)
language=c++,
basicstyle=\tiny, % print whole listing tiny
keywordstyle=\color{black}\bfseries, % bold black keywords
identifierstyle=, % nothing happens
commentstyle=\color{blue}, % white comments
stringstyle=\ttfamily, % typewriter type for strings
showstringspaces=false,
tabsize=4,
breaklines=true,
numbers=left,
framexleftmargin=7.5mm,
xleftmargin=0.13\textwidth,
xrightmargin=0.1\textwidth,
frame=tlbr}
}
\mode<article>{
\lstset{% general command to set parameter(s)
language=c++,
basicstyle=\footnotesize, % print whole listing small
keywordstyle=\color{black}\bfseries, % bold black keywords
identifierstyle=, % nothing happens
commentstyle=\color{blue}, % white comments
stringstyle=\ttfamily, % typewriter type for strings
showstringspaces=false,
tabsize=4,
breaklines=true,
numbers=left,
framexleftmargin=7.5mm,
xleftmargin=0.13\textwidth,
xrightmargin=0.1\textwidth,
frame=tlbr}
}
% \lstset{language=c++,basicstyle={\ttfamily\footnotesize},tabsize=4,breaklines=true,numbers=left,framexleftmargin=7.5mm,xleftmargin=0.13\textwidth,xrightmargin=0.1\textwidth,frame=tlbr}
\maketitle

\mode<article>{
\chapter*{Outline}
}

\mode<presentation>{
\part{Outline}
}

\mode<presentation>{
\begin{frame}
	\frametitle{Outline}
	\begin{itemize}
		\item Introduction
		\item System-C basics/CLM
		\item System-C TLM\newline
		\item {\color{blue}\ldots and your questions}
	\end{itemize}
\end{frame}
}

\mode<article>{
Welcome to the SystemC tutorial. 
We will introduce you to the exciting world of System-C and the possibilities opened to you with System-C Transaction Level Modelling (aka TLM).
We start with a presentation on what is System-C and what you can use it for.
Then the basics of System-C programming will be introduced, so you can start coding System-C cycle level models.
The third part introduces System-C TLM programming model, and more concretely TLM 2.0, which is deemed to become an IEEE standard like its big brother SystemC.
%Once the basics have been introduced, we will give an overview of the different existent frameworks that use System-C.
%A special section will be reserved for the UNISIM simulation framework.
%More on them at the end of the tutorial.

% \begin{figure}[!h]
% 	\begin{center}
% 		\includegraphics[page=1,height=6cm]{main_beamer.pdf}
% 	\end{center}
% 	\caption{Outline slide.}
% 	\label{slide:outline}
% \end{figure}
}

\newpage

\mode<article>{
\tableofcontents
\listoffigures
\listoftables
}

\input{introduction/intro_main}

\newpage

\input{systemc_cycle/main}

\newpage

\input{tlm/main}

% \newpage

% \input{systemc_basics/basics_main}

% \newpage

% \input{tlm/tlm_main}

% \newpage

% \input{services/services_main}

\end{document}


% \newpage

% \input{systemc_basics/basics_main}

% \newpage

% \input{tlm/tlm_main}

% \newpage

% \input{services/services_main}

\end{document}


% \newpage

% \input{systemc_basics/basics_main}

% \newpage

% \input{tlm/tlm_main}

% \newpage

% \input{services/services_main}

\end{document}


% \newpage

% \input{systemc_basics/basics_main}

% \newpage

% \input{tlm/tlm_main}

% \newpage

% \input{services/services_main}

\end{document}
