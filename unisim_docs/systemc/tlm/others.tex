\mode<article>{
	\clearpage
}

\section{Other things}

\mode<presentation>{
\begin{frame}
	\frametitle{Other things}
	\begin{itemize}
		\item Things appearing in the current TLM 2.0 revision:
		\begin{itemize}
			\item Analysis interface and analysis ports.
			\item Adapters examples.
			\item Generic payload extension mechanism.
		\end{itemize}
		\item Things missing in the current TLM 2.0 revision:
		\begin{itemize}
			\item New standard protocols.
			\item Atomic transactions.
		\end{itemize}
	\end{itemize}
\end{frame}
}

\mode<article>{
In this tutorial we have learn the basics of TLM 2.0, but still some things remain if you really want to learn all about TLM 2.0. 
For example:
\begin{itemize}
	\item we have not talked about the analysis interface and analysis ports.
	They are intended to support the distribution of transactions to multiple 
components for analysis, meaning tasks such as checking for functional correctness or collecting functional coverage statistics.
	\item we have not seen examples on how to connect modules using different coding styles or TLM modules with cycle level modules. 
	Those are called adapters, and the TLM 2.0 provides some examples on how this can be done. 
	Usually it is done with a module that translates (the adapter) the coding style of the initiator module with the coding style of the second module.
	\item you have seen that the generic payload could be extended, however we have not talked about this because it could be considered as a non basic feature. 
	Again, the TLM 2.0 user manual explains how this can be done. 
\end{itemize}

There are even some things that the TLM 2.0 proposed standard does not explain. 
For example, new protocols (e.g., interrupts) can appear in future version of the standard.
Also the TLM 2.0 user manual explicitly states that the other capabilities like the atomic transactions might appear in future revisions of the Accellera TLM.

}
