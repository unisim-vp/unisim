\section{Getting UNISIM}
\label{getting_unisim}

\subsection{Distribution as source code}
\label{distribution_as_source_code}

\subsubsection{Tarball}
\label{distribution_as_source_code_tarball}

Go to http://www.unisim.org/ \ldots

\subsubsection{Subversion}
\label{distribution_as_source_code_subversion}

\begin{verbatim}
$ svn checkout https://www.unisim.org/svn/devel
\end{verbatim}

\subsection{Distribution as binary package}
\label{distribution_as_binary_package}

\subsubsection{Ubuntu}
\label{distribution_as_binary_package_ubuntu}

Add the following line in your /etc/apt/source.list:
\begin{verbatim}
deb http://www.unisim.org/..... feisty contrib
deb http://www.unisim.org/..... gutsy contrib
\end{verbatim}

Then do the following to update the package manager:
\begin{verbatim}
$ sudo apt-get update
\end{verbatim}

\subsubsection{Windows}
\label{distribution_as_binary_package_windows}

Go to http://www.unisim.org/ \ldots

\section{Compiling UNISIM for Linux}
\label{compiling_unisim_for_linux}

\subsection{Ubuntu}
\label{compiling_unisim_for_ubuntu}

\subsubsection{Preparing}

\noindent \textbf{Installing some required packages}. The following packages are needed to compile UNISIM:
\begin{itemize}
\item bash
\item make
\item autoconf
\item automake
\item g++
\item flex
\item bison
\item libreadline5-dev
\item libxml2-dev
\item libsdl1.2-dev
\item libboost-graph-dev
\item libboost-thread-dev
\item zlib1g-dev
\end{itemize}

To install these packages, use apt-get or your usual package manager (synaptic, adept\ldots):
\begin{verbatim}
$ sudo apt-get bash make autoconf automake g++ flex bison libreadline5-dev
$ sudo apt-get libxml2-dev libsdl1.2-dev libboost-graph-dev libboost-thread-dev
\end{verbatim}

\noindent \textbf{Compiling systemc 2.2.0}. First, get your own SystemC 2.2.0 tarball from http://www.systemc.org (registration is needed), then do the following:

\begin{verbatim}
$ tar zxvf systemc-2.2.0.tgz
$ mkdir where-you-want-to-install-systemc
$ cd systemc-2.2.0
$ mkdir objdir
$ cd objdir
$ ../configure --prefix=where-you-want-to-install-systemc
$ make
$ make install
\end{verbatim}

\subsubsection{Compiling}

\noindent \textbf{Compiling unisim\_tools}.

\begin{verbatim}
$ ./configure --prefix=path-to-unisim-install-dir
$ make
$ make install
\end{verbatim}

\noindent \textbf{Compiling unisim\_lib}.

\begin{verbatim}
$ ./configure \
      --prefix=path-to-unisim-install-dir \
      --with-unisim-tools=path-to-unisim-install-dir \
      --with-systemc=path-to-systemc-install-dir
$ make
$ make install
\end{verbatim}

\noindent \textbf{Compiling unisim\_simulators}.

\begin{verbatim}
$ ./configure \
      --prefix=path-to-unisim-install-dir \
      --with-unisim-lib=path-to-unisim-install-dir \
      --with-systemc=path-to-systemc-install-dir
$ make
$ make install
\end{verbatim}

\section{UNISIM/Windows}
\label{compiling_unisim_for_windows}

\subsection{Cygwin}
\label{compiling_unisim_for_cygwin}

\subsubsection{Preparing}

\subsubsection{Compiling}

\subsection{Mingw32}
\label{compiling_unisim_for_mingw32}

To compile the unisim simulators for a Windows native environment, we rely on the ``autotools'' either on cygwin or Linux (Ubuntu).
We advice you to use Linux because it's the fastest solution we found.
Actually cygwin g++ compiler also includes gcc-mingw32 when invoked with option -mno-cygwin, and Ubuntu provides a MinGW32 cross-compiler (i586-mingw32msvc-g++).

\subsubsection{Preparing}
As UNISIM needs some third party libraries and MinGW32 only provides standard windows libraries, you will need to compile the following libraries for Windows:
\begin{itemize}
\item readline
\item libxml2
\item libsdl
\item boost
\item zlib
\end{itemize}

We greatly recommend to download the prebuilt libraries,  (we name this bootstrap-unisim-mingw32 for later reference) from: http://www.unisim.org/ \ldots

\subsubsection{Cross-compiling from Cygwin}

\noindent \textbf{Compiling unisim\_tools}.

\begin{verbatim}
$ cd unisim_tools
$ ./configure --prefix=path-to-unisim-install-dir \
    --host=i586-mingw32msvc \
    CXXFLAGS="-mno-cygwin" \
    LDFLAGS="-mno-cygwin"
$ make
$ make install
\end{verbatim}

\noindent \textbf{Compiling unisim\_lib}.

\begin{verbatim}
$ cd unisim_lib
$ ./configure --prefix=path-to-unisim-install-dir \
   --with-unisim-tools=path-to-unisim-install-dir \
   --host=i586-mingw32msvc \
    CXXFLAGS="-mno-cygwin" \
    LDFLAGS="-mno-cygwin" \
   --with-systemc=path-bootstrap-unisim-mingw32/systemc \
   --with-unisim-tools=path-to-unisim-install-dir \
   --with-sdl=path-bootstrap-unisim-mingw32/SDL \
   --with-boost=path-bootstrap-unisim-mingw32/boost \
   --with-readline=path-bootstrap-unisim-mingw32/readline \
   --with-libxml2/opt/mingw32/boost --enable-release \
   --with-zlib=path-bootstrap-unisim-mingw32/zlib \
   --enable-release
\end{verbatim}

\noindent \textbf{Compiling unisim\_simulators}.

\begin{verbatim}
$ cd unisim_simulators
$ ./configure --prefix=path-to-unisim-install-dir \
   --with-unisim-lib=path-to-unisim-install-dir \
   --host=i586-mingw32msvc \
    CXXFLAGS="-mno-cygwin" \
    LDFLAGS="-mno-cygwin" \
   --with-systemc=path-bootstrap-unisim-mingw32/systemc \
   --with-unisim-tools=path-to-unisim-install-dir \
   --with-sdl=path-bootstrap-unisim-mingw32/SDL \
   --with-boost=path-bootstrap-unisim-mingw32/boost \
   --with-readline=path-bootstrap-unisim-mingw32/readline \
   --with-libxml2/opt/mingw32/boost \
   --with-zlib=path-bootstrap-unisim-mingw32/zlib \
   --enable-release
\end{verbatim}

\subsubsection{Cross-compiling from Linux (Ubuntu)}

\noindent \textbf{Compiling unisim\_tools}.

\begin{verbatim}
$ cd unisim_tools
$ ./configure --prefix=path-to-unisim-install-dir --host=i586-mingw32msvc
$ make
$ make install
\end{verbatim}

\noindent \textbf{Compiling unisim\_lib}.

\begin{verbatim}
$ cd unisim_lib
$ ./configure --prefix=path-to-unisim-install-dir \
   --with-unisim-tools=path-to-unisim-install-dir \
   --host=i586-mingw32msvc \
   --with-systemc=path-bootstrap-unisim-mingw32/systemc \
   --with-unisim-tools=path-to-unisim-install-dir \
   --with-sdl=path-bootstrap-unisim-mingw32/SDL \
   --with-boost=path-bootstrap-unisim-mingw32/boost \
   --with-readline=path-bootstrap-unisim-mingw32/readline \
   --with-libxml2/opt/mingw32/boost --enable-release \
   --with-zlib=path-bootstrap-unisim-mingw32/zlib \
   --enable-release
\end{verbatim}

\noindent \textbf{Compiling unisim\_simulators}.

\begin{verbatim}
$ cd unisim_simulators
$ ./configure --prefix=path-to-unisim-install-dir \
   --with-unisim-lib=path-to-unisim-install-dir \
   --host=i586-mingw32msvc \
   --with-systemc=path-bootstrap-unisim-mingw32/systemc \
   --with-unisim-tools=path-to-unisim-install-dir \
   --with-sdl=path-bootstrap-unisim-mingw32/SDL \
   --with-boost=path-bootstrap-unisim-mingw32/boost \
   --with-readline=path-bootstrap-unisim-mingw32/readline \
   --with-libxml2/opt/mingw32/boost \
   --with-zlib=path-bootstrap-unisim-mingw32/zlib \
   --enable-release
\end{verbatim}
