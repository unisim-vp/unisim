\section{Getting UNISIM}
\label{getting_unisim}

\subsection{Distribution as source code}
\label{distribution_as_source_code}

\subsubsection{Tarball}
\label{distribution_as_source_code_tarball}

\subsubsection{Subversion}
\label{distribution_as_source_code_subversion}

\subsection{Distribution as binary package}
\label{distribution_as_binary_package}

\subsubsection{Ubuntu}
\label{distribution_as_binary_package_ubuntu}

\subsubsection{Windows}
\label{distribution_as_binary_package_windows}

\section{Compiling UNISIM for Linux}
\label{compiling_unisim_for_linux}

\subsection{Ubuntu}
\label{compiling_unisim_for_ubuntu}

\subsubsection{Preparing}

\noindent \textbf{Installing some required packages}. The following packages are needed to compile UNISIM:
\begin{itemize}
\item bash
\item make
\item autoconf
\item automake
\item g++
\item flex
\item bison
\item libreadline5-dev
\item libxml2-dev
\item libsdl1.2-dev
\item libboost-graph-dev
\item libboost-thread-dev
\item zlib1g-dev
\end{itemize}

To install these packages, use apt-get or your usual package manager (synaptic, adept\ldots):
\begin{verbatim}
$ sudo apt-get bash make autoconf automake g++ flex bison libreadline5-dev
$ sudo apt-get libxml2-dev libsdl1.2-dev libboost-graph-dev libboost-thread-dev
\end{verbatim}

\noindent \textbf{Compiling systemc 2.2.0}. First, get your own SystemC 2.2.0 tarball from http://www.systemc.org (registration is needed), then do the following:

\begin{verbatim}
$ tar zxvf systemc-2.2.0.tgz
$ mkdir where-you-want-to-install-systemc
$ cd systemc-2.2.0
$ mkdir objdir
$ cd objdir
$ ../configure --prefix=where-you-want-to-install-systemc
$ make
$ make install
\end{verbatim}

\subsubsection{Compiling}

\noindent \textbf{Compiling unisim\_tools}.

\begin{verbatim}
$ ./configure --prefix=path-to-unisim-install-dir
$ make
$ make install
\end{verbatim}

\noindent \textbf{Compiling unisim\_lib}.

\begin{verbatim}
$ ./configure \
      --prefix=path-to-unisim-install-dir \
      --with-unisim-tools=path-to-unisim-install-dir \
      --with-systemc=path-to-systemc-install-dir
$ make
$ make install
\end{verbatim}

\noindent \textbf{Compiling unisim\_simulators}.

\begin{verbatim}
$ ./configure \
      --prefix=path-to-unisim-install-dir \
      --with-unisim-lib=path-to-unisim-install-dir \
      --with-systemc=path-to-systemc-install-dir
$ make
$ make install
\end{verbatim}

\section{UNISIM/Windows}
\label{compiling_unisim_for_windows}

\subsection{Cygwin}
\label{compiling_unisim_for_cygwin}

\subsubsection{Preparing}

\subsubsection{Compiling}

\subsection{Mingw32}
\label{compiling_unisim_for_mingw32}

\subsubsection{Preparing}

\subsubsection{Compiling}
