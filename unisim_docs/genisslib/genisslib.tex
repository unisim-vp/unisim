\section{Introduction}

%-------------------------------------------------------------------------
Instruction set simulators are usefull for processor simulation, either
architecture or micro-architecture simulation. Architecture simulation,
also called functional simulation, refers to simulation of the processor
instruction set, whereas micro-architecture simulation refers to components
inside the processor such as pipelines, caches, functional units, branch
predictors. Implementing the instruction set into a software instruction set
simulator is needed for developping and testing software before the target
processor is available or for analyzing softwares without disturbing their execution.
To go from the instruction set specification to a software
implementation of the instruction set, it is convenient to have a description language
for easily describing the instruction encoding, but most description languages
are restricted to only few syntaxical constructions for describing the
instructions behavior. Beside, it is important to have a multipurpose instruction decoder as needs evolves.
GenISSLib is intended for developping such instruction set simulators. It uses a
description language for describing instruction encoding. C++ source code blended with
the description language allow the use of complex instruction behavior, or even more functionalities
like disassembling, binary translation, system calls translation.

\section{Quick Start}

We will now presents you how to use GenISSLib through an example. Suppose that we want to build an instruction set simulator for
a simple 16-bit RISC processor, see table~\ref{instruction_set_summary}. We need GenISSLib to generate C++ class 'Decoder' to decode an instruction, and a C++ class 'Operation' representing the decoded instruction.
class 'Operation' would have a C++ method named 'execute' to execute the instruction, and a C++ method to disassemble the instruction into a string buffer:
\begin{itemize}
	\item \texttt{Operation *Decoder::decode(uint16\_t addr, uint16\_t instruction)}
	\begin{itemize}
		\item \texttt{addr} is the address of the instruction
		\item \texttt{instruction} is the instruction word to decode
		\item \texttt{decode} returns an instance of class \texttt{Operation} containing the decoded instruction
	\end{itemize}
	\item \texttt{void Operation::execute()}
	\item \texttt{void Operation::disasm(uint16\_t pc, char *s)}
	\begin{itemize}
		\item \texttt{pc} is the address of the instruction
		\item \texttt{s} is a string where to disassemble the instruction
	\end{itemize}
\end{itemize}

GenISSLib automatically generates the \texttt{Decoder} class. To ask GenISSLib to generate \texttt{execute} and \texttt{disasm} methods, we have
to declare them using the description language, see figure~\ref{action}. The grammar of the description language is on figure~\ref{grammar}. In the description language, \texttt{execute} and \texttt{disasm} are actions having a prototype.
The action prototype is simply a template for the action. As you can see, the declaration in the instruction set simulator source code directly derives
from these action prototypes. If an instruction is invalid, we can tell GenISSLib to use a default implementation for the action. The default
action implementation for \texttt{execute} is printing \texttt{"Unknown instruction\textbackslash n"} and exit whereas the default action implementation
for \texttt{disasm} is writing a \texttt{"?"} into the string \texttt{s}.

\begin{figure}[ht]
\begin{verbatim}
action {void} execute() {
    printf("Unknown instruction\n");
    exit(-1);
}

action {void} disasm({uint16_t} {pc}, {char *} {s}) {
    sprintf(s, "?");
}
\end{verbatim}
\caption{Declaring the \texttt{execute} and \texttt{disasm} action prototypes.}
\label{action}
\end{figure}

\begin{figure}[h!tb]
	\begin{center}
		\begin{tabular}{|c|c|c|c|c|c|c|c|c|c|c|c|c|c|c|c|c|}
			\hline
			instructions &  15  &  14  &  13  &  12  &  11  &  10  &  ~9  & ~8  &  ~7  &  ~6  &  ~5  &  ~4  &  ~3  &  ~2  &  ~1  &  ~0 \\
			\hline
			add    &   0  &   0  &   0  &   0  & \multicolumn{4}{|c|}{rd} & \multicolumn{4}{|c|}{rs1} & \multicolumn{4}{|c|}{rs2}\\
			\hline
			sub    &   0  &   0  &   0  &   1  & \multicolumn{4}{|c|}{rd} & \multicolumn{4}{|c|}{rs1} & \multicolumn{4}{|c|}{rs2}\\
			\hline
			or     &   0  &   0  &   1  &   0  & \multicolumn{4}{|c|}{rd} & \multicolumn{4}{|c|}{rs1} & \multicolumn{4}{|c|}{rs2}\\
			\hline
			and    &   0  &   0  &   1  &   1  & \multicolumn{4}{|c|}{rd} & \multicolumn{4}{|c|}{rs1} & \multicolumn{4}{|c|}{rs2}\\
			\hline
			not    &   0  &   1  &   0  &   0  & \multicolumn{4}{|c|}{rd}  & \multicolumn{4}{|c|}{rs} &   x  &   x  &   x  &   x \\
			\hline
			shl    &   0  &   1  &   0  &   1  & \multicolumn{4}{|c|}{rd}  & \multicolumn{4}{|c|}{rs} &   x  &   x  &   x  &   x \\
			\hline
			shr    &   0  &   1  &   1  &   0  & \multicolumn{4}{|c|}{rd}  & \multicolumn{4}{|c|}{rs} &   x  &   x  &   x  &   x \\
			\hline
			load   &   0  &   1  &   1  &   1  & \multicolumn{4}{|c|}{rd}  & \multicolumn{4}{|c|}{base} & \multicolumn{4}{|c|}{index}\\
			\hline
			store  &   1  &   0  &   0  &   0  & \multicolumn{4}{|c|}{rd}  & \multicolumn{4}{|c|}{base} & \multicolumn{4}{|c|}{index}\\
			\hline
			branch &   1  &   0  &   0  &   1  & \multicolumn{12}{|c|}{addr}\\
			\hline
			bne    &   1  &   0  &   1  &   0  & \multicolumn{4}{|c|}{rs} & \multicolumn{8}{|c|}{offset}\\
			\hline
		\end{tabular}
		\caption{Instruction Set Summary.}
		\label{instruction_set_summary}
	\end{center}
\end{figure}

\begin{figure}[h!tb]
\begin{center}
	\begin{tabular}{cccccccccccccccc}
		\multicolumn{4}{l}{\Large{\bf{add}}} & \multicolumn{12}{l}{add}\\
		\multicolumn{16}{c}{}\\
		15  &  14  &  13  &  12  &  11  &  10  &  ~9  & ~8  &  ~7  &  ~6  &  ~5  &  ~4  &  ~3  &  ~2  &  ~1  &  ~0 \\
		\hline
			\multicolumn{1}{|c|}{0}  & \multicolumn{1}{|c|}{0} & \multicolumn{1}{|c|}{0} & \multicolumn{1}{|c|}{0} & \multicolumn{4}{|c|}{rd} & \multicolumn{4}{|c|}{rs1} & \multicolumn{4}{|c|}{rs2}\\
		\hline
		\multicolumn{16}{c}{}\\
		\multicolumn{16}{l}{(rd) $\leftarrow$ (rs1) + (rs2)}\\
	\end{tabular}
\end{center}
\caption{The add instruction.}
\label{add_instr}
\end{figure}

\begin{figure}[h!tb]
\begin{center}
\begin{verbatim}
op add(0b0000[4]:rd[4]:rs1[4]:rs2[4])
\end{verbatim}
\label{add_decl}
\caption{declaration of the add operation}
\end{center}
\end{figure}

\begin{figure}[h!tb]
\begin{center}
	\begin{tabular}{cccccccccccccccc}
		\multicolumn{4}{l}{\Large{\bf{not}}} & \multicolumn{12}{l}{negation}\\
		\multicolumn{16}{c}{}\\
		15  &  14  &  13  &  12  &  11  &  10  &  ~9  & ~8  &  ~7  &  ~6  &  ~5  &  ~4  &  ~3  &  ~2  &  ~1  &  ~0 \\
		\hline
		\multicolumn{1}{|c|}{0}  & \multicolumn{1}{|c|}{1} & \multicolumn{1}{|c|}{0} & \multicolumn{1}{|c|}{0} & \multicolumn{4}{|c|}{rd} & \multicolumn{4}{|c|}{rs} & \multicolumn{1}{|c|}{x} & \multicolumn{1}{|c|}{x} & \multicolumn{1}{|c|}{x} & \multicolumn{1}{|c|}{x}\\
		\hline
		\multicolumn{16}{c}{}\\
		\multicolumn{16}{l}{(rd) $\leftarrow$ $~$(rs)}\\
	\end{tabular}
\end{center}
\caption{The not instruction.}
\label{not_instr}
\end{figure}

\begin{figure}[h!tb]
\begin{center}
\begin{verbatim}
op not(0b0100[4]:rd[4]:rs[4]:?[4])
\end{verbatim}
\label{not_decl}
\caption{declaration of the not operation}
\end{center}
\end{figure}

We can now declare instruction encoding and the action implementation with the description language. First, let's consider the add instruction, see figures \ref{add_instr} and \ref{add_decl}: it has a 4-bit opcode
which is 0000 and three 4-bit long operand register numbers encoded into the instruction word which are rd, rs1 and rs2. In the description language \texttt{operation} refers to an instruction.
\texttt{0b0000[4]} refers to the 4-bit opcode, \texttt{rd[4]} to the destination register number which is 4-bit long, \texttt{rs1[4]} and \texttt{rs2[4]} respectively
to the first and second source register numbers. Each bit fields are separated by \texttt{:}. Instead of using binary digits as in \texttt{0b0000[4]}, we can also use decimal digits as in \texttt{0[4]}, or
hexadecimal digits as in \texttt{0x0[4]}.

Now, let's declare the not instruction encoding, see figures \ref{not_instr} and \ref{not_decl}:  it has a 4-bit opcode which is 0001, two 4-bit long operand register numbers which are rd and rs, and 4 don't care bits which can be either 0 or 1.
The 4 don't care bits are represented by \texttt{?[4]} in the declaration of the not operation.

Now, let's implement the actions of the add operation, see figures \ref{add_instr} and \ref{add_impl}. Implementation of an action is just C code.
We can directly use rd, rs1 and rs2 which have been declared in the operation. We use an array (\texttt{gpr}) to represent the registers. The external declaration of that C array can be done anywhere into the description language : we've just added between \{ and \}, the external declarations of \texttt{gpr} and the memory access functions \texttt{mem\_read} and \texttt{mem\_write}, see figure~\ref{c_decl}.

\begin{figure}[h!tb]
\begin{center}
\begin{verbatim}
add.execute = {
    gpr[rd] = gpr[rs1] + gpr[rs2];
}

add.disasm = {
    sprintf(s, "add r%u, r%u, r%u", rd, rs1, rs2);
}
\end{verbatim}
\caption{implementation of the \texttt{execute} and \texttt{disasm} of the add operation.}
\label{add_impl}
\end{center}
\end{figure}

\begin{figure}[ht]
\begin{center}
\begin{verbatim}
{
#include <inttypes.h>
#include <stdio.h>

extern uint16_t cia;
extern uint16_t nia;
extern uint16_t gpr[16];

extern uint16_t mem_read(uint16_t addr);
extern void mem_write(uint16_t addr, uint16_t value);
}
\end{verbatim}
\caption{external declaration of C/C++ variables and functions.}
\label{c_decl}
\end{center}
\end{figure}

\begin{figure}[h!tb]
\begin{center}
	\begin{tabular}{cccccccccccccccc}
		\multicolumn{4}{l}{\Large{\bf{load}}} & \multicolumn{12}{l}{load word}\\
		\multicolumn{16}{c}{}\\
		15  &  14  &  13  &  12  &  11  &  10  &  ~9  & ~8  &  ~7  &  ~6  &  ~5  &  ~4  &  ~3  &  ~2  &  ~1  &  ~0 \\
		\hline
		\multicolumn{1}{|c|}{0}  & \multicolumn{1}{|c|}{1} & \multicolumn{1}{|c|}{1} & \multicolumn{1}{|c|}{1} & \multicolumn{4}{|c|}{rd} & \multicolumn{4}{|c|}{base} & \multicolumn{4}{|c|}{index}\\
		\hline
		\multicolumn{16}{c}{}\\
		\multicolumn{16}{l}{ea $\leftarrow$ (base) + index}\\
		\multicolumn{16}{l}{(rd) $\leftarrow$ Mem[ea]}\\
	\end{tabular}
\end{center}
\caption{The load instruction.}
\label{load_instr}
\end{figure}

Now, let's write the declaration of the load operation, and the implementation of its actions, see figures \ref{load_instr} and \ref{load_decl}: it has a 4-bit index which
must be sign extended and added to a base register to compute the effective address of the load.
Thus we use the \texttt{sext} modifier into the operation declaration. The minimum size in bits of the C variable \texttt{index} holding the \texttt{index} operand field of the operation is supplied between \textless and \textgreater. When decoding a load instruction, GenISSLib will sign extends the \texttt{index} bit field, and store it into the \texttt{index} C variable.

\begin{figure}[h!tb]
\begin{center}
\begin{verbatim}
op load(0b0111[4]:rd[4]:base[4]:sext<16> immed[4])

load.execute = {
    gpr[rd] = mem_read(gpr[base] + immed);
}

load.disasm = {
    sprintf(s, "load r%u, %d(r%u)", rd, immed, base);
}
\end{verbatim}
\caption{The load declaration and implementation.}
\label{load_decl}
\end{center}
\end{figure}

\begin{figure}[h!tb]
\begin{center}
	\begin{tabular}{cccccccccccccccc}
		\multicolumn{4}{l}{\Large{\bf{bne}}} & \multicolumn{12}{l}{conditional branch}\\
		\multicolumn{16}{c}{}\\
		15  &  14  &  13  &  12  &  11  &  10  &  ~9  & ~8  &  ~7  &  ~6  &  ~5  &  ~4  &  ~3  &  ~2  &  ~1  &  ~0 \\
		\hline
		\multicolumn{1}{|c|}{1}  & \multicolumn{1}{|c|}{0} & \multicolumn{1}{|c|}{1} & \multicolumn{1}{|c|}{0} & \multicolumn{4}{|c|}{rs} &\multicolumn{8}{|c|}{offset}\\
		\hline
		\multicolumn{16}{c}{}\\
		\multicolumn{16}{l}{if (rs) \textless \textgreater 0 then}\\
		\multicolumn{1}{l}{} &\multicolumn{15}{l}{NIA $\leftarrow$ CIA + offset}\\
		\multicolumn{16}{l}{else}\\
		\multicolumn{1}{l}{} & \multicolumn{15}{l}{NIA $\leftarrow$ CIA + 1}\\
	\end{tabular}
\end{center}
\caption{The bne instruction.}
\label{bne_instr}
\end{figure}

The implementation of the \texttt{disasm} action of the bne operation uses the \texttt{pc} parameter because we need that the disassembling of the bne instruction contains the target address of the conditional branch, that is \texttt{pc + offset} not \texttt{offset}, see figures \ref{bne_instr} and \ref{bne_decl}.

\begin{figure}[h!tb]
\begin{center}
\begin{verbatim}
op bne(0b1010[4]:rs[4]:sext<16> offset[8])
bne.execute = {
    if(gpr[rs] != 0)
        nia = cia + offset;
}
bne.disasm = {
    sprintf(s, "bne r%u, 0x%04x", rs, pc + offset);
}
\end{verbatim}
\caption{The bne declaration and implementation.}
\label{bne_decl}
\end{center}
\end{figure}

Finally, we uses the generated function to implement the instruction set simulator main loop, see figure~\ref{main}.
We also implement the external function used into the action implementations, see figure~\ref{extern_impl}.

\begin{figure}[h!tb]
\begin{center}
\begin{verbatim}
#include <inttypes.h>"

uint16_t cia;
uint16_t nia;
uint16_t gpr[16];
uint16_t mem[1 << 16];

uint16_t mem_read(uint16_t addr) {
    return mem[addr];
}

void mem_write(uint16_t addr, uint16_t value) {
    mem[addr] = value;
}
\end{verbatim}
\caption{The implementation of the external data and functions.}
\label{extern_impl}
\end{center}
\end{figure}

\begin{figure}[h!tb]
\begin{center}
\begin{verbatim}
#include "risc16.hh"
#include <stdio.h>

int main(int argc, char *argv[]) {
    int i;
    FILE *f;
    uint16_t instruction;
    Operation *operation;
    char disasm_buffer[256];

    cia = 0;
    nia = 0;
    for(i = 0; i < 16; i++) gpr[i] = 0;
    f = fopen("image", "rb");
    fread(mem, 1, sizeof(mem), f);
    fclose(f);
    Decoder decoder;
    for(i = 0; i < 1000; i++)	{
        instruction = mem_read(cia);
        operation = decoder.decode(cia, instruction);
        operation->disasm(cia, disasm_buffer);
        printf("0x%04x: 0x%04x %s\n", cia, instruction,
                                            disasm_buffer);
        nia = cia + 1;
        operation->execute();
        cia = nia;
    }
    return 0;
}
\end{verbatim}
\caption{The main simulation loop.}
\label{main}
\end{center}
\end{figure}



% \begin{figure}[ht]
% \begin{center}
% 	\begin{tabular}{cccccccccccccccc}
% 		\multicolumn{4}{l}{\Large{\bf{sub}}} & \multicolumn{12}{l}{substract}\\
% 		\multicolumn{16}{c}{}\\
% 		15  &  14  &  13  &  12  &  11  &  10  &  ~9  & ~8  &  ~7  &  ~6  &  ~5  &  ~4  &  ~3  &  ~2  &  ~1  &  ~0 \\
% 		\hline
% 			\multicolumn{1}{|c|}{0}  & \multicolumn{1}{|c|}{0} & \multicolumn{1}{|c|}{0} & \multicolumn{1}{|c|}{1} & \multicolumn{4}{|c|}{rd} & \multicolumn{4}{|c|}{rs1} & \multicolumn{4}{|c|}{rs2}\\
% 		\hline
% 		\multicolumn{16}{c}{}\\
% 		\multicolumn{16}{l}{(rd) $\leftarrow$ (rs1) - (rs2)}\\
% 	\end{tabular}
% \end{center}
% \caption{The sub instruction.}
% \label{sub_instr}
% \end{figure}
% 
% \begin{figure}[ht]
% \begin{center}
% 	\begin{tabular}{cccccccccccccccc}
% 		\multicolumn{4}{l}{\Large{\bf{or}}} & \multicolumn{12}{l}{logical or}\\
% 		\multicolumn{16}{c}{}\\
% 		15  &  14  &  13  &  12  &  11  &  10  &  ~9  & ~8  &  ~7  &  ~6  &  ~5  &  ~4  &  ~3  &  ~2  &  ~1  &  ~0 \\
% 		\hline
% 			\multicolumn{1}{|c|}{0}  & \multicolumn{1}{|c|}{0} & \multicolumn{1}{|c|}{1} & \multicolumn{1}{|c|}{0} & \multicolumn{4}{|c|}{rd} & \multicolumn{4}{|c|}{rs1} & \multicolumn{4}{|c|}{rs2}\\
% 		\hline
% 		\multicolumn{16}{c}{}\\
% 		\multicolumn{16}{l}{(rd) $\leftarrow$ (rs1) \textbar (rs2)}\\
% 	\end{tabular}
% \end{center}
% \caption{The or instruction.}
% \label{or_instr}
% \end{figure}
% 
% \begin{figure}[ht]
% \begin{center}
% 	\begin{tabular}{cccccccccccccccc}
% 		\multicolumn{4}{l}{\Large{\bf{and}}} & \multicolumn{12}{l}{logical and}\\
% 		\multicolumn{16}{c}{}\\
% 		15  &  14  &  13  &  12  &  11  &  10  &  ~9  & ~8  &  ~7  &  ~6  &  ~5  &  ~4  &  ~3  &  ~2  &  ~1  &  ~0 \\
% 		\hline
% 			\multicolumn{1}{|c|}{0}  & \multicolumn{1}{|c|}{0} & \multicolumn{1}{|c|}{1} & \multicolumn{1}{|c|}{1} & \multicolumn{4}{|c|}{rd} & \multicolumn{4}{|c|}{rs1} & \multicolumn{4}{|c|}{rs2}\\
% 		\hline
% 		\multicolumn{16}{c}{}\\
% 		\multicolumn{16}{l}{(rd) $\leftarrow$ (rs1) \& (rs2)}\\
% 	\end{tabular}
% \end{center}
% \caption{The and instruction.}
% \label{and_instr}
% \end{figure}
% 
% 
% \begin{figure}[ht]
% \begin{center}
% 	\begin{tabular}{cccccccccccccccc}
% 		\multicolumn{4}{l}{\Large{\bf{shl}}} & \multicolumn{12}{l}{Shift left}\\
% 		\multicolumn{16}{c}{}\\
% 		15  &  14  &  13  &  12  &  11  &  10  &  ~9  & ~8  &  ~7  &  ~6  &  ~5  &  ~4  &  ~3  &  ~2  &  ~1  &  ~0 \\
% 		\hline
% 		\multicolumn{1}{|c|}{0}  & \multicolumn{1}{|c|}{1} & \multicolumn{1}{|c|}{0} & \multicolumn{1}{|c|}{1} & \multicolumn{4}{|c|}{rd} & \multicolumn{4}{|c|}{rs} & \multicolumn{1}{|c|}{x} & \multicolumn{1}{|c|}{x} & \multicolumn{1}{|c|}{x} & \multicolumn{1}{|c|}{x}\\
% 		\hline
% 		\multicolumn{16}{c}{}\\
% 		\multicolumn{16}{l}{(rd) $\leftarrow$ (rs) \textless \textless 1}\\
% 	\end{tabular}
% \end{center}
% \caption{The shl instruction.}
% \label{shl_instr}
% \end{figure}
% 
% \begin{figure}[ht]
% \begin{center}
% 	\begin{tabular}{cccccccccccccccc}
% 		\multicolumn{4}{l}{\Large{\bf{shr}}} & \multicolumn{12}{l}{Shift right}\\
% 		\multicolumn{16}{c}{}\\
% 		15  &  14  &  13  &  12  &  11  &  10  &  ~9  & ~8  &  ~7  &  ~6  &  ~5  &  ~4  &  ~3  &  ~2  &  ~1  &  ~0 \\
% 		\hline
% 		\multicolumn{1}{|c|}{0}  & \multicolumn{1}{|c|}{1} & \multicolumn{1}{|c|}{1} & \multicolumn{1}{|c|}{0} & \multicolumn{4}{|c|}{rd} & \multicolumn{4}{|c|}{rs} & \multicolumn{1}{|c|}{x} & \multicolumn{1}{|c|}{x} & \multicolumn{1}{|c|}{x} & \multicolumn{1}{|c|}{x}\\
% 		\hline
% 		\multicolumn{16}{c}{}\\
% 		\multicolumn{16}{l}{(rd) $\leftarrow$ (rs) \textgreater \textgreater 1}\\
% 	\end{tabular}
% \end{center}
% \caption{The shr instruction.}
% \label{shr_instr}
% \end{figure}
% 
% 
% \begin{figure}[ht]
% \begin{center}
% 	\begin{tabular}{cccccccccccccccc}
% 		\multicolumn{4}{l}{\Large{\bf{store}}} & \multicolumn{12}{l}{store word}\\
% 		\multicolumn{16}{c}{}\\
% 		15  &  14  &  13  &  12  &  11  &  10  &  ~9  & ~8  &  ~7  &  ~6  &  ~5  &  ~4  &  ~3  &  ~2  &  ~1  &  ~0 \\
% 		\hline
% 		\multicolumn{1}{|c|}{1}  & \multicolumn{1}{|c|}{0} & \multicolumn{1}{|c|}{0} & \multicolumn{1}{|c|}{0} & \multicolumn{4}{|c|}{rs} & \multicolumn{4}{|c|}{base} & \multicolumn{4}{|c|}{index}\\
% 		\hline
% 		\multicolumn{16}{c}{}\\
% 		\multicolumn{16}{l}{ea $\leftarrow$ (base) + index}\\
% 		\multicolumn{16}{l}{Mem[ea]  $\leftarrow$ (rs)}\\
% 	\end{tabular}
% \end{center}
% \caption{The store instruction.}
% \label{store_instr}
% \end{figure}
%
% \begin{figure}[ht]
% \begin{center}
% 	\begin{tabular}{cccccccccccccccc}
% 		\multicolumn{4}{l}{\Large{\bf{branch}}} & \multicolumn{12}{l}{unconditional branch}\\
% 		\multicolumn{16}{c}{}\\
% 		15  &  14  &  13  &  12  &  11  &  10  &  ~9  & ~8  &  ~7  &  ~6  &  ~5  &  ~4  &  ~3  &  ~2  &  ~1  &  ~0 \\
% 		\hline
% 		\multicolumn{1}{|c|}{1}  & \multicolumn{1}{|c|}{0} & \multicolumn{1}{|c|}{0} & \multicolumn{1}{|c|}{1} & \multicolumn{12}{|c|}{addr}\\
% 		\hline
% 		\multicolumn{16}{c}{}\\
% 		\multicolumn{16}{l}{NIA $\leftarrow$ addr}\\
% 	\end{tabular}
% \end{center}
% \caption{The branch instruction.}
% \label{branch_instr}
% \end{figure}

\newpage

\section{Technical Reference}

\begin{figure}[p]
input $\rightarrow$ decl\_list\\
decl\_list $\rightarrow$ $\epsilon$ \textbar~~decl\_list decl {\bf $\hookleftarrow$}\\
decl $\rightarrow$ $\epsilon$ \textbar~~op\_decl \textbar~~action\_proto\_decl \textbar~~action\_decl \textbar~~source\_code\_decl \textbar~~include\_decl\\
source\_code\_decl $\rightarrow$ {\bf source\_code}\\
op\_decl $\rightarrow$ {\bf op} {\bf identifier} {\bf (} bitfield\_list {\bf )}\\
bitfield\_list $\rightarrow$ bitfield \textbar~ bitfield\_list {\bf :} bitfield\\
bitfield $\rightarrow$ {\bf integer} {\bf [} {\bf integer} {\bf ]} \textbar~ sext\_modifier size\_modifier {\bf identifier} {\bf [} {\bf integer} {\bf ]} \textbar~ ? {\bf [} {\bf integer}�{\bf ]}\\
sext\_modifier $\rightarrow$ $\epsilon$ \textbar~ {\bf sext}\\
size\_modifier $\rightarrow$ $\epsilon$ \textbar~ {\bf \textless}~ {\bf integer} {\bf \textgreater}~ \textbar~ {\bf \textless}~ {\bf \textgreater}~\\
action\_proto\_decl $\rightarrow$ static\_modifier {\bf action} {\bf source\_code} {\bf identifier} {\bf (} params {\bf )} {\bf source\_code}\\
params $\rightarrow$ $\epsilon$ \textbar~ {\bf source\_code}\\
static\_modifier $\rightarrow$ $\epsilon$ \textbar~ {\bf static}\\
action\_decl $\rightarrow$ {\bf identifier} {\bf.} {\bf identifier} {\bf =} {\bf source\_code}\\
include\_decl $\rightarrow$ {\bf include} {\bf string}\\
\\
{\bf source\_code} is C++ source between \{ and \}\\
{\bf identifier} is an alphanumeric identifier\\
{\bf integer} is a positive integer either binary, decimal or hexadecimal\\
{\bf string} is a double-quoted character string\\
C ( /* */) et C++ ( // ) comments are allowed anywhere.
\caption{The description language grammar.}
\label{grammar}
\end{figure}
