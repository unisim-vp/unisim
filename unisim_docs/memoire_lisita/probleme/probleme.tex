\subsection{Etat de l'art}

\subsubsection{Présentation des méthodes de conception d'un simulateur}

présentation à partir du livre "Study of the techniques for emulation programming"

\subsubsection{Présentation de GenISSLib}

voir tutoriel "Unisim GenISSLib Manual" de G Mouchard 

 \subsubsection{Introduction à SystemC TLM2}

présentation à partir des exercice du document "SystemC and SystemC TLM tutorial" de G. Mouchard, R. Nouacer
 
\subsubsection{Présentation de la plateforme unisim}

\subsection{Présentation des outils existants}

\begin{itemize}
\item GenISSLib
\item autoconf,automake,make
\item chaine de compilation GNU pour ATMEL AVR32: gcc,newlib,objdump,readelf
\item crosstool-ng
\item carte et logiciel ATMEL studio: objectif dans les tests
\item mibench
\item plateforme Unisim-vp (site internet unisim-vp)
	
\end{itemize}

\subsection{Question posé par l'encadrant}

\begin{itemize}
\item choisir une réprésentation des registres.
\item implémenter le maximum d'instructions.
\item mettre en place un mecanisme de gestion des exceptions (interruptions externes, opcode illégal...) 
\item écrire un module de controleur d'interruption et un "timer" en systemC
\item réaliser des tests de validation (carte et benchmark) afin de vérifier le comportement du processeur et corriger les bugs
\end{itemize}