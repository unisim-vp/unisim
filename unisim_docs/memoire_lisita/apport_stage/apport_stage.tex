Ce stage m'a permis dans un premiers temps d'aborder une partie des étapes de la réalisation d'un simulateur. 
Cette expérience me permettra d'éviter de faire les mêmes erreurs et donc de gagner considérablement du temps si j'ai l'occasion de refaire 
un simulateur dans l'avenir . Au délà j'ai appris a utilisé les outils de debuggage et j'ai pu approndir mes connaissance sur le fonctionnement d'un processeur

Au cours de ces derniers mois j'ai également eu l'occasion de me familiariser avec un certain nombre d'outils qui me serviront constament dans ma carrière de 
développeur. En particulier les bases du langage C++, SystemC TLM 2.0, l'environnement Linux...