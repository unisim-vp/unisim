\subsection {Eléments de contexte}
Au cours de mon cursus à l'Ecole d'ingénieur Denis Diderot, j'ai eu l'occasion d'effectuer un stage de 6 mois entre la 2 ième et 3 ième année. 
J'ai donc choisi d'effectuer ce stage au C.E.A List afin d'avoir une expérience professionelle dans un organisme de recherche en informatique et de développer de nouvelles connaissances dans le domaine des systèmes embarqués.
Ma formation étant spécialisé dans la conception de logiciels embarqués et ayant suivi des cours d'architecture des processeurs, 
j'ai pensé qu'il serai intéressant d'approfondir mes connaissances dans le fonctionnenent des processeur afin d'avoir une vision plus complète de l'éxecution des programmes (niveau assembleur).
Gilles Mouchard et Reda Nouacer de l'équipe UNISIM recherchaient un étudiant afin d'effectuer un stage de 6 mois consistant à réaliser un simulateur de jeu d'instruction (ISS) d'un proceseur embarqué. 
\subsection{Contenue du document}
Ce document décrit le travail réalisé au cours de ce stage pendant la période du 10 mars au 31 août 2014 au sein du laboratoire LSL du CEA LIST.
Après une brève présentation du laboratoire et de l'équipe, j'établirai un  état de l'art sur les méthodes de conception et validation d'un simulateur suivie la présentation des outils utilisés pour le projet.
Puis je décrirai les étapes de réalisation et les résulats obtenues sur le processeur choisie par l'équipe.
J'évoquerai également les problèmes rencontrés et les moyens pour les résoudre. 
Enfin j'établirai un bilan sur les connaissances acquises durant ce stage concernant la simulation mais aussi plus généralement sur les méthodes de développement de logiciel.      