\subsection{Présentation de l'entreprise/équipe}
Le laboratoire ou j'ai effectué mon stage fait partie du Commissariat à l'énergie atomique et aux énergie alternatives (CEA) qui est un organisme de recherche scientifique
français dans les domaines de l'énergie, de la défense, des technologies de l'information, des sciences de la matière, des sciences de la vie et de la santé.
Le CEA est divisé en 5 directions opérationnelles dont la directions de la recherche technologique (DRT) appelé aujourd'hui CEA Tech dont le travail est essentiellement consacré
au développement des technologies génériques et au transfert de ces connaissances vers l'industrie.
L'un des ces 3 instituts, le LIST focalise ses recherches sur les systèmes numériques intelligents par des programmes en R\&D centrés sur les systèmes interactifs (intelligence ambiante),
les systèmes embarqués (architectures,ingénierie logicielle et systèmes), les capteurs et le traitement du signal(contrôle industriel, santé, sécurité, métrologie).
Le projet "UNISIM  Virtual Plateforme" est maintenue Gilles Mouchard et Reda Nouacer et a pour objectif de fournir des plateformes virtuelles 
et un environement pour faciliter le développement de nouvelles qui peuvent être utilisées dans des domaine variés tel que l'automobile,l'aéronautique,la sureté nucléaire...

\subsection{Problématique scientifique du stage}
Une plateforme virtuelle, souvent appelé simulateur, est un outil logiciel permettant d'immiter le comportement d'un système électronique (microprocesseurs et périphériques).
Ceci permet d'éxecuter des logiciels dessus avant que sa mise en oeuvre matériel ne soit disponible et permettre donc au industrie d'économiser dans le cout de production de leur produit. 
Suivant les besoins de repésentativité et le budget de développement, le simulateur peut être un simple jeu d'instruction ou un système complet (Bus,dispositifs d'E/S,capteur)
de tel sorte qu'il soient suffisament représentatif du système réel pour qu'un système d'exploitation complet puisse fonctionner dessus. 
Les simulateurs de la palteforme UNISIM sont modulaire, leurs composants en C/C++ et basées sur les normes de l'industries comme IEEE1666TM, OSCI SytemCTM et
OSCI TLM SystemCTM 2.0.

\subsection{Sujet du stage}
L'objectif de ce stage est de spécifier et réaliser un simulateur de jeu d'instruction (ISS) pour le processeur ATMEL AVR32UC3C afin de l'intégrer à l'environement UNISIM.