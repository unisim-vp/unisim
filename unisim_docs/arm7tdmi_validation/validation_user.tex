\section{User Level ISA Validation}
\label{sec:user_level}

When performing the validation of the ISA at the \emph{user level} we are only interested in the instructions that the processor can execute when running at the \emph{user mode}.
The \emph{user mode} is usually the most used mode on a processor (whenever the processor has multiple execution modes, which the ARM7TDMI has).
It is used to run the non-system dependent part of the applications.
Whenever an application needs to access the system (write on the screen, read the input from the keyboard, etc), it executes an special instruction (usually we call it \textit{system call}) which switches the processor to \emph{system mode} to perform the required operation.

The interest on performing a validation of \emph{user level} instructions resides on the difficulty to perform complete tests when using the \emph{system level} instructions (as we will see on Section~\ref{sec:system_level}).

\begin{figure}[!h]
	\begin{center}
		\includegraphics[width=\textwidth]{figures/ARM7TDMI_user_level.pdf}
	\end{center}
	\caption{ARM7TDMI schematic architecture to run applications at user level.}
	\label{fig:arm7tdmi_user_level}
\end{figure}

In order to validate only the \emph{user level} ISA, the simulator is run with a special configuration, see Figure~\ref{fig:arm7tdmi_user_level}.
To the \emph{ELF Loader} and the \emph{GDB Server} services a third service has been added: the \emph{Linux System Call Translator}.
This service takes control of the simulation when a system call is executed, providing the program being executed by the simulator a translation of the system calls, without having to execute any system level instruction.
More concreatelly, the \emph{Linux System Call Translator} provides a traduction to Linux programs, which means that programs running under this configuration need to be compiled for ARM Linux.

\begin{figure}[!h]
	\begin{center}
		\includegraphics[width=\textwidth]{figures/random_tests_methodology.pdf}
	\end{center}
	\caption{Random tests methodology.}
	\label{fig:random_tests_methodology}
\end{figure}

Figure~\ref{fig:random_tests_methodology} shows the steps taken to perform the validation of the user level ISA.
Once the simulator is written an application is run into booth the simulator and the target platform.
The output results of the simulator and the target platform are then compared.
If results match then it means that the simulator is correctly implemented.
If the results do not match we debug the simulator to fix it using the results differences, and the previous steps are repeated until the results match.

The following two subsections presents different validation tests performed under this configuration.
As target platform we used an ARM9 platform.
While it is not an ARM7TDMI platform, the ARM9 shares the same instructions at the user level, making it a valid target platform for the ARM7TDMI simulator at the user level.

\input{validation_user_random}

\input{validation_user_benchmark}

