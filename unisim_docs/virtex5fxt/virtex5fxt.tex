\section{Introduction}

\subsection{UNISIM}

UNISIM provides several virtual platforms and a framework to ease the development of new virtual platforms. A virtual platform is a software tool, often called simulator, that mimics the behavior of an electronic system so that software can run on it before silicon or FPGA implementation of that electronic system is available. The simulated electronic system can include lots of microprocessors and devices.
Depending on the needed representativeness and simulator development budget, a simulator can be as simple as an instruction set simulator as well as a full system simulator. A full system simulator, not only executes the microprocessor instruction set, like an instruction set simulator, but also simulates buses, I/O devices, sensors, actuators, so that real application workloads and operating systems can run on them. Most of UNISIM virtual platforms are full system simulators, which means that they are sufficiently representative of the real hardware that whole operating systems (e.g. Linux, VxWorks), unmodified software stacks (e.g. an AUTOSAR software stack), and industrial applications can run on them.
The UNISIM virtual platforms are modular: a simulator is the assembly of properly configured simulation components (e.g. CPU, RAM, buses). They are written in C/C++ and based on industry standards, like IEEE1666$^{TM}$, OSCI SystemC$^{TM}$ and OSCI SystemC$^{TM}$ TLM 2.0.

\noindent Some use cases of UNISIM virtual platforms are:
\begin{itemize}
\item Development of SystemC IPs (intellectual property) and new virtual platforms: UNISIM is an open development environment that comprise a SystemC module library, and a set of services (debugging, program loaders, …). It can be a fundation for the development of new SystemC IPs and new virtual virtual platforms.
\item Hybrid virtual platform: UNISIM/SystemC and an FPGA accelerator can be mixed to build some hybrid virtual platforms: for instance simulating processor cores within UNISIM/SystemC, and prototyping specialized IPs/devices within an FPGA accelerator. Hybridization allows using indifferently both UNISIM/SystemC IPs (on a standard host machine) and VHDL IPs (on an FPGA accelerator), but also speeding up simulation of large systems.
\item Non-intrusive debugging and testing of software: It means that, unlike on the real hardware, software can be debugged and tested without affecting either its functional and/or temporal behavior. With such virtual instrumentation, the user can seamlessly stop and resume execution of software, profile the software, inspect the system status, inject values on the sensors, modify the state of program variables and microprocessor/device registers, and then analyze the result without modifying the software.
\item Hardware/software integration: software stack can be debugged and tested within a representative hardware environment before the availability of either the FPGA prototypes or the real hardware. The software stack can be composed of low level software (e.g. drivers), of a real-time operating system, and of applications generated from high level models (Papyrus, Matlab Simulink, Statemate Stateflow, …)
\end{itemize}

Several open source virtual platforms for different targets (ARM, PowerPC, Star12X, and TMS320C3X) and different hosts (Linux, Windows, Mac OS X) are available for download here. These virtual platforms have been evaluated and used in various industry domains such as automotive, avionic, military, electrical equipments for medium tension, nuclear safety.

\subsection{Virtex 5 FXT Simulator}

The UNISIM Virtex 5 FXT is a virtual platform that tries to mimic a Xilinx ML507 development board that include a Xilinx Virtex 5 FXT (FPGA + PPC440).

\begin{figure}[!ht]
	\begin{center}
		\includegraphics[width=\textwidth]{virtex5fxt/fig_schematic.pdf}
	\end{center}
	\caption{UNISIM Virtex 5 FXT simulator schematic.}
	\label{fig:simulator_schematic}
\end{figure}

\noindent The UNISIM Virtex 5 FXT simulator, which Figure~\ref{fig:simulator_schematic} shows the schematic, implements the following:
\begin{itemize}
	\item PPC440 Embedded Processor Block (UG200):
		\begin{itemize}
			\item All the PPC440x5 \cite{PPC440x5} instruction set of a Xilinx Virtex 5 FXT
			\item Optional FPU that is similar to Xilinx FPU APU \cite{APU_FPU}
			\item PowerPC Book E MMU (shadow ITLB, shadow DTLB, unified TLB)
			\item Caches (instruction and data)
			\item Integrated timers (decrementer, fixed interval, watchdog)
			\item Exception handling mechanisms
			\item DCR (device control register) bus controller
			\item Crossbar
			\item MCI (Memory Controller Interface)
			\item MPLB (master processor local bus) interface
			\item SPLB0 and SPLB1 (slave processor local bus) interfaces
		\end{itemize}
	\item MPLB (master processor local bus) link
	\item 256 MB RAM on MCI
	\item 256 KB BRAM on MPLB
	\item XPS IntC interrupt controller (v2.01a) \cite{XPS_INTC} on MPLB
	\item XPS Timer/Counter (v1.02a) \cite{XPS_TIMER_COUNTER} on MPLB
	\item Spansion 256 Mbits (32 MB) S29GL256P off-chip flash memory \cite{S29GLP} on MPLB
	\item XPS UART Lite (v1.01a) \cite{XPS_UART_LITE} on MPLB
	\item Four XPS GPIO modules (v2.00a) \cite{XPS_GPIO} on MPLB connected to two LED boards and two DIP switch/push buttons boards
\end{itemize}

\noindent Several stub modules are currently integrated in the simulator to test the XPS Timer/Counter module:
\begin{itemize}
\item GenerateOut stubs connected on XPS Timer/Counter « GenerateOut » outputs
\item PWM stub connected on XPS Timer/Counter « PWM0 » output
\item CaptureTrigger stubs (optional randomized outputs) connected on XPS Timer/Counter « CaptureTrigger » inputs
\item SPLB stubs
\item DCR stubs
\end{itemize}

\noindent The simulator also supports the following features :
\begin{itemize}
\item Loading of ELF32, ELF64 \cite{ELF} and Motorola S19 (S-Record) \cite{S19} files
\item An integrated console debugger that supports debugging both at assembly level and source level (e.g. C source code). Source level debugging is only available for ELF binary files including DWARF v2 or v3 \cite{DWARF3} debugging informations
\item Support for the GDB \cite{GDB} serial remote protocol over TCP/IP. That allows debugging a binary running into the simulator at assembly and/or source level using the GNU debugger (aka GDB)
\item Support for a telnet console over the XPS UART Lite
\end{itemize}

Table~\ref{table:memory_mapping} shows the simulator memory mapping.

\begin{table}[!ht]
	\begin{center}
		\begin{tabular}{|l|l|l|}
		\hline
		\multicolumn{1}{|c|}{\textbf{Address range}} & \multicolumn{1}{|c|}{\textbf{Component name}} & \multicolumn{1}{|c|}{\textbf{Hardware device}}\\
		\hline
		\texttt{0x00000000-0x0003ffff} & \texttt{ram} & 256 MB RAM\\
		\hline
		\texttt{0x81400000-0x8140ffff} & \texttt{gpio-leds-8bit} & XPS GPIO\\
		\hline
		\texttt{0x81420000-0x8142ffff} & \texttt{gpio-5-leds-positions} & XPS GPIO\\
		\hline
		\texttt{0x81440000-0x8144ffff} & \texttt{gpio-push-buttons-5bit} & XPS GPIO\\
		\hline
		\texttt{0x81460000-0x8146ffff} & \texttt{gpio-dip-switches-8bit} & XPS GPIO\\
		\hline
		\texttt{0x81800000-0x8180ffff} & \texttt{intc} & XPS IntC\\
		\hline
		\texttt{0x83c00000-0x83c0ffff} & \texttt{timer} & XPS Timer/Counter\\
		\hline
		\texttt{0x84000000-0x8400ffff} & \texttt{uart-lite} & XPS UART Lite\\
		\hline
		\texttt{0xfc000000-0xfdffffff} & \texttt{flash} & S29GL256P\\
		\hline
		\texttt{0xfffc0000-0xffffffff} & \texttt{bram} & 256 KB BRAM\\
		\hline
		\end{tabular}
	\end{center}
	\caption{Simulator memory mapping.}
	\label{table:memory_mapping}
\end{table}

\noindent The 1.0 release of the simulator is composed of:
\begin{itemize}
\item the simulator source code: unisim-virtex5fxt-1.0.tar.gz
\item the present documentation
\end{itemize}

Please follow the installation instructions in Section~\ref{building_simulator} to get the simulator building on your own building environment.
Section~\ref{getting_started} presents the basics for using the simulator.
Section~\ref{examples_of_use} presents some examples of use of the simulator.
Appendix~\ref{techref} contains the technical references (generated) of the simulator.

\section{Building the simulator}
\label{building_simulator}

\subsection{Requirements}

\noindent The following tools or libraries must be installed:
\begin{itemize}
\item GNU C++ compiler (aka. g++)
\item GNU Flex (aka. flex)
\item GNU Bison (aka. bison)
\item Standard GNU C++ library (aka. libstdc++-dev that comes with g++)
\item Boost (aka. libboost-dev)
\item Editline/Libedit (aka. libedit-dev)
\item zlib (aka. zlib1g-dev)
\item libxml2 (aka. libxml2-dev)
\end{itemize}

\subsection{Installing SystemC 2.2.0 and SystemC TLM 2.0.1}
\subsubsection{Download the source code}
Register at \url{http://www.systemc.org} and then download systemc-2.2.0.tgz and TLM-2.0.1.tgz from the OSCI standards download page.

\subsubsection{Uncompressing the source code tarballs}
\begin{script}
   $ tar zxvf systemc-2.2.0.tgz
   $ tar zxvf TLM-2.0.1.tgz
\end{script}
  
This will uncompress the source of SystemC in directory systemc-2.2.0 and the source TLM 2.0.1 in directory TLM-2009-07-15 
\subsubsection{Patching the source code (recommended for g++ $\geq$ 4.1)}
Apply the following patch to SystemC 2.2.0: patch-systemc-2.2.0.gz 

\begin{script}
   $ cd systemc-2.2.0
   $ gzip -dc patch-systemc-2.2.0.gz | patch -p1
\end{script}

\subsubsection{Configuring SystemC}
To configure, compile, and install SystemC in your home directory, do the following at the command prompt: 

\begin{script}
   $ cd systemc-2.2.0
   $ mkdir objdir
   $ cd objdir
   $ mkdir ${HOME}/systemc
   $ ../configure --prefix=${HOME}/systemc
\end{script}

Note : if configure needs automake or autoconf, please install these tools, then rebuild SystemC configure script by running these commands before actually running the configure script:

\begin{script}
   $ cd systemc-2.2.0
   $ aclocal
   $ automake
\end{script}

\subsubsection{Compiling and installing SystemC}
To handle threads, SystemC relies on QuickThreads, a fast implementation of user's threads. QuickThreads speeds-up threads switching compared to the slower kernel POSIX threads and thus considerably improves overall simulation performance. To compile SystemC with the built-in QuickThreads (recommended), do the following at the command prompt: 
\begin{script}
   $ make
   $ make install
\end{script}

However, if you intend to instrument your simulator (e.g. with valgrind) to debug the simulator memory leaks, bad memory accesses, pointers, and uninitialized memory reads, you should use the slower kernel POSIX threads. To compile SystemC with the kernel POSIX threads, do the following at the command prompt: 
\begin{script}
   $ make pthreads
   $ make install
\end{script}

\subsection{Building the UNISIM Virtex 5 FXT simulator}
\subsubsection{Uncompressing the source code tarball}
\begin{script}
   $ tar zxvf unisim-virtex5fxt-1.0.tar.gz
\end{script}

\subsubsection{Configuring the simulator building process}
\begin{script}
  $ cd unisim-virtex5fxt-1.0
  $ ./configure \
          --with-systemc=${HOME}/systemc \
          --with-tlm20=${HOME}/TLM-2009-07-15 CXXFLAGS='-O3 -g3 -Wall'
\end{script}

\subsubsection{Compiling the simulator}
\begin{script}
   $ make
\end{script}

\noindent The simulator binaries are in \texttt{virtex5fxt/bin} subdirectory. The simulators comes in four flavors in:
\begin{itemize}
\item \texttt{unisim-virtex5fxt-1.0}: release simulator without FPU
\item \texttt{unisim-virtex5fxt-wfpu-1.0}: release simulator with FPU
\item \texttt{unisim-virtex5fxt-debug-1.0}: developement simulator
\item \texttt{unisim-virtex5fxt-wfpu-debug-1.0}: developement simulator with FPU
\end{itemize}

\section{Getting started}
\label{getting_started}

In this section, we present the basics for using the simulator. More details are available in Appendix~\ref{techref}.

\subsection{Run-time configuration}

The simulator has a parametrization system that allows configuring individual simulator components, that is the hardware components, and the services.
\noindent The simulator stores its configuration (a set of parameters) in a XML configuration file. 
\newline\\
\noindent The simulator can provide the user with a default XML configuration file with option \texttt{-g}:
\begin{script}
$ unisim-virtex5fxt-wfpu-1.0 -g default_sim_config.xml
\end{script}

The XML configuration file can be edited, and then reloaded by the simulator with option \texttt{-c}:
\begin{script}
$ unisim-virtex5fxt-wfpu-1.0 -c sim_config.xml
\end{script}

\noindent The user can also simply set the value of an individual parameter with option \texttt{-s}:
\begin{script}
$ unisim-virtex5fxt-wfpu-1.0 -s enable-inline-debugger=true
\end{script}

\noindent The simulator can prints the list of parameter set on the console with option \texttt{-l}:
\begin{script}
$ unisim-virtex5fxt-wfpu-1.0 -l
\end{script}

\noindent In general, each simulator components have log messages that can be switched on or off using a parameter named \texttt{verbose} (or approching):
\begin{script}
unisim-virtex5fxt-wfpu-1.0 -s cpu.verbose-exception=true
\end{script}

\noindent The simulator accepts any combination of the above options. For example, you can combine these options to change the value of one or more parameters in an XML configuration file:
\begin{script}
unisim-virtex5fxt-wfpu-1.0 -c sim_config.xml -s enable-inline-debugger=true -g sim_config.xml
\end{script}

\subsection{Loading binaries}
\label{loading_binaries}

The simulators has a multi-format loader service that can detect the format of binaries and accordingly instantiate the right loader.
The user can set the list of binary files to load in Parameter loader.filename, each filenames being separated by a comma:
\begin{script}
$ unisim-virtex5fxt-wfpu-1.0 -s loader.filename='boot.elf,vmlinux,device_tree.dtb,initrd.img'
\end{script}
In the hypothetic case where the multi-format loader would wrongly guess the format of a binary, the user tells the loader what is the format of the binary file:
\begin{script}
$ unisim-virtex5fxt-wfpu-1.0 \
   -s loader.filename='boot.elf:elf32,vmlinux:elf32,device_tree.dtb:raw,initrd.img:raw'
\end{script}

\noindent If, for any reason (virtual memory, self relocation), the simulator must load a binary file to an address that is not the final address indicated in the binary file itself, the user tells the loader to override (when possible) the base address:
\begin{script}
$ unisim-virtex5fxt-wfpu-1.0 \
     -s loader.filename='boot.elf,vmlinux,device_tree.dtb,initrd.img' \
     -s loader.file1.base-addr=0 \
     -s loader.file1.force-base-addr=true
\end{script}

\subsection{Serial console}

The simulator comes with a UART Lite module on MPLB that the target application can use as a serial console.
The simulator telnet service, that is actually a server for the telnet protocol, manages communication between the real network and the virtual UART Lite module.
The combination of the UART Lite module, the telnet service and a serial console aware target application, enables using a real telnet client (running on the host machine or any machine on the internet) as virtual serial console.

To enable the serial console, do the following at the command prompt:
\begin{script}
$ unisim-virtex5fxt-wfpu-1.0 \
     -s enable-telnet=true \
     -s telnet.telnet-tcp-port=1234
\end{script}

During initialization, the simulator waits for a telnet client connection on the telnet port.
In another console, connect a telnet client to the simulator:
\begin{script}
$ telnet localhost 1234
\end{script}

At this point, the user can interact with the target application using the telnet client.

\subsection{Using the builtin debugger}

The simulator has an integrated debugger for debugging the target application an a non-intrusive manner.

To enable the inline-debugger, do the following at the command prompt:
\begin{script}
$ unisim-virtex5fxt-wfpu-1.0 -s enable-inline-debugger=true
\end{script}

The user can enter classical debug commands from the debugger, such as putting breakpoints, watchpoints, stepping instructions, disassembling, dumping memory regions, etc.

To obtain help within the debugger, uses the debugger command \texttt{help}:
\begin{script}
inline-debugger> help
\end{script}

\subsection{Using a GNU crosstool chain}

\subsubsection{Building a GNU crosstool chain}
\label{building_crosstool_chain}

A crosstool chain is a tool chain to create and manipulate binary programs for a target architecture and operating system (e.g. PowerPC/Linux) that are different from host architecture and operating system (e.g. x86/Linux) that runs the tool chain.
Building a GNU crosstool chain from scratch is a very tedious task.
The process consists of a quite high number of undocumented steps, and mostly relies on the users experience.
Fortunately a community sharing its experience, the crossgcc mailing list, actively supports a tool, crosstool-NG, that considerably simplifies the process of building a crosstool chain.
Xilinx Virtex-5 FXT embedded processor is a PPC440x5 and Xilinx Platform Studio provides a PowerPC hardware FPU.
Hence, from now, we will focus our effort on creating a crosstool chain with GNU GCC 4.4.6 (\texttt{gcc} and \texttt{g++}), GNU C library 2.9 (\texttt{glibc}), GNU binutils 2.19.1 (\texttt{ld}, \texttt{objdump}, and \texttt{readelf}), and GNU GDB 6.8 (\texttt{gdb}) for target \texttt{powerpc-440fp-linux-gnu}.

\begin{enumerate}
\item Get crosstool-NG: Download the crosstool-NG archive (e.g crosstool-ng-1.11.3.tar.bz2) from this page: \url{http://crosstool-ng.org}
\item Uncompress crosstool-NG archive:
\begin{script}
$ tar jxvf crosstool-ng-1.11.3.tar.bz2
\end{script}
\item Configure crosstool-NG build:
\begin{script}
$ cd crosstool-ng-1.5.3
$ ./configure --local
\end{script}  
\item Compile crosstool-NG
\begin{script}
$ make
\end{script}
\item Configure the crosstool chain build as shown on Figure~\ref{fig:ctng_configuration}:
\begin{script}
$ ./ct-ng menuconfig
\end{script}
\item Build the crosstool chain (this may take tens of minutes):
\begin{script}
$ ./ct-ng build
\end{script}
\item You crosstool chain is ready in \texttt{\$\{HOME\}/crosstool/powerpc-440fp-linux-gnu}
\end{enumerate}

\begin{figure}[!ht]
\begin{center}
\begin{script}
Paths and misc options  --->
(${HOME}/crosstool/powerpc-440fp-linux-gnu) Prefix directory

Target options  --->
        Target Architecture (powerpc)  --->
    (440fp) Emit assembly for CPU
    (440fp) Tune for CPU
        Floating point: (hardware (FPU))  --->

Toolchain options  --->
    (440fp) Tuple's vendor string

Operating System  --->
    Target OS (linux)  --->
    Linux kernel version (2.6.35.13 (longterm))  --->

Binary utilities  --->
    binutils version (2.19.1)  --->

    [*] binutils libraries for the target
    [*]   libiberty (NEW)
    [*]   libbfd (NEW)

C compiler  --->
    gcc version (4.4.6)  --->
    [*] C++
    [ ] Link libstdc++ statically into the gcc binary

C-library  --->
    C library (glibc)  --->
    glibc version (2.9)  --->
    (-U_FORTIFY_SOURCE) extra target CFLAGS
    [*] Use the ports addon

Debug facilities  --->
    [*] gdb  --->

Companion libraries  --->
    MPFR version (2.4.2)  --->
    CLooG/ppl version (0.15.9)  --->
\end{script}
\end{center}
\caption{crosstool chain build configuration (changes from default only) for Crosstool-NG 1.11.3.}
\label{fig:ctng_configuration}
\end{figure}

\subsubsection{Using cross-GCC}

A GCC cross-compiler together with a LD cross-linker can be used to create binaries for the target machine from the host machine:
\begin{script}
$ ${HOME}/crosstool/powerpc-440fp-linux-gnu/bin/powerpc-440fp-linux-gnu-gcc \
                  nodefaultlibs -nostdlib -mcpu=440fp -c hello.c -o hello.o
$ ${HOME}/crosstool/powerpc-440fp-linux-gnu/bin/powerpc-440fp-linux-gnu-ld hello.lds -o hello.elf
\end{script}

\noindent Simulator can directly loads such binary files, see Section~\ref{loading_binaries}.

\subsubsection{Using cross-GDB}

GNU GDB client can debug applications running on a remote/local host on the network.
The application is run under the control of program \texttt{gdbserver} while program \texttt{gdb} only manages interactions with the user.
Program \texttt{gdbserver} and \texttt{gdb} communicates over the TCP/IP network using a documented serial remote protocol.
The simulator has a GDB server service that implements the GDB serial remote protocol, so that the simulator can acts as Program \texttt{gdbserver} from the GDB client point of view.

To enable the simulator GDB server, do the following at the command prompt:
\begin{script}
$ unisim-virtex5fxt-wfpu-1.0 -s enable-gdb-server=true -s gdb-server.tcp-port=1234
\end{script}

During initialization, the simulator waits for a GDB client connection on the GDB server TCP/IP port.
In another console, connect a GDB client to the simulator:
\begin{script}
$ powerpc-440fp-linux-gnu-gdb boot.elf
(gdb) target remote :1234
\end{script}

\section{Examples of use}
\label{examples_of_use}

In this section, we present some examples of use for the simulator.
We provide you with scripts and makefiles to build in an automatic manner all the examples.

\subsection{Building the examples}

The top level \texttt{build.sh} scripts take at least one argument, and an optional second argument.
The first argument is either \texttt{all} or \texttt{clean}.
The second optional argument is the path to a crosstool chain for target \texttt{powerpc-440fp-linux-gnu}, that is a GCC cross-compiler and its standard C library.
If that second argument is ommitted, the \texttt{build.sh} script will build the crosstool chain itself.
To launch the build of the crosstool chain and every binaries for target \texttt{powerpc-440fp-linux-gnu}, do the following at the command prompt:
\begin{script}
$ ./build.sh all
\end{script}

\subsection{Basic system level tests}

Table~\ref{table:basic_system_test_summary} shows a summary of basic system tests.

\begin{table}[!ht]
	\begin{center}
		\small
		\begin{tabular}{|l|l|l|l|l|l|}
		\hline
		\multicolumn{1}{|c|}{\textbf{Test Name}} &
		\multicolumn{1}{|c|}{\textbf{Directory}} &
		\multicolumn{1}{|c|}{\VROT{\textbf{IntC}}} &
		\multicolumn{1}{|c|}{\VROT{\textbf{Timer}}} &
		\multicolumn{1}{|c|}{\VROT{\textbf{UART Lite}}} &
		\multicolumn{1}{|c|}{\VROT{\textbf{NOR Flash}}}\\
		\hline
		Echo &
		\texttt{\footnotesize xps\_uart\_lite/echo} &
		\X &
		   &
		\X &
		   \\
		\hline
		Poll &
		\texttt{\footnotesize xps\_timer/poll} &
		   &
		\X &
		\X &
		   \\
		\hline
		Gen &
		\texttt{\footnotesize xps\_timer/gen} &
		\X &
		\X &
		\X &
		   \\
		\hline
		Cap &
		\texttt{\footnotesize xps\_timer/cap} &
		\X &
		\X &
		\X &
		   \\
		\hline
		PWM &
		\texttt{\footnotesize xps\_timer/pwm} &
		\X &
		\X &
		\X &
		   \\
		\hline
		Autoselect &
		\texttt{\footnotesize s29gl256p/autoselect} &
		\X &
		   &
		\X &
		\X \\
		\hline
		CFI query &
		\texttt{\footnotesize s29gl256p/cfi\_query} &
		\X &
		   &
		\X &
		\X \\
		\hline
		Chip erase &
		\texttt{\footnotesize s29gl256p/chip\_erase} &
		\X &
		   &
		\X &
		\X \\
		\hline
		Sector erase &
		\texttt{\footnotesize s29gl256p/sector\_erase} &
		\X &
		   &
		\X &
		\X \\
		\hline
		Single word programming &
		\texttt{\footnotesize s29gl256p/single\_word\_programming} &
		\X &
		   &
		\X &
		\X \\
		\hline
		Unlock bypass chip erase &
		\texttt{\footnotesize s29gl256p/unlock\_bypass\_chip\_erase} &
		\X &
		   &
		\X &
		\X \\
		\hline
		Unlock bypass sector erase &
		\texttt{\footnotesize s29gl256p/unlock\_bypass\_sector\_erase} &
		\X &
		   &
		\X &
		\X \\
		\hline
		Unlock bypass word programming &
		\texttt{\footnotesize s29gl256p/unlock\_bypass\_word\_programming} &
		\X &
		   &
		\X &
		\X \\
		\hline
		Write buffer programming &
		\texttt{\footnotesize s29gl256p/write\_buffer\_programming} &
		\X &
		   &
		\X &
		\X \\
		\hline
		\end{tabular}
	\end{center}
	\caption{Summary of basic system level tests.}
	\label{table:basic_system_test_summary}
\end{table}

\subsubsection{Echo}

The test reads characters from the serial console.
It prints the read characters on the serial console.

\begin{enumerate}
\item At the command prompt, do the following:
\begin{script}
$ cd basic_system/xps_uart_lite/echo
$ unisim-virtex5fxt-wfpu-1.0 -c sim_config.xml
\end{script}
\item In another console, do the following at command prompt:
\begin{script}
$ telnet localhost 1234
\end{script}
\item Press some keys in the telnet console. The corresponding characters should prints in the telnet console as you type
\end{enumerate}

\subsubsection{Poll}

The test polls the timer/counter \#0.
It prints some of the sampled values on the serial console.

\noindent To run the test within the simulator:
\begin{enumerate}
\item Enter Directory \texttt{basic\_system}, start the simulator:
\begin{script}
$ cd basic_system
$ unisim-virtex5fxt-wfpu-1.0 -c s29gl256p/poll/sim_config.xml
\end{script}
\item In another console, connect a telnet client to the simulator:
\begin{script}
$ telnet localhost 1234
\end{script}
\item The output of the test on the telnet console should be:
\begin{script}
Time is 6 us
Time is 116 us
Time is 227 us
Time is 338 us
Time is 449 us
Time is 560 us
Time is 671 us
Time is 782 us
Time is 893 us
Time is 1004 us
Time is 1115 us
Time is 1226 us
Time is 1337 us
Time is 1448 us
Time is 1559 us
Time is 1670 us
Time is 1782 us
Time is 1893 us
Time is 2004 us
Time is 2115 us
Time is 2226 us
Time is 2337 us
Time is 2448 us
Time is 2559 us
Time is 2670 us
Time is 2782 us
Time is 2894 us
Time is 3005 us
Time is 3116 us
Time is 3228 us
Time is 3340 us
Time is 3451 us
Time is 3562 us
Time is 3673 us
Time is 3785 us
Time is 3897 us
Time is 4008 us
Time is 4119 us                                                                                                                                                                                      
Time is 4230 us                                                                                                                                                                                      
Time is 4342 us                                                                                                                                                                                      
Test is finished                                                                                                                                                                                     \end{script}
\end{enumerate}

\subsubsection{Gen}

The test uses the timer generate mode with interrupt generation every 100 $\mu$s.
It prints the tick of timer on the serial console.

\noindent To run the test within the simulator:
\begin{enumerate}
\item Enter Directory \texttt{basic\_system}, start the simulator:
\begin{script}
$ cd basic_system
$ unisim-virtex5fxt-wfpu-1.0 -c s29gl256p/gen/sim_config.xml
\end{script}
\item In another console, connect a telnet client to the simulator:
\begin{script}
$ telnet localhost 1234
\end{script}
\item The output of the test on the telnet console should be:
\begin{script}
Tick: 100 us
Tick: 200 us
Tick: 300 us
Tick: 400 us
Tick: 500 us
Tick: 600 us
Tick: 700 us
Tick: 800 us
Tick: 900 us
Tick: 1000 us
Tick: 1100 us
Tick: 1200 us
Tick: 1300 us
Tick: 1400 us
Tick: 1500 us
Tick: 1600 us
....

....
Tick: 8300 us
Tick: 8400 us
Tick: 8500 us
Tick: 8600 us
Tick: 8700 us
Tick: 8800 us
Tick: 8900 us
Tick: 9000 us
Tick: 9100 us
Tick: 9200 us
Tick: 9300 us
Tick: 9400 us
Tick: 9500 us
Tick: 9600 us
Tick: 9700 us
Tick: 9800 us
Tick: 9900 us
Tick: 10000 us
Test is finished
\end{script}
\end{enumerate}

\subsubsection{Cap}

The test uses the timer capture mode (randomized input between 1 $\mu$s and 3.995 $\mu$s).
It prints the captured time stamp on the serial console.

\noindent To run the test within the simulator:
\begin{enumerate}
\item Enter Directory \texttt{basic\_system}, start the simulator:
\begin{script}
$ cd basic_system
$ unisim-virtex5fxt-wfpu-1.0 -c s29gl256p/cap/sim_config.xml
\end{script}
\item In another console, connect a telnet client to the simulator:
\begin{script}
$ telnet localhost 1234
\end{script}
\item The output of the test on the telnet console should be:
\begin{script}
Last capture is 360 ns
Last capture is 23090 ns
Last capture is 45930 ns
Last capture is 69285 ns
Last capture is 93040 ns
Last capture is 117545 ns
Last capture is 143545 ns
Last capture is 169325 ns
Last capture is 195480 ns
Last capture is 220590 ns
Last capture is 244635 ns
Last capture is 270840 ns
Last capture is 294800 ns
Last capture is 319990 ns
Last capture is 348385 ns
Last capture is 370810 ns
Last capture is 397785 ns
Last capture is 422900 ns
Last capture is 449070 ns
Last capture is 473000 ns
Last capture is 497565 ns
Last capture is 521180 ns
Last capture is 547700 ns
Last capture is 572140 ns
Last capture is 596700 ns
Last capture is 622295 ns
Last capture is 647230 ns
Last capture is 671745 ns
Last capture is 697080 ns
Last capture is 722645 ns
Last capture is 746020 ns
Last capture is 771475 ns
Last capture is 797765 ns
Last capture is 818050 ns
Last capture is 847275 ns
Last capture is 872910 ns
Last capture is 897765 ns
Last capture is 922820 ns
Last capture is 947330 ns
Last capture is 972370 ns
Last capture is 998170 ns
Last capture is 1024420 ns
Test is finished
\end{script}
\end{enumerate}

\subsubsection{PWM}

The test uses the timer in PWM (Pulse Width Modulation) mode with a period of 2 $\mu$s and a duty cycle of 300 ns.
\noindent To run the test within the simulator:
\begin{enumerate}

\item Enter Directory \texttt{basic\_system}, start the simulator and filter PWM0 activity:
\begin{script}
$ cd basic_system
$ unisim-virtex5fxt-wfpu-1.0 -c s29gl256p/pwm/sim_config.xml \
                             -s timer.verbose=true | grep 'PWM0 signal'
\end{script}
\item In another console, connect a telnet client to the simulator:
\begin{script}
$ telnet localhost 1234
\end{script}
\item The output of the test on the telnet console should be:
\begin{script}
You should observe a PWM output with a period of 2000 ns, and a duty cycle of 300 ns.
\end{script}
\begin{script}
timer: 7815 ns: PWM0 signal goes high
timer: 8115 ns: PWM0 signal goes low
timer: 9815 ns: PWM0 signal goes high
timer: 10115 ns: PWM0 signal goes low
timer: 11815 ns: PWM0 signal goes high
timer: 12115 ns: PWM0 signal goes low
timer: 13815 ns: PWM0 signal goes high
timer: 14115 ns: PWM0 signal goes low
timer: 15815 ns: PWM0 signal goes high
timer: 16115 ns: PWM0 signal goes low
timer: 17815 ns: PWM0 signal goes high
timer: 18115 ns: PWM0 signal goes low
timer: 19815 ns: PWM0 signal goes high
timer: 20115 ns: PWM0 signal goes low
timer: 21815 ns: PWM0 signal goes high
timer: 22115 ns: PWM0 signal goes low
timer: 23815 ns: PWM0 signal goes high
timer: 24115 ns: PWM0 signal goes low
timer: 25815 ns: PWM0 signal goes high
timer: 26115 ns: PWM0 signal goes low
timer: 27815 ns: PWM0 signal goes high
timer: 28115 ns: PWM0 signal goes low
timer: 29815 ns: PWM0 signal goes high
timer: 30115 ns: PWM0 signal goes low
timer: 31815 ns: PWM0 signal goes high
timer: 32115 ns: PWM0 signal goes low
timer: 33815 ns: PWM0 signal goes high
timer: 34115 ns: PWM0 signal goes low
....
\end{script}
\end{enumerate}

\subsubsection{Autoselect}

The test puts the S29GL256P NOR Flash chip in autoselect mode.
It reads the manufacturer and device IDs.
It prints the two IDs on the serial console.
It prints also protection status of each sector.

\noindent To run the test within the simulator:
\begin{enumerate}
\item Enter Directory \texttt{basic\_system}, and start the simulator:
\begin{script}
$ cd basic_system
$ unisim-virtex5fxt-wfpu-1.0 -c s29gl256p/autoselect/sim_config.xml
\end{script}
\item In another console, connect a telnet client to the simulator:
\begin{script}
$ telnet localhost 1234
\end{script}
\item The output of the test on the telnet console should be:
\begin{script}
Write unlock cycle 1
Write unlock cycle 2
Write autoselect command
Manufacturer ID: 0x0001
Device ID word 1: 0x227e
Device ID word 2: 0x2222
Device ID word 3: 0x2201
Protection status (U=unprotected/P=protected): UUUUUUUUUUUUUUUUUUUUUUUUUUUUUUUUU
UUUUUUUUUUUUUUUUUUUUUUUUUUUUUUUUUUUUUUUUUUUUUUUUUUUUUUUUUUUUUUUUUUUUUUUUUUUUUUUU
UUUUUUUUUUUUUUUUUUUUUUUUUUUUUUUUUUUUUUUUUUUUUUUUUUUUUUUUUUUUUUUUUUUUUUUUUUUUUUUU
UUUUUUUUUUUUUUUUUUUUUUUUUUUUUUUUUUUUUUUUUUUUUUUUUUUUUUUUUUUUUUU
Exit autoselect
Test is finished
\end{script}

\end{enumerate}

\subsubsection{CFI query}

The test puts the S29GL256P NOR Flash chip in CFI (Common Flash Interface) query mode.
It queries the unique ASCII string "QRY".
For each of the three characters, it prints on the serial console whether they match what is expected ("QRY").

\noindent To run the test within the simulator:
\begin{enumerate}
\item Enter Directory \texttt{basic\_system}, and start the simulator:
\begin{script}
$ cd basic_system
$ unisim-virtex5fxt-wfpu-1.0 -c s29gl256p/cfi_query/sim_config.xml
\end{script}
\item In another console, connect a telnet client to the simulator:
\begin{script}
$ telnet localhost 1234
\end{script}
\item The output of the test on the telnet console should be:
\begin{script}
Write CFI entry command
Read Q: OK
Read R: OK
Read Y: OK
Exit CFI query mode (Write reset command)
Test is finished
\end{script}

\end{enumerate}


\subsubsection{Chip erase}

The test erases the S29GL256P NOR Flash chip.
It checks that all sectors have been erased.

\noindent To run the test within the simulator:
\begin{enumerate}
\item Enter Directory \texttt{basic\_system}, and start the simulator:
\begin{script}
$ cd basic_system
$ unisim-virtex5fxt-wfpu-1.0 -c s29gl256p/chip_erase/sim_config.xml
\end{script}
\item In another console, connect a telnet client to the simulator:
\begin{script}
$ telnet localhost 1234
\end{script}
\item The output of the test on the telnet console should be:
\begin{script}
Write unlock cycle 1
Write unlock cycle 2
Write setup command
Write additional unlock cycle 1
Write additional unlock cycle 2
Write chip erase command
Verifying sectors..............................................................
...............................................................................
...............................................................................
....................................
Done
Test is finished
\end{script}

\end{enumerate}

\subsubsection{Sector erase}

The test erases the S29GL256P NOR Flash chip, one sector at a time.
It checks that all sectors have been erased.

\noindent To run the test within the simulator:
\begin{enumerate}
\item Enter Directory \texttt{basic\_system}, and start the simulator:
\begin{script}
$ cd basic_system
$ unisim-virtex5fxt-wfpu-1.0 -c s29gl256p/sector_erase/sim_config.xml
\end{script}
\item In another console, connect a telnet client to the simulator:
\begin{script}
$ telnet localhost 1234
\end{script}
\item The output of the test on the telnet console should be:
\begin{script}
Erasing sector #0 of 256
Write unlock cycle 1
Write unlock cycle 2
Write setup command
Write additional unlock cycle 1
Write additional unlock cycle 2
Write sector erase command
Verifying sector: OK
....

....
Erasing sector #255 of 256
Write unlock cycle 1
Write unlock cycle 2
Write setup command
Write additional unlock cycle 1
Write additional unlock cycle 2
Write sector erase command
Verifying sector: OK
\end{script}

\end{enumerate}

\subsubsection{Single word programming}

The test program a word on the S29GL256P NOR Flash chip.
It verifies that word has been effectively programmed.

\noindent To run the test within the simulator:
\begin{enumerate}
\item Enter Directory \texttt{basic\_system}, and start the simulator:
\begin{script}
$ cd basic_system
$ unisim-virtex5fxt-wfpu-1.0 -c s29gl256p/single_word_programming/sim_config.xml
\end{script}
\item In another console, connect a telnet client to the simulator:
\begin{script}
$ telnet localhost 1234
\end{script}
\item The output of the test on the telnet console should be:
\begin{script}
Write unlock cycle 1
Write unlock cycle 2
Write program setup command
Write data to be programmed
Verify programmed data: OK
\end{script}

\end{enumerate}

\subsubsection{Unlock bypass chip erase}

The test erases the S29GL256P NOR Flash chip four times.
It checks that all sectors have been erased.

\noindent To run the test within the simulator:
\begin{enumerate}
\item Enter Directory \texttt{basic\_system}, and start the simulator:
\begin{script}
$ cd basic_system
$ unisim-virtex5fxt-wfpu-1.0 -c s29gl256p/unlock_bypass_chip_erase/sim_config.xml
\end{script}
\item In another console, connect a telnet client to the simulator:
\begin{script}
$ telnet localhost 1234
\end{script}
\item The output of the test on the telnet console should be:
\begin{script}
Write unlock cycle 1
Write unlock cycle 2
Write unlock bypass command
=== Pass #0 ===
Write setup command
Write chip erase command
Verifying sectors...............................................................
................................................................................
................................................................................
.................................
Done
=== Pass #1 ===
Write setup command
Write chip erase command
Verifying sectors...............................................................
................................................................................
................................................................................
.................................
Done
=== Pass #2 ===
Write setup command
Write chip erase command
Verifying sectors...............................................................
................................................................................
................................................................................
.................................
Done
=== Pass #3 ===
Write setup command
Write chip erase command
Verifying sectors...............................................................
................................................................................
................................................................................
.................................
Done
Exit unlock bypass
Test is finished
\end{script}

\end{enumerate}

\subsubsection{Unlock bypass sector erase}

The test erases the S29GL256P NOR Flash chip, one sector at a time.
It checks that all sectors have been erased.

\noindent To run the test within the simulator:
\begin{enumerate}
\item Enter Directory \texttt{basic\_system}, and start the simulator:
\begin{script}
$ cd basic_system
$ unisim-virtex5fxt-wfpu-1.0 -c s29gl256p/unlock_bypass_sector_erase/sim_config.xml
\end{script}
\item In another console, connect a telnet client to the simulator:
\begin{script}
$ telnet localhost 1234
\end{script}
\item The output of the test on the telnet console should be:
\begin{script}
Write unlock cycle 1
Write unlock cycle 2
Write unlock bypass command
Erasing sector #0 of 256
Write setup command
Write sector erase command for sector #0
Verifying sector: OK
....

....
Erasing sector #255 of 256
Write setup command
Write sector erase command for sector #255
Verifying sector: OK
Exit unlock bypass
Test is finished
\end{script}

\end{enumerate}

\subsubsection{Unlock bypass word programming}

The test programs four words in the S29GL256P NOR Flash chip, one word at a time.
It verifies that the four words have beed effectively programmed.

\noindent To run the test within the simulator:
\begin{enumerate}
\item Enter Directory \texttt{basic\_system}, and start the simulator:
\begin{script}
$ cd basic_system
$ unisim-virtex5fxt-wfpu-1.0 -c s29gl256p/unlock_bypass_sector_erase/sim_config.xml
\end{script}
\item In another console, connect a telnet client to the simulator:
\begin{script}
$ telnet localhost 1234
\end{script}
\item The output of the test on the telnet console should be:
\begin{script}
Write unlock cycle 1
Write unlock cycle 2
Write unlock bypass command
Write program setup command
Write data #0 to be programmed (0x4321)
Write program setup command
Write data #1 to be programmed (0x8765)
Write program setup command
Write data #2 to be programmed (0x9087)
Write program setup command
Write data #3 to be programmed (0x3852)
Re-reading and Verifying programmed data: OK (0x4321)
Re-reading and Verifying programmed data: OK (0x8765)
Re-reading and Verifying programmed data: OK (0x9087)
Re-reading and Verifying programmed data: OK (0x3852)
Exit unlock bypass
Test is finished
\end{script}

\end{enumerate}

\subsubsection{Write buffer programming}

The test programs four words in the S29GL256P NOR Flash chip, using the write buffer.
It verifies that the four words have beed effectively programmed.

\noindent To run the test within the simulator:
\begin{enumerate}
\item Enter Directory \texttt{basic\_system}, and start the simulator:
\begin{script}
$ cd basic_system
$ unisim-virtex5fxt-wfpu-1.0 -c s29gl256p/unlock_bypass_sector_erase/sim_config.xml
\end{script}
\item In another console, connect a telnet client to the simulator:
\begin{script}
$ telnet localhost 1234
\end{script}
\item The output of the test on the telnet console should be:
\begin{script}
Write unlock cycle 1
Write unlock cycle 2
Write write buffer load command
Write write word count (minus 1)
Write data word #0 (0x4321)
Write data word #1 (0x8765)
Write data word #2 (0x9087)
Write data word #3 (0x3852)
Write confirm command
Re-reading and verifying word #0: OK (0x4321)
Re-reading and verifying word #1: OK (0x8765)
Re-reading and verifying word #2: OK (0x9087)
Re-reading and verifying word #3: OK (0x3852)
Test is finished
\end{script}

\end{enumerate}

\subsection{A light weight Linux distribution}

The simulator can easily run a minimalist Linux distribution for a Xilinx Virtex-5 FXT development board.
Table~\ref{table:linux_files} shows the files to load in the simulator to run such a minimalist Linux distribution.
In later sub-sections, we explain how these files are obtained.
We provides you with prebuilt files to allow you quickly boot a minimalist Linux distribution within the simulator.
Go directly to Sub-section~\ref{running_a_linux_distro} if you don't bother about the technique behind creating a light weight Linux distribution for the simulator.
To build such a minimalist Linux distribution, you need a working crosstool chain, see Section~\ref{building_crosstool_chain}.

\begin{table}[!ht]
	\begin{center}
		\begin{tabular}{|l|l|l|l|}
		\hline
		\multicolumn{1}{|c|}{\textbf{File}} & \multicolumn{1}{|c|}{\textbf{Start address}} & \multicolumn{1}{|c|}{\textbf{Memory}} & \multicolumn{1}{|c|}{\textbf{Description}}\\
		\hline
		\texttt{boot.elf} & \texttt{0xffff0000} & BRAM & Boot program\\
		\hline
		\texttt{vmlinux} & \texttt{0x00000000} & RAM & Linux kernel\\
		\hline
		\texttt{device\_tree.dtb} & \texttt{0x00800000} & RAM & Device Tree\\
		\hline
		\texttt{initrd.img} & \texttt{0x00900000} & RAM & Initial RAM disk\\
		\hline
		\end{tabular}
	\end{center}
	\caption{Files for booting Linux in the simulator.}
	\label{table:linux_files}
\end{table}

\subsubsection{The boot program}
\label{boot_program}

The boot program is loaded in BRAM which is behind the MPLB.
At reset, the processor starts executing instructions from physical address \texttt{0xfffffffc}.
Thus the boot program is located in memory so that it has an instruction (usually a branch) at that physical address.
The role of the boot program is to initialize early boot and provides the Kernel with some parameters in the processor registers.
Table~\ref{table:linux_register_parameters} shows the registers that needs to be initialized before branching into the Linux Kernel entry point.

\begin{table}[!ht]
	\begin{center}
		\begin{tabular}{|l|l|}
		\hline
		\multicolumn{1}{|c|}{\textbf{Registers}} & \multicolumn{1}{|c|}{\textbf{Description}}\\
		\hline
		\texttt{r3} & Address of the device tree\\
		\hline
		\texttt{r4} & Start address of the initial RAM disk\\
		\hline
		\texttt{r5} & End address of the initial RAM disk\\
		\hline
		\end{tabular}
	\end{center}
	\caption{Linux kernel register parameters.}
	\label{table:linux_register_parameters}
\end{table}

The boot program, which source is shown on Figure~\ref{fig:boot}, has been designed to make instruction at Label \texttt{start} match the processor start address.
Actually, a branch instructions to elsewhere in the boot program (Label \texttt{init}) is placed at the reset address.
The boot program then starts enabling the MCI (Memory Controller Interface) so that processor can use RAM which is behind the MCI.
It programs the MMU to map the whole 256 MB RAM in the processor address space.
It initializes required register parameters of the Linux kernel, that is the device tree address into Register \texttt{r3}, the initial RAM disk address range into Registers \texttt{r4} and \texttt{r5}, the Linux kernel start address into Register \texttt{SRR0}, and the value of MSR into Register \texttt{SRR1}.
It then branches to the Linux kernel using an \texttt{rfi} instruction (return from interrupt).

\begin{figure}[p]
	\begin{center}
		\input{virtex5fxt/boot}
	\end{center}
	\caption{Boot program (boot.S compiled as boot.elf) loaded in BRAM.}
	\label{fig:boot}
\end{figure}

\subsubsection{The Linux kernel}

\begin{enumerate}
\item Get the Linux kernel source code: Download the \texttt{linux-2.6.38.7.tar.bz2} tarball at \url{ftp://ftp.kernel.org/pub/linux/kernel/v2.6/linux-2.6.38.7.tar.bz2}.
\item Uncompress the tarball
\begin{script}
$ tar jxvf linux-2.6.38.7.tar.bz2
\end{script}
\item Before starting configuring the Linux kernel build, we need to modify a Linux configuration file to enable support of hardware FPU.
In File \texttt{arch/powerpc/platforms/44x/Kconfig}, after:
\begin{script}
 config XILINX_VIRTEX_5_FXT
        bool
        select XILINX_VIRTEX
\end{script}
\noindent add the following line:
\begin{script}
        select PPC_FPU
\end{script}
\item We can now configure the Linux kernel build:
\begin{script}
$ make \
    ARCH=powerpc \
    CROSS_COMPILE=${HOME}/crosstool/powerpc-440fp-linux-gnu/bin/powerpc-440fp-linux-gnu- \
    V=1 menuconfig
\end{script}
The Linux kernel should be configured as shown on Figure~\ref{fig:kernel_configuration}.
Once configuration is finished, at exit don't forget to answer "Yes" when prompted for saving the settings in File \texttt{.config}.
\item The Linux kernel (\texttt{vmlinux}) can now be built.
The build may take tens of minutes.
Do the following at the command prompt:
\begin{script}
$ make \
    ARCH=powerpc \
    CROSS_COMPILE=${HOME}/crosstool/powerpc-440fp-linux-gnu/bin/powerpc-440fp-linux-gnu- \
    V=1 vmlinux
\end{script}
\end{enumerate}


\begin{figure}[!ht]
\begin{center}
\begin{script}
Processor support  --->
        Processor Types (AMCC 44x, 46x or 47x)

General Setup  --->
    [*] Initial RAM filesystem and RAM disk (initramfs/initrd) support

Platform support  --->
    [*] Generic Xilinx Virtex 5 FXT board support

Device Drivers  --->
    [*] Block devices  --->
        (16)    Default number of RAM disks
         (16384) Default RAM disk size (kbytes)

        Character devices  --->
            Serial drivers  --->
                <*> Xilinx uartlite serial port support
                [*]   Support for console on Xilinx uartlite serial port

    [*] GPIO Support  --->
        [*] Xilinx GPIO support

    [*] Watchdog Timer Support  --->
        <*> PowerPC Book-E Watchdog Timer

File systems  --->
    <*> Second extended fs support

Kernel hacking  --->
    [*] Compile the kernel with debug info
\end{script}
\end{center}
\caption{Linux kernel configuration (changes from default only).}
\label{fig:kernel_configuration}
\end{figure}



\subsubsection{The device Tree}

On PowerPC embedded platforms, the Linux kernel uses a hierarchical list of devices, namely a device tree where leaves are devices and non-leaf nodes are buses, bridges and interconnects.
It is a rather detailed machine description to allow the Linux kernel correctly initialize devices and route interrupts to the interrupt routines.
The boot loader or anything else lauching the Linux kernel should provide the Linux kernel with the device tree.
Linux provides a device tree compiler that compile a text description (\texttt{.dts} file) in a loadable binary form (\texttt{.dtb} file).

\subsubsection{The initial RAM disk}
The initial RAM disk file is an file containing an image root file system. Data within the image is formated as an Ext2 file system.
Creating an image requires root privileges because it involves mounting a file system which requires root privileges.

\begin{enumerate}
\item Create an empty image (\texttt{initrd.img}) of 16 MB:
\begin{script}
[root@localhost] $ dd if=/dev/zero of=initrd.img count=16384 bs=1024
\end{script}
\item Bind image to \texttt{/dev/loop0} virtual device:
\begin{script}
[root@localhost] $ losetup /dev/loop0 initrd.img
\end{script}
\item Create an \texttt{Ext2} file system on \texttt{/dev/loop0}:
\begin{script}
[root@localhost] $ mke2fs -m 0 /dev/loop0
\end{script}
\item Create a mount point and mount \texttt{/dev/loop0} on it:
\begin{script}
[root@localhost] $ mkdir /media/initrd
[root@localhost] $ mount /dev/loop0 /mnt/initrd
\end{script}
\item Now you can directly access withing the image accessing directory \texttt{/mnt/initrd}. In directory \texttt{/mnt/initrd}, copy every files and directories you want in your image.
\item Unmount \texttt{/dev/loop0}:
\begin{script}
[root@localhost] $ umount /dev/loop0
\end{script}
\item Unbind image and \texttt{/dev/loop0}
\begin{script}
[root@localhost] $ losetup -d /dev/loop0
\end{script}
\end{enumerate}

Busybox is a light weight shell that includes most of standard UNIX commands. Actually commands are symbolic links to the busybox binary.
Busybox knows which commands to implement looking at its \texttt{argv[0]}. It is usually a good idea to install Busybox in the initial RAM disk image of an Linux-based embedded platform.

\subsubsection{Running the example}
\label{running_a_linux_distro}

A very small Linux distribution based on Busybox together with a configuration file are provided for the simulator.
To boot that small Linux distribution within the simulator, do the following at the command prompt:

\begin{script}
$ cd distro_linux
$ unisim-virtex5fxt-1.0-wfpu-1.0 -c sim_config.xml
\end{script}

The simulator acts as a telnet server to emulate a terminal over the XPS UART Lite. Once started, it waits for a telnet client connection like \texttt{telnet} or \texttt{PuTTY}. To connect the telnet client to the simulator, do the following at the command prompt:

\begin{script}
$ telnet localhost 1234
\end{script}
